% Szglab4
% ===========================================================================
%
\chapter{Részletes tervek}

\thispagestyle{fancy}

\section{Osztályok és metódusok tervei}

\subsection{Osztály1}
\begin{itemize}
\item Felelősség\newline
\comment{Mi az osztály felelőssége. Kb 1 bekezdés. Ha szükséges, akkor state-chart is.}
\item Ősosztályok\newline
\comment{Mely osztályokból származik (öröklési hierarchia)\newline
Legősebb osztály $\rightarrow$ Ősosztály2 $\rightarrow$ Ősosztály3...}
\item Interfészek\newline
\comment{Mely interfészeket valósítja meg.}
\item Attribútumok\newline
\comment{Milyen attribútumai vannak}
	\begin{itemize}
		\item attribútum1: attribútum jellemzése: mire való, láthatósága (UML jelöléssel), típusa
		\item attribútum2: attribútum jellemzése: mire való, láthatósága (UML jelöléssel), típusa
	\end{itemize}
\item Metódusok\newline
\comment{Milyen publikus, protected és privát  metódusokkal rendelkezik. Metódusonként precíz leírás, ha szükséges, activity diagram is  a metódusban megvalósítandó algoritmusról.}
	\begin{itemize}
		\item int foo(Osztály3 o1, Osztály4 o2): metódus leírása, láthatósága (UML jelöléssel)
		\item int bar(Osztály5 o1): metódus leírása, láthatósága (UML jelöléssel)
	\end{itemize}
\end{itemize}

\subsection{Osztály2}
\begin{itemize}
\item Felelősség\newline
\comment{Mi az osztály felelőssége. Kb 1 bekezdés. Ha szükséges, akkor state-chart is.}
\item Ősosztályok\newline
\comment{Mely osztályokból származik (öröklési hierarchia)\newline
Legősebb osztály $\rightarrow$ Ősosztály2 $\rightarrow$ Ősosztály3...}
\item Interfészek\newline
\comment{Mely interfészeket valósítja meg.}
\item Attribútumok\newline
\comment{Milyen attribútumai vannak}
	\begin{itemize}
		\item attribútum1: attribútum jellemzése: mire való, láthatósága (UML jelöléssel), típusa
		\item attribútum2: attribútum jellemzése: mire való, láthatósága (UML jelöléssel), típusa
	\end{itemize}
\item Metódusok\newline
\comment{Milyen publikus, protected és privát  metódusokkal rendelkezik. Metódusonként precíz leírás, ha szükséges, activity diagram is  a metódusban megvalósítandó algoritmusról.}
	\begin{itemize}
		\item int foo(Osztály3 o1, Osztály4 o2): metódus leírása, láthatósága (UML jelöléssel)
		\item int bar(Osztály5 o1): metódus leírása, láthatósága (UML jelöléssel)
	\end{itemize}
\end{itemize}

\section{A tesztek részletes tervei, leírásuk a teszt nyelvén}
[A tesztek részletes tervei alatt meg kell adni azokat a bemeneti adatsorozatokat, amelyekkel a program működése ellenőrizhető. Minden bemenő adatsorozathoz definiálni kell, hogy az adatsorozat végrehajtásától a program mely részeinek, funkcióinak ellenőrzését várjuk és konkrétan milyen eredményekre számítunk, ezek az eredmények hogyan vethetők össze a bemenetekkel.]

\subsection{Áramkörök betöltése}
\begin{itemize}
Minden teszteset elején betöltjük a megfelelő fájból az áramkört. Ezt mindig meg kell tenni, azonban csak egy esetben mutatjuk meg az egyszerűség és átláthatóság kedvéért.

\textbf{Alap áramkör}\newline
loadCircuit test1.ovr\newline
\newline
load successful

\end{itemize}


\subsection{Alap áramkör}
\begin{itemize}
\item Leírás\newline
Olyan áramkör, melyben 2 kapcsolóval állíthatjuk egy ÉS kapu bemeneteit, melyet egy LED jelenít meg.
\item Ellenőrzött funkcionalitás, várható hibahelyek\newline
Ellenőrízzük a kapcsoló helyes váltását,az ÉS kapu kimenetének helyes kiszámítását és a LED működését
\item Bemenet\newline
\newline
{\bf Áramkör létrehozása:}\newline
kapcs1=TOGGLE()\newline
kapcs2=TOGGLE()\newline
es=AND(kapcs1,kapcs2)\newline
led=LED(es)\newline
{\bf Felhasználó ténykedései:}\newline
switch kapcs1
step
switch kapcs2
step
\item Elvárt kimenet\newline
kapcs1: 1\newline
simulation successful\newline
kapcs1: 1\newline
kapcs2: 0\newline
led: 0\newline
kapcs2: 1\newline
simulation successful\newline
kapcs1: 1\newline
kapcs2: 1\newline
led: 1\newline
\end{itemize}

\subsection{MPX-es áramkör}
\begin{itemize}
\item Leírás\newline
Olyan áramkört hozunk létre, melyben egy 7 szegmenses kijelzőt hajtunk meg kapcsolókkal és egy MPX-xel. A 7szegmenses kijelző [2]-[7] bemeneteire kapcsolókat kötünk, a [1] bemenetét egy MPX adja, mely 4 kapcsolóból választja ki az egyiket, tehát egy 4/1 es MPX.
\item Ellenőrzött funkcionalitás, várható hibahelyek\newline
Ellenőrízzük a MPX helyes működését, és a 7 szegmenses kijelzőt. Hiba a MPX kiválasztása során történhet, hogy rossz jelet juttat a kimenetére.
\item Bemenet\newline
\newline
{\bf Áramkör létrehozása:}\newline
inmpx1=TOGGLE()\newline
inmpx2=TOGGLE()\newline
inmpx3=TOGGLE()\newline
inmpx4=TOGGLE()\newline
selmpx1=TOGGLE()\newline
selmpx2=TOGGLE()\newline
mux=MPX(inmpx1,inmpx2,inmpx3,inmpx4,selmpx1,selmpx2)\newline
seg7=TOGGLE()\newline
seg6=TOGGLE()\newline
seg5=TOGGLE()\newline
seg4=TOGGLE()\newline
seg3=TOGGLE()\newline
seg2=TOGGLE()\newline
display=7SEG(mux,seg2,seg3,seg4,seg5,seg6, seg7)\newline

{\bf Felhasználó ténykedései:}\newline
switch inmpx1\newline
switch inmpx3\newline
step\newline
switch selmpx2\newline
switch seg2\newline
step\newline
switch selmpx2\newline
switch selmpx1\newline
step\newline

\item Elvárt kimenet\newline
inmpx1: 1\newline
inmpx3: 1\newline
simulation successful
inmpx1: 1\newline
inmpx2: 0\newline
inmpx3: 1\newline
inmpx4: 0\newline
selmpx1: 0\newline
selmpx2: 0\newline
seg7: 0\newline
seg6: 0\newline
seg5: 0\newline
seg4: 0\newline
seg3: 0\newline
seg2: 0\newline
display: 100000\newline

selmpx2: 1\newline
seg2: 1\newline

simulation successful\newline
inmpx1: 1\newline
inmpx2: 0\newline
inmpx3: 1\newline
inmpx4: 0\newline
selmpx1: 0\newline
selmpx2: 1\newline
seg7: 0\newline
seg6: 0\newline
seg5: 0\newline
seg4: 0\newline
seg3: 0\newline
seg2: 1\newline
display: 010000\newline

selmpx2: 0\newline
selmpx1: 1\newline

simulation successful\newline
inmpx1: 1\newline
inmpx2: 0\newline
inmpx3: 1\newline
inmpx4: 0\newline
selmpx1: 1\newline
selmpx2: 0\newline
seg7: 0\newline
seg6: 0\newline
seg5: 0\newline
seg4: 0\newline
seg3: 0\newline
seg2: 1\newline
display: 110000\newline



\end{itemize}

\subsection{Visszacsatolt stabil áramkör}
\begin{itemize}
\item Leírás\newline
Egy olyan áramkört hozunk létre, melyben egy VAGY kapu szerepel, aminek egyik bemenete egy kapcsoló, kimenetét pedig visszakötjük a második bemenetére, illetve egy csomóponton keresztül egy LED-re is eljuttatjuk.
\item Ellenőrzött funkcionalitás, várható hibahelyek\newline
Ellenőrízzük, hogy az áramkör helyesen stabilnak érzékeli e a kapcsolást, illetve a VAGY kapu helyes működését is ellenőrízzük. Hibát a visszakötés okozhat.
\item Bemenet\newline
\newline
{\bf Áramkör létrehozása:}\newline
kapcs=TOGGLE()\newline
vagy=OR(kapcs,node[2])\newline
node=NODE(vagy,2)\newline
led=LED(node[1])\newline

{\bf Felhasználó ténykedései:}\newline
step\newline
switch kapcs\newline
step

\item Elvárt kimenet\newline
simulation successful\newline
kapcs: 0\newline
led: 0\newline

kapcs: 1\newline

simulation successful\newline
kapcs: 1\newline
led: 1\newline


\end{itemize}



\subsection{Visszacsatolt nem stabil áramkör}
\begin{itemize}
\item Leírás\newline
Egy olyan áramkört hozunk létre, melyben egy ÉS kapu szerepel, aminek egyik bemenete egy kapcsoló, kimenetét pedig visszakötjük egy inverteren keresztül a második bemenetére, illetve egy csomóponton keresztül egy LED-re is eljuttatjuk.
\item Ellenőrzött funkcionalitás, várható hibahelyek\newline
Ellenőrízzük, hogy az áramkör helyesen instabilnak érzékeli e a kapcsolást. Továbbá, hogy a hálózat helyesen egy bizonyos lépésszám után instabillá nyilvánítja e a hálózatot. Hibás működést ez okozhat, tehát ha az áramkör ezt rosszul állapítja meg, és nem jelzi.
\item Bemenet\newline
\newline
{\bf Áramkör létrehozása:}\newline
kapcs=TOGGLE()\newline
inv=INV(node[2])\newline
es=AND(kapcs,inv)\newline
node=NODE(es,2)\newline
led=LED(node[1])\newline

{\bf Felhasználó ténykedései:}\newline
switch kapcs\newline
step\newline

\item Elvárt kimenet\newline
kapcs: 1\newline

simulation failed\newline


\end{itemize}


\subsection{Flip-flop-os áramkör}
\begin{itemize}
\item Leírás\newline
Egy olyan áramkört hozunk létre, melyben egy JK flipflop szerepel, J és K bemenetére kapcsolókat kötünk, órajelét egy jelgenerátorból kapja, és a kimenetét egy oszcilloszkóp kapja meg.
\item Ellenőrzött funkcionalitás, várható hibahelyek\newline
Ellenőrízzük a jelgenerátort, hogy megfelelő jelet ad e ciklikusan, ellenőrízzük a JK flipflop működését, illetve, hogy megfelelelően lép e az órajelre, továbbá ellenőrízzük az oszcilloszkóp helyes működését. Hiba lehetséges a jelgenerátor működésében, a JK flipflop működésében illetve számolásában, és az oszcilloszkóp működésében.
\item Bemenet\newline
\newline
{\bf Áramkör létrehozása:}\newline
j=TOGGLE()\newline
k=TOGGLE()\newline
seqgen=SEQGEN()\newline
jk=FFJK(seqgen,j,k)\newline
led=LED(jk)\newline

{\bf Felhasználó ténykedései:}\newline
%j: kapcs1
%k: kapcs2
switch k\newline
step\newline
step\newline
switch j\newline
step\newline
step\newline
switch j\newline
switch k\newline
step\newline
step\newline

\item Elvárt kimenet\newline
k: 1\newline

simulation successful\newline
j: 0\newline
k: 1\newline
seqgen: 1\newline
led: 0\newline

simulation successful\newline
j: 0\newline
k: 1\newline
seqgen: 0\newline
led: 0\newline

j: 1\newline

simulation successful\newline
j: 1\newline
k: 1\newline
seqgen: 1\newline
led: 1\newline

simulation successful\newline
j: 1\newline
k: 1\newline
seqgen: 0\newline
led: 1\newline

j: 0\newline
k: 0\newline

simultion successful\newline
j: 0\newline
k: 0\newline
seqgen: 1\newline
led: 1\newline

simulation successful\newline
j: 0\newline
k: 0\newline
seqgen: 1\newline
led: 1\newline

\end{itemize}





\subsection{Kompozitos áramkör}
\begin{itemize}
\item Leírás\newline
Egy olyan áramkört valósítunk meg, melyben egy kompozit szerepel. Ez a kompozit egy 2 bites balról tölthető shiftregisztert valósít meg. A kompozitnak két bemenete van egy kapcsoló ami a balról bejövő értéket adja, és egy jelgenerátor, amely az órajelet. Belül 2 D flipflop található összekötve. Az első flipflop kimenetét kiadja a kompozit kimenetén is, és a 2-ik flipflop bemenetére is adja, ezért NODE is kell. Kompozit kimenete a 2 bit és a carry.
\item Ellenőrzött funkcionalitás, várható hibahelyek\newline
Kompozit helyes működését ellenőrízzük.
\item Bemenet\newline
\newline
{\bf Áramkör létrehozása:}\newline
input=TOGGLE() \newline
seqgen=SEQGEN()
\begin{verbatim}
composite SHR(clk,in){		
    nodeclk=NODE(clk,2)		
    d1=FFD(nodeclk[1],in)		
    node1=NODE(d1,2)			
    d2=FFD(nodeclk[2],node1[2])
    node2=NODE(d2,2)			
}(node1[1],node2[2],node2[2])	
\end{verbatim}
myshr=SHR(seqgen,input)\newline
led1=LED(myshr[1])\newline
led2=LED(myshr[2])\newline
ledcarry=LED(myshr[3])\newline

{\bf Felhasználó ténykedései:}\newline
switch input\newline
step\newline
step\newline

switch input\newline
step\newline
step\newline

step\newline
step\newline

\item Elvárt kimenet\newline
input: 1\newline

simulation successful
input: 1\newline
seqgen: 1\newline
led1: 1\newline
led2: 0\newline
led3: 0\newline

simulation successful
input: 1\newline
seqgen: 0\newline
led1: 1\newline
led2: 0\newline
led3: 0\newline

input: 0\newline

simulation successful
input: 0\newline
seqgen: 1\newline
led1: 0\newline
led2: 1\newline
led3: 1\newline

simulation successful
input: 0\newline
seqgen: 0\newline
led1: 0\newline
led2: 1\newline
led3: 1\newline

simulation successful
input: 0\newline
seqgen: 1\newline
led1: 0\newline
led2: 0\newline
led3: 0\newline

simulation successful
input: 0\newline
seqgen: 0\newline
led1: 0\newline
led2: 0\newline
led3: 0\newline



\end{itemize}


\subsection{Kompoziton belüli kompozitos áramkör}
\begin{itemize}
\item Leírás\newline
Egy olyan áramkört hozunk létre melyben egy kompozit szerepel ami egy 4bites shiftregiszter. Ezt shiftregisztert úgy hozzuk létre, hogy a kompoziton belül 2db 2 bites shiftregiszter szerepel mint kompozitok. Kívülről csak a 4 bites shiftregisztert látjuk, ami belül 4 kompozittal jön létre. 4 bit és carry kimeneteket leden jelezzük, míg az input és órajel bemenetét kapcsolóval és jelgenerátorral adjuk.
\item Ellenőrzött funkcionalitás, várható hibahelyek\newline
Leteszteljük, hogy működik e a kompozit elem, ha belül bonyolultabb áramköri hálózat szerepel, egy kompozit, illetve jelen esetben több kompozit.
\item Bemenet\newline
\newline
{\bf Áramkör létrehozása:}\newline
\begin{verbatim}
composite SHR2BIT(clk,in){		
    nodeclk=NODE(clk,2)		
    d1=FFD(nodeclk[1],in)		
    node1=NODE(d1,2)			
    d2=FFD(nodeclk[2],node1[2])
    node2=NODE(d2,2)			
}(node1[1],node2[2],node2[2])	
\end{verbatim}

input=TOGGLE() \newline
seqgen=SEQGEN()

\begin{verbatim}
composite SHR4BIT(clk,in){		
    nodeclk=NODE(clk,2)		
    shr2bit_1=SHR2BIT(nodeclk[1],in)
    shr2bit_2=SHR2BIT(nodeclk[2],shr2bit[3])			
}(shr2bit_2[2],shr2bit_2[1],shr2bit_1[2],shr2bit_1[1],shr2bit_2[3])	
\end{verbatim}

my4bitshr=SHR4BIT(clk,in)\newline
ledbit1=LED(my4bitshr[1])\newline
ledbit2=LED(my4bitshr[2])\newline
ledbit3=LED(my4bitshr[3])\newline
ledbit4=LED(my4bitshr[4])\newline
ledcarry=LED(my4bitshr[5])\newline


{\bf Felhasználó ténykedései:}\newline
switch input\newline
step\newline
step\newline

step\newline
step\newline

switch input\newline
step\newline
step\newline

switch input\newline
step \newline
step \newline


\item Elvárt kimenet\newline
input: 1\newline

simulation successful\newline
input: 1\newline
seqgen: 1\newline
ledbit1: 1\newline
ledbit2: 0\newline
ledbit3: 0\newline
ledbit3: 0\newline
ledcarry: 0\newline

simulation successful\newline
input: 1\newline
seqgen: 0\newline
ledbit1: 1\newline
ledbit2: 0\newline
ledbit3: 0\newline
ledbit3: 0\newline
ledcarry: 0\newline

simulation successful\newline
input: 1\newline
seqgen: 1\newline
ledbit1: 1\newline
ledbit2: 1\newline
ledbit3: 0\newline
ledbit3: 0\newline
ledcarry: 0\newline

simulation successful\newline
input: 1\newline
seqgen: 0\newline
ledbit1: 1\newline
ledbit2: 1\newline
ledbit3: 0\newline
ledbit3: 0\newline
ledcarry: 0\newline

input: 0\newline

simulation successful\newline
input: 0\newline
seqgen: 1\newline
ledbit1: 0\newline
ledbit2: 1\newline
ledbit3: 1\newline
ledbit3: 0\newline
ledcarry: 0\newline

simulation successful\newline
input: 0\newline
seqgen: 0\newline
ledbit1: 0\newline
ledbit2: 1\newline
ledbit3: 1\newline
ledbit3: 0\newline
ledcarry: 0\newline

input: 1\newline

simulation successful\newline
input: 1\newline
seqgen: 1\newline
ledbit1: 1\newline
ledbit2: 0\newline
ledbit3: 1\newline
ledbit3: 1\newline
ledcarry: 0\newline

simulation successful\newline
input: 1\newline
seqgen: 0\newline
ledbit1: 1\newline
ledbit2: 0\newline
ledbit3: 1\newline
ledbit3: 1\newline
ledcarry: 0\newline


\end{itemize}

\section{A tesztelést támogató programok tervei}
\comment{A tesztadatok előállítására, a tesztek eredményeinek kiértékelésére szolgáló segédprogramok részletes terveit kell elkészíteni.}

