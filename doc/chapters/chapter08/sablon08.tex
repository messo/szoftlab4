% Szglab4
% ===========================================================================
%
\chapter{Részletes tervek}

\thispagestyle{fancy}

\section{Osztályok és metódusok tervei}

\subsection{Osztály1}
\begin{itemize}
\item Felelősség\newline
\comment{Mi az osztály felelőssége. Kb 1 bekezdés. Ha szükséges, akkor state-chart is.}
\item Ősosztályok\newline
\comment{Mely osztályokból származik (öröklési hierarchia)\newline
Legősebb osztály $\rightarrow$ Ősosztály2 $\rightarrow$ Ősosztály3...}
\item Interfészek\newline
\comment{Mely interfészeket valósítja meg.}
\item Attribútumok\newline
\comment{Milyen attribútumai vannak}
	\begin{itemize}
		\item attribútum1: attribútum jellemzése: mire való, láthatósága (UML jelöléssel), típusa
		\item attribútum2: attribútum jellemzése: mire való, láthatósága (UML jelöléssel), típusa
	\end{itemize}
\item Metódusok\newline
\comment{Milyen publikus, protected és privát  metódusokkal rendelkezik. Metódusonként precíz leírás, ha szükséges, activity diagram is  a metódusban megvalósítandó algoritmusról.}
	\begin{itemize}
		\item int foo(Osztály3 o1, Osztály4 o2): metódus leírása, láthatósága (UML jelöléssel)
		\item int bar(Osztály5 o1): metódus leírása, láthatósága (UML jelöléssel)
	\end{itemize}
\end{itemize}

\subsection{Osztály2}
\begin{itemize}
\item Felelősség\newline
\comment{Mi az osztály felelőssége. Kb 1 bekezdés. Ha szükséges, akkor state-chart is.}
\item Ősosztályok\newline
\comment{Mely osztályokból származik (öröklési hierarchia)\newline
Legősebb osztály $\rightarrow$ Ősosztály2 $\rightarrow$ Ősosztály3...}
\item Interfészek\newline
\comment{Mely interfészeket valósítja meg.}
\item Attribútumok\newline
\comment{Milyen attribútumai vannak}
	\begin{itemize}
		\item attribútum1: attribútum jellemzése: mire való, láthatósága (UML jelöléssel), típusa
		\item attribútum2: attribútum jellemzése: mire való, láthatósága (UML jelöléssel), típusa
	\end{itemize}
\item Metódusok\newline
\comment{Milyen publikus, protected és privát  metódusokkal rendelkezik. Metódusonként precíz leírás, ha szükséges, activity diagram is  a metódusban megvalósítandó algoritmusról.}
	\begin{itemize}
		\item int foo(Osztály3 o1, Osztály4 o2): metódus leírása, láthatósága (UML jelöléssel)
		\item int bar(Osztály5 o1): metódus leírása, láthatósága (UML jelöléssel)
	\end{itemize}
\end{itemize}

\section{A tesztek részletes tervei, leírásuk a teszt nyelvén}
\comment{A tesztek részletes tervei alatt meg kell adni azokat a bemeneti adatsorozatokat, amelyekkel a program működése ellenőrizhető. Minden bemenő adatsorozathoz definiálni kell, hogy az adatsorozat végrehajtásától a program mely részeinek, funkcióinak ellenőrzését várjuk és konkrétan milyen eredményekre számítunk, ezek az eredmények hogyan vethetők össze a bemenetekkel.}


\subsection{Alap áramkör}
\begin{itemize}
\item Leírás\newline
Olyan áramkör, melyben 2 kapcsolóval állíthatjuk egy ÉS kapu bemeneteit, melyet egy LED jelenít meg.
\item Ellenőrzött funkcionalitás, várható hibahelyek\newline
Ellenőrizzük a kapcsoló helyes váltását, az ÉS kapu kimenetének helyes kiszámítását és a LED működését
\item Áramkör létrehozása

\begin{verbatim}
kapcs1=TOGGLE()
kapcs2=TOGGLE()
es=AND(kapcs1,kapcs2)
led=LED(es)
\end{verbatim}

\item Bemenet és kimenet\newline

\begin{tabular}{|p{7cm}|p{7cm}|} 
\hline 
\textit{Bemenet} & \textit{Kimenet} \\ \hline
\begin{verbatim}
loadCircuit test1.ovr
step
switch kapcs1
step
check -all
switch kapcs2
step
\end{verbatim}
& 
\begin{verbatim}
simulation successful
kapcs1: 0
kapcs2: 0
led: 0

kapcs1: 1

simulation successful
kapcs1: 1
kapcs2: 0
led: 0

led:
 in: 0
 out: 
kapcs1:
 in: 
 out: 1
kapcs2:
 in: 
 out: 0
es:
 in: 1, 0
 out: 0
 
kapcs2: 1

simulation successful
kapcs1: 1
kapcs2: 1
led: 1
\end{verbatim}
\\ \hline
\end{tabular}

\end{itemize}

\subsection{MPX-es áramkör}
\begin{itemize}
\item Leírás\newline
Olyan áramkört hozunk létre, melyben egy 7 szegmenses kijelzőt hajtunk meg kapcsolókkal és egy MPX-xel. A 7szegmenses kijelző [2]-[7] bemeneteire kapcsolókat kötünk, a [1] bemenetét egy MPX adja, mely 4 kapcsolóból választja ki az egyiket, tehát egy 4/1-es MPX.
\item Ellenőrzött funkcionalitás, várható hibahelyek\newline
Ellenőrizzük a MPX helyes működését, és a 7 szegmenses kijelzőt. Hiba a MPX kiválasztása során történhet, hogy rossz jelet juttat a kimenetére.

\item Áramkör létrehozása

\begin{verbatim}
inmpx1=TOGGLE()
inmpx2=TOGGLE()
inmpx3=TOGGLE()
inmpx4=TOGGLE()
selmpx1=TOGGLE()
selmpx2=TOGGLE()
mux=MPX(inmpx1,inmpx2,inmpx3,inmpx4,selmpx1,selmpx2)
seg=TOGGLE()
display=7SEG(mux,seg,0,0,0,0,0)
\end{verbatim}

\item Bemenet és kimenet\newline

\begin{tabular}{|p{7cm}|p{7cm}|} 
\hline 
\textit{Bemenet} & \textit{Kimenet} \\ \hline
\begin{verbatim}
loadCircuit test2.ovr
switch inmpx1
switch inmpx3
step
switch selmpx2
switch seg2
step
switch selmpx2
switch selmpx1
step
\end{verbatim}
& 
\begin{verbatim}
load successful

inmpx1: 1

inmpx3: 1

simulation successful
inmpx1: 1
inmpx2: 0
inmpx3: 1
inmpx4: 0
selmpx1: 0
selmpx2: 0
seg: 0
display: 1, 0, 0, 0, 0, 0, 0

selmpx2: 1

seg: 1

simulation successful
inmpx1: 1
inmpx2: 0
inmpx3: 1
inmpx4: 0
selmpx1: 0
selmpx2: 1
seg: 1
display: 1, 1, 0, 0, 0, 0, 0

selmpx2: 0

selmpx1: 1

simulation successful
inmpx1: 1
inmpx2: 0
inmpx3: 1
inmpx4: 0
selmpx1: 1
selmpx2: 0
seg: 1
display: 0, 1, 0, 0, 0, 0, 0

\end{verbatim}
\\ \hline
\end{tabular}

\end{itemize}

\subsection{Visszacsatolt stabil áramkör}
\begin{itemize}
\item Leírás\newline
Egy olyan áramkört hozunk létre, melyben egy VAGY kapu szerepel, aminek egyik bemenete egy kapcsoló, kimenetét pedig visszakötjük a második bemenetére, illetve egy csomóponton keresztül egy LED-re is eljuttatjuk.
\item Ellenőrzött funkcionalitás, várható hibahelyek\newline
Ellenőrizzük, hogy az áramkör helyesen stabilnak érzékeli-e a kapcsolást, illetve a VAGY kapu helyes működését is ellenőrizzük. Hibát a visszakötés okozhat.

\item Áramkör létrehozása

\begin{verbatim}
kapcs=TOGGLE()
vagy=OR(kapcs,node[2])
node=NODE(vagy,2)
led=LED(node[1])
\end{verbatim}

\item Bemenet és kimenet\newline

\begin{tabular}{|p{7cm}|p{7cm}|} 
\hline 
\textit{Bemenet} & \textit{Kimenet} \\ \hline
\begin{verbatim}
loadCircuit test3.ovr
step
switch kapcs
step
\end{verbatim}
& 
\begin{verbatim}
load successful

simulation successful
kapcs: 0
led: 0

kapcs: 1

simulation successful
kapcs: 1
led: 1
\end{verbatim}
\\ \hline
\end{tabular}

\end{itemize}


\subsection{Visszacsatolt nem stabil áramkör}
\begin{itemize}
\item Leírás\newline
Egy olyan áramkört hozunk létre, melyben egy ÉS kapu szerepel, aminek egyik bemenete egy kapcsoló, kimenetét pedig visszakötjük egy inverteren keresztül a második bemenetére, illetve egy csomóponton keresztül egy LED-re is eljuttatjuk.
\item Ellenőrzött funkcionalitás, várható hibahelyek\newline
Ellenőrizzük, hogy az áramkör helyesen instabilnak érzékeli-e a kapcsolást. Hibás működést ez okozhat, tehát ha az áramkör ezt rosszul állapítja meg, és nem jelzi.

\item Áramkör létrehozása

\begin{verbatim}
kapcs=TOGGLE()
inv=INV(node[2])
es=AND(kapcs,inv)
node=NODE(es,2)
led=LED(node[1])
\end{verbatim}

\item Bemenet és kimenet\newline

\begin{tabular}{|p{7cm}|p{7cm}|} 
\hline 
\textit{Bemenet} & \textit{Kimenet} \\ \hline
\begin{verbatim}
loadCircuit test4.ovr
switch kapcs
step
\end{verbatim}
& 
\begin{verbatim}
load successful

kapcs: 1

simulation failed
\end{verbatim}
\\ \hline
\end{tabular}

\end{itemize}


\subsection{Flip-flop-os áramkör}
\begin{itemize}
\item Leírás\newline
Egy olyan áramkört hozunk létre, melyben egy JK flipflop szerepel, J és K bemenetére kapcsolókat kötünk, órajelét egy jelgenerátorból kapja, és a kimenetét egy oszcilloszkóp kapja meg.
\item Ellenőrzött funkcionalitás, várható hibahelyek\newline
Ellenőrizzük a jelgenerátort, hogy megfelelő jelet adja-e ciklikusan, ellenőrizzük a JK flipflop működését, illetve, hogy megfelelelően lép-e az órajelre, továbbá ellenőrizzük, hogy az oszcilloszkóp helyesen működik-e. Hiba lehetséges a jelgenerátor működésében, a JK flipflop működésében illetve számolásában, és az oszcilloszkóp működésében.

\item Áramkör létrehozása

\begin{verbatim}
j=TOGGLE()
k=TOGGLE()
seqgen=SEQGEN()
jk=FFJK(seqgen,j,k)
scope=SCOPE(jk, 3)
\end{verbatim}

\item Bemenet és kimenet\newline

\begin{tabular}{|p{7cm}|p{7cm}|} 
\hline 
\textit{Bemenet} & \textit{Kimenet} \\ \hline
\begin{verbatim}
loadCircuit test5.ovr
switch k
step
step
switch j
step
step
switch j
switch k
step
step
\end{verbatim}
& 
\begin{verbatim}
load successful

k: 1

simulation successful
j: 0
k: 1
seqgen: 0
scope: 0

simulation successful
j: 0
k: 1
seqgen: 1
scope: 00

j: 1

simulation successful
j: 1
k: 1
seqgen: 0
scope: 000

simulation successful
j: 1
k: 1
seqgen: 1
scope: 001

j: 0

k: 0

simulation successful
j: 0
k: 0
seqgen: 0
scope: 011

simulation successful
j: 0
k: 0
seqgen: 1
scope: 111
\end{verbatim}
\\ \hline
\end{tabular}

\end{itemize}


\subsection{Kompozitos áramkör}
\begin{itemize}
\item Leírás\newline
Egy olyan áramkört valósítunk meg, melyben egy kompozit szerepel. Ez a kompozit egy 2 bites balról tölthető shiftregisztert valósít meg. A kompozitnak két bemenete van egy kapcsoló ami a balról bejövő értéket adja, és egy jelgenerátor, amely az órajelet. Belül 2 db D flipflop található összekötve. Az első flipflop kimenetét kiadja a kompozit kimenetén is, és a második flipflop bemenetére is ráadja, ezért NODE is kell. A kompozit kimenete a 2 bit és a carry.
\item Ellenőrzött funkcionalitás, várható hibahelyek\newline
Kompozit helyes működését ellenőrizzük.

\item Áramkör létrehozása

\begin{verbatim}
input=TOGGLE()
seqgen=SEQGEN()
composite SHR(clk, in){
    in2 = NODE(in, 1)
    d1 = FFD(clk, in)
    node1 = NODE(d1,2)
    d2 = FFD(clk,node1[1])
} (in2, node1[2], d2)
myshr = SHR(seqgen, input)
led1=LED(myshr[1])
led2=LED(myshr[2])
ledcarry=LED(myshr[3])
\end{verbatim}

\item Bemenet és kimenet\newline

\begin{tabular}{|p{7cm}|p{7cm}|} 
\hline 
\textit{Bemenet} & \textit{Kimenet} \\ \hline
\begin{verbatim}
loadCircuit test6.ovr
switch input
step
step
switch input
step
step
step
step
\end{verbatim}
& 
\begin{verbatim}
load successful

input: 1

simulation successful
input: 1
seqgen: 0
led1: 1
led2: 0
ledcarry: 0

simulation successful
input: 1
seqgen: 1
led1: 1
led2: 0
ledcarry: 0

input: 0

simulation successful
input: 0
seqgen: 0
led1: 0
led2: 1
ledcarry: 0

simulation successful
input: 0
seqgen: 1
led1: 0
led2: 1
ledcarry: 0

simulation successful
input: 0
seqgen: 0
led1: 0
led2: 0
ledcarry: 1

simulation successful
input: 0
seqgen: 1
led1: 0
led2: 0
ledcarry: 1
\end{verbatim}
\\ \hline
\end{tabular}

\end{itemize}


\subsection{Kompoziton belüli kompozitos áramkör}
\begin{itemize}
\item Leírás\newline
Egy olyan áramkört hozunk létre melyben egy kompozit szerepel, ami egy másik kompozitot foglal magába. A belső kompozit egyetlen invertert tartalmaz. A külső kompozit tartalmaz még egy VAGY kaput, melynek egyik bementére a belső kompozit kimenetét, másik bemenetére pedig a külső kompozit bemenetére érkező jelet kötjük. A külső kompozit bemenetére egy kapcsolót, kimenetére egy LED-et kötünk.
\item Ellenőrzött funkcionalitás, várható hibahelyek\newline
Leteszteljük, hogy működik-e a kompozit elem, ha belül bonyolultabb áramköri hálózat szerepel, jelen esetben egy kompozit, illetve egy VAGY kapu.

\item Áramkör létrehozása

\begin{verbatim}
tog = TOGGLE()
composite innerComp(in){
inv = INV(in)
}(inv)
composite Main(in){
inC = innerComp(in)
or = OR(in, inC)
}(or)
m = Main(tog)
led = LED(m)
\end{verbatim}

\item Bemenet és kimenet\newline

\begin{tabular}{|p{7cm}|p{7cm}|} 
\hline 
\textit{Bemenet} & \textit{Kimenet} \\ \hline
\begin{verbatim}
loadCircuit test7.ovr
step
step
switch tog
step
step
\end{verbatim}
& 
\begin{verbatim}
load successful

simulation successful
tog: 0
led: 1

simulation successful
tog: 0
led: 1

tog: 1

simulation successful
tog: 1
led: 1

simulation successful
tog: 1
led: 1
\end{verbatim}
\\ \hline
\end{tabular}

\end{itemize}


\section{A tesztelést támogató programok tervei}
\comment{A tesztadatok előállítására, a tesztek eredményeinek kiértékelésére szolgáló segédprogramok részletes terveit kell elkészíteni.}

