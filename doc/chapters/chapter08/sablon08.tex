% Szglab4
% ===========================================================================
%
\chapter{Részletes tervek}

\thispagestyle{fancy}

\section{Osztályok és metódusok tervei}

\subsection{Config}
\begin{itemize}
\item Felelősség\\
Konfigurációs fájlok kezelése, azok írása az áramkör alapján, illetve azok betöltése  az áramkörbe.
\item Ősosztályok\ Object $\rightarrow{}$ Config.
\item Interfészek (nincs)
\item Attribútumok $\ $
\begin{itemize}
	\item[] \texttt{$-$ Circuit circuit} Áramkör, aminek mentjük a dolgait
	\item[] \texttt{$-$ Pattern \underline{sourceComponentPattern}} Regex kifejezés az illesztéshez (beolvasásnál)
\end{itemize}
\item Metódusok$\ $
\begin{itemize}
	\item[] \texttt{$+$ Config(Circuit circuit)}: Példány létrehozása az áramkörhöz.
	\item[] \texttt{$+$ int load(File file)}: Betölti egy fájlból a kapcsolók illetve jelgenerátorok konfigurációját
	\item[] \texttt{$+$ int save(File file)}: Elmenti a kapcsolók illetve jelgenerátorok aktuális állapotát egy fájlba
\end{itemize}
\end{itemize}

\subsection{Controller}
Interfész.
\begin{itemize}
\item Felelősség\\
Kontroller interfész.
\item Ősosztályok\ Controller.
\item Interfészek (nincs)
\item Metódusok$\ $
\begin{itemize}
	\item[] \texttt{$+$ void run(BufferedReader input)}: Vezérlés elindítása adott bemenetről.
\end{itemize}
\end{itemize}

\subsection{Parser}
\begin{itemize}
\item Felelősség\\
Áramkör értelmező objektum, feladata, hogy a paraméterként átadott, illetve  fájlban elhelyezett komponenseket értelmezze, a kapcsolatokat feltérképezze,  elvégezze az összeköttetéseket, és ezáltal felépítse az áramkört.
\item Ősosztályok\ Object $\rightarrow{}$ Parser.
\item Interfészek (nincs)
\item Attribútumok $\ $
\begin{itemize}
	\item[] \texttt{$-$ Circuit circuit} A leíróból létrehozott áramkör.
	\item[] \texttt{$-$ Pattern \underline{componentPattern}} Regex minta egy komponens-sor feldolgozásához
	\item[] \texttt{$-$ Pattern \underline{compositeEndPattern}} Regex minta egy kompozit véghez
	\item[] \texttt{$-$ Map composites} Komponensek listája név szerint.
	\item[] \texttt{$-$ Pattern \underline{compositeStartPattern}} Regex minta egy kompozit kezdethez
	\item[] \texttt{$-$ Map parameters} Kompozitokban lévő komponensek paraméter listája.
\end{itemize}
\item Metódusok$\ $
\begin{itemize}
	\item[] \texttt{$+$ Parser()}: 
 % TODO
	\item[] \texttt{$-$ void addComponentsToComposite(Composite composite, List lines, String[] inputs, String[] outputs)}: Komponens hozzáadása a kompozithoz
	\item[] \texttt{$-$ void parse(BufferedReader br)}: Bementről feldolgozás
	\item[] \texttt{$+$ Circuit parse(File file)}: Létrehoz egy áramkört a megadott fájlból
	\item[] \texttt{$-$ AbstractComponent parseComponentFromLine(Matcher matcher, Composite composite)}: Egy komponens-sor feldolgozása a fájlban
\end{itemize}
\end{itemize}

\subsection{Proto}
\begin{itemize}
\item Felelősség\\
Prototípus vezérlő osztálya.
\item Ősosztályok\ Object $\rightarrow{}$ Proto.
\item Interfészek Controller.
\item Attribútumok $\ $
\begin{itemize}
	\item[] \texttt{$-$ Circuit c} Áramkör
	\item[] \texttt{$-$ Config config} Konfiguráció menedzselése
	\item[] \texttt{$-$ Simulation s} Szimuláció
	\item[] \texttt{$-$ Viewable view} Megjelenítő
\end{itemize}
\item Metódusok$\ $
\begin{itemize}
	\item[] \texttt{$+$ Proto()}: 
 % TODO
	\item[] \texttt{$-$ void eval(String s)}: Parancs értelmezése
	\item[] \texttt{$+$ static void main(String[] args)}: Program belépési pontja.
	\item[] \texttt{$+$ void run(BufferedReader input)}: Felhasználó parancsait olvassa
\end{itemize}
\end{itemize}

\subsection{View}
\begin{itemize}
\item Felelősség\\
Egy konkrét kimeneti implementáció, mely OutputStreamWriter-be ír ki,  így a konzolos megjelenítés és fájlba írás megoldott.
\item Ősosztályok\ Object $\rightarrow{}$ View.
\item Interfészek Viewable.
\item Attribútumok $\ $
\begin{itemize}
	\item[] \texttt{$-$ Controller controller} Kontroller
	\item[] \texttt{$-$ PrintWriter out} Kimeneti adatfolyam, ide írunk.
\end{itemize}
\item Metódusok$\ $
\begin{itemize}
	\item[] \texttt{$+$ View(Controller c, OutputStreamWriter out)}: Létehozzuk a Viewt egy kontrollerrel és a kimenettel, ide fog menni a kimenet.
	\item[] \texttt{$+$ void newline()}: Új sor a kimeneten
	\item[] \texttt{$+$ void writeDetails(AbstractComponent ac)}: Kiírunk egy komponenst (be és kimenetek)
	\item[] \texttt{$+$ void writeLedValue(Led led)}: Kiírja a led értékét
	\item[] \texttt{$+$ void writeLoadFailed()}: Kiírjuk, hogy a betöltés sikertelen
	\item[] \texttt{$+$ void writeLoadSuccessful()}: Kiírjuk, hogy betöltés sikeres
	\item[] \texttt{$+$ void writeSaveFailed()}: Kiírjuk, hogy a config fájl sikertelen
	\item[] \texttt{$+$ void writeSaveSuccessful()}: Kiírjuk, hogy a config fájl mentés sikeres
	\item[] \texttt{$+$ void writeScopeDetails(Scope ac)}: Kiírunk egy scope-ot
	\item[] \texttt{$+$ void writeScopeValues(Scope scope)}: Kiírja a scope által tárolt értékeket
	\item[] \texttt{$+$ void writeSequenceGenerator(SequenceGenerator sg)}: Szekvenciagenerátor szekvenciájának kiírása
	\item[] \texttt{$+$ void writeSequenceGeneratorSequence(SequenceGenerator sg)}: Kiírja a jelgenerátor szekvenciáját
	\item[] \texttt{$+$ void writeSequenceGeneratorValue(SequenceGenerator sg)}: Kiírja a jelgenerátor éppen kiadott értékét
	\item[] \texttt{$+$ void writeSevenSegmentDisplayValues(SevenSegmentDisplay seg)}: Kiírja a 7-szegmentes kijelző szegmenseit.
	\item[] \texttt{$+$ void writeSimulationFailed()}: Kiírjuk, hogy a szimuláció sikertelen
	\item[] \texttt{$+$ void writeSimulationSuccessful()}: Kiírjuk, hogy a szimuláció sikeres
	\item[] \texttt{$+$ void writeToggleValue(Toggle sc)}: Kiírja a kapcsoló állapotát
\end{itemize}
\end{itemize}

\subsection{Viewable}
Interfész.
\begin{itemize}
\item Felelősség\\
A kimenet interfésze.
\item Ősosztályok\ Viewable.
\item Interfészek (nincs)
\item Metódusok$\ $
\begin{itemize}
	\item[] \texttt{$+$ void newline()}: Új sor a kimeneten
	\item[] \texttt{$+$ void writeDetails(AbstractComponent ac)}: Kiírunk egy komponenst (be és kimenetek)
	\item[] \texttt{$+$ void writeLedValue(Led led)}: Kiírja a led értékét
	\item[] \texttt{$+$ void writeLoadFailed()}: Kiírjuk, hogy a betöltés sikertelen
	\item[] \texttt{$+$ void writeLoadSuccessful()}: Kiírjuk, hogy betöltés sikeres
	\item[] \texttt{$+$ void writeSaveFailed()}: Kiírjuk, hogy a config fájl sikertelen
	\item[] \texttt{$+$ void writeSaveSuccessful()}: Kiírjuk, hogy a config fájl mentés sikeres
	\item[] \texttt{$+$ void writeScopeDetails(Scope scope)}: Kiírunk egy scope-ot
	\item[] \texttt{$+$ void writeScopeValues(Scope scope)}: Kiírja a scope által tárolt értékeket
	\item[] \texttt{$+$ void writeSequenceGenerator(SequenceGenerator sg)}: Szekvenciagenerátor szekvenciájának kiírása
	\item[] \texttt{$+$ void writeSequenceGeneratorSequence(SequenceGenerator sg)}: Kiírja a jelgenerátor szekvenciáját
	\item[] \texttt{$+$ void writeSequenceGeneratorValue(SequenceGenerator sg)}: Kiírja a jelgenerátor éppen kiadott értékét
	\item[] \texttt{$+$ void writeSevenSegmentDisplayValues(SevenSegmentDisplay seg)}: Kiírja a 7-szegmentes kijelző szegmenseit.
	\item[] \texttt{$+$ void writeSimulationFailed()}: Kiírjuk, hogy a szimuláció sikertelen
	\item[] \texttt{$+$ void writeSimulationSuccessful()}: Kiírjuk, hogy a szimuláció sikeres
	\item[] \texttt{$+$ void writeToggleValue(Toggle toggle)}: Kiírja a kapcsoló állapotát
\end{itemize}
\end{itemize}


\subsection{Circuit}
\begin{itemize}
\item Felelősség\\
Áramkört reprezentáló osztály, igazából egy kompozit. Felelőssége megegyzik a kompozitéval.
\item Ősosztályok\ Object $\rightarrow{}$ AbstractComponent $\rightarrow{}$ Composite $\rightarrow{}$ Circuit.
\item Interfészek (nincs)
\item Attribútumok $\ $
\begin{itemize}
\item (nincs)
\end{itemize}
\item Metódusok$\ $
\begin{itemize}
	\item[] \texttt{$+$ Circuit()}: 
 % TODO
\end{itemize}
\end{itemize}

\subsection{Simulation}
\begin{itemize}
\item Felelősség\\
Egy szimulációt reprezentáló objektum.  Utasítja az áramkört, hogy értékelje ki magát. Ha az áramkör azt jelzi magáról,  hogy nincs stacionárius állapota akkor jelezzük a felhasználónak.
\item Ősosztályok\ Object $\rightarrow{}$ Simulation.
\item Interfészek (nincs)
\item Attribútumok $\ $
\begin{itemize}
	\item[] \texttt{$\#$ Circuit circuit} Szimulált áramkör
\end{itemize}
\item Metódusok$\ $
\begin{itemize}
	\item[] \texttt{$+$ Simulation()}: 
 % TODO
	\item[] \texttt{$+$ void setCircuit(Circuit circuit)}: Szimulált áramkör beállítása
	\item[] \texttt{$+$ boolean start()}: Egy adott bemeneti kombinációkra kiértékeli a hálózatot.
\end{itemize}
\end{itemize}

\subsection{Value}
\begin{itemize}
\item Felelősség\\
Az áramkörben előfordulható értéket reprezentál.
\item Ősosztályok\ Object $\rightarrow{}$ Enum $\rightarrow{}$ Value.
\item Interfészek (nincs)
\item Attribútumok $\ $
\begin{itemize}
	\item[] \texttt{$+$ final Value \underline{FALSE}} 
 % TODO
	\item[] \texttt{$+$ final Value \underline{TRUE}} 
 % TODO
\end{itemize}
\item Metódusok$\ $
\begin{itemize}
	\item[] \texttt{$-$ Value()}: 
 % TODO
	\item[] \texttt{$+$ Value invert()}: Érték invertálása
	\item[] \texttt{$+$ static Value valueOf(String name)}: 
 % TODO
	\item[] \texttt{$+$ static Value[] values()}: 
 % TODO
\end{itemize}
\end{itemize}


\subsection{AbstractComponent}
Absztrakt osztály.
\begin{itemize}
\item Felelősség\\
Egy komponens absztrakt megvalósítása, ebből származik az összes többi  komponens. A közös logikát valósítja meg. A gyakran használt dolgokra  ad alapértelmezett implementációt (kimenetekre és bemenetekre kötés, kiértékelés stb.)
\item Ősosztályok\ Object $\rightarrow{}$ AbstractComponent.
\item Interfészek (nincs)
\item Attribútumok $\ $
\begin{itemize}
	\item[] \texttt{$-$ boolean changed} Változott-e a komponens kimenete
	\item[] \texttt{$\#$ Wire[] inputs} Bemenetekre kötött vezetékek
	\item[] \texttt{$\#$ String name} Komponens neve
	\item[] \texttt{$\#$ Wire[] outputs} Kimenetekre kötött vezetékek
\end{itemize}
\item Metódusok$\ $
\begin{itemize}
	\item[] \texttt{$\#$ AbstractComponent(String name, int inputCount, int outputCount)}: Konstruktor
	\item[] \texttt{$+$ void addTo(Composite composite)}: Komponens hozzáadása az áramkörhöz
	\item[] \texttt{$+$ abstract AbstractComponent copy(String newName)}: Lemásoljuk a komponenst.
	\item[] \texttt{$+$ void evaluate()}: Komponens kimeneti lábain lévő vezetékeken lévő értékek újraszámolása  a bemenetek alapján.
	\item[] \texttt{$\#$ Value getInput(int inputPin)}: Lekérjük egy adott bemenetre kötött értéket
	\item[] \texttt{$+$ int getInputsCount()}: Bemeneti lábak száma
	\item[] \texttt{$+$ Wire getInputWire(int inputPin)}: Lekérünk egy bemeneti lábon lévő vezetéket
	\item[] \texttt{$+$ String getName()}: Komponens nevének lekérése.
	\item[] \texttt{$+$ int getOutputsCount()}: Kimeneti lábak száma
	\item[] \texttt{$+$ Wire getOutputWire(int outputPin)}: Lekérünk egy kimeneti lábon lévő vezetéket
	\item[] \texttt{$+$ boolean isChanged()}: Visszaadja, hogy a komponensünk kimeneti értéke változott-e a kiértékelés során
	\item[] \texttt{$\#$ abstract void onEvaluation()}: Ebben a metódusban kell implementálni az alkatrész logikáját, vagyis  az adott bemenet(ek) függvényében mit kell kiadnia a kimenet(ek)re.
	\item[] \texttt{$+$ void setInput(int inputPin, Wire wire)}: Beállítunk egy bemenetet
	\item[] \texttt{$+$ void setOutput(int outputPin, Wire wire)}: Beállítunk egy kimenetet
	\item[] \texttt{$+$ void writeTo(Viewable view)}: Komponens kiírása a viewra.
	\item[] \texttt{$+$ void writeValueTo(Viewable view)}: Kiírja az értékét a viewra (csak kijelző és forrásra!)
\end{itemize}
\end{itemize}

\subsection{Composite}
\begin{itemize}
\item Felelősség\\
Kompozit elem leírása, kiértékelésnél a tartalmazott komponenseket kiértékeli, lépteti  a jelgenerátorokat stb. Ha nem áll be stacionárius állapotba a kiértékelésnél, akkor ezt jelzi kifelé.
\item Ősosztályok\ Object $\rightarrow{}$ AbstractComponent $\rightarrow{}$ Composite.
\item Interfészek (nincs)
\item Attribútumok $\ $
\begin{itemize}
	\item[] \texttt{$-$ Map components} Komponensek listája
	\item[] \texttt{$-$ List composites} Kompozit típusú komponensek listája
	\item[] \texttt{$-$ final int \underline{cycleLimit}} Max. ciklusok száma
	\item[] \texttt{$-$ List displays} Megjelenítő típusú komponensek listája (pl. led)
	\item[] \texttt{$-$ List flipFlops} Flipflopok listája
	\item[] \texttt{$-$ List generators} Jelgenerátorok listája
	\item[] \texttt{$-$ Node[] inputNodes} Bemeneti csomópontok
	\item[] \texttt{$-$ Pattern \underline{inputPattern}} Regex minta egy komponens bemeneteinek a feldolgozásához
	\item[] \texttt{$-$ Node[] outputNodes} Kimeneti csomópontok
	\item[] \texttt{$-$ List scopes} Oszcillátor típusú komponensek listája
	\item[] \texttt{$-$ List sources} Jelforrás típusú komponensek listája (pl. kapcsoló)
	\item[] \texttt{$-$ String type} Kompozit típusa
\end{itemize}
\item Metódusok$\ $
\begin{itemize}
	\item[] \texttt{$+$ Composite(String type, String name, int inputCount, int outputCount)}: Adott típusú és nevű komponens létrehozása a megfelelő lábszámmal.
	\item[] \texttt{$+$ void add(AbstractComponent c)}: Általános típusú komponens hozzáadása
	\item[] \texttt{$+$ void add(Composite c)}: Kompozit típusú komponens hozzáadása
	\item[] \texttt{$+$ void add(DisplayComponent dc)}: Kijelző típusú komponens hozzáadása
	\item[] \texttt{$+$ void add(FlipFlop ff)}: Flipflop komponens hozzáadása
	\item[] \texttt{$+$ void add(Scope scope)}: Oszcillátor típusú komponens hozzáadása
	\item[] \texttt{$+$ void add(SequenceGenerator sg)}: Jelgenerátor komponens hozzáadása
	\item[] \texttt{$+$ void add(SourceComponent sc)}: Jelforrás típusú komponens hozzáadása
	\item[] \texttt{$+$ void addTo(Composite composite)}: Kompozit hozzáadása kompozithoz.
	\item[] \texttt{$-$ void commitFlipFlops()}: A flipflopok jelenlegi kimenetét elmentjük belső állapotnak, és az órajel  bemenetén lévő értéket pedig eltároljuk az éldetektálás érdekében.
	\item[] \texttt{$-$ void commitScopes()}: Oszcilloszkópok véglegesítése
	\item[] \texttt{$+$ void connectComponents(Map connections, String[] inputs, String[] outputs)}: Komponensek összekötése
	\item[] \texttt{$+$ Composite copy(String variableName)}: Kompozit lemásolása (példányosításnál használjuk.)
	\item[] \texttt{$+$ AbstractComponent getComponentByName(String name)}: Komponens lekérése a neve alapján (delegálja a kérést, ha kell).
	\item[] \texttt{$+$ Collection getComponents()}: Összes tartalmazott komponens listája
	\item[] \texttt{$+$ List getDisplayComponents()}: Megjelenítők listája
	\item[] \texttt{$+$ List getSourceComponents()}: Jelforrások listája
	\item[] \texttt{$+$ List getStepGenerators()}: Jelgenerátorok listája
	\item[] \texttt{$\#$ void onEvaluation()}: Kiértékelési ciklus
	\item[] \texttt{$+$ void setInput(int inputPin, Wire wire)}: Bemenet beállítása
	\item[] \texttt{$+$ void setOutput(int outputPin, Wire wire)}: Kimenet beállítása
	\item[] \texttt{$-$ void stepGenerators()}: Jelgenerátorok léptetése
\end{itemize}
\end{itemize}

\subsection{DisplayComponent}
Absztrakt osztály.
\begin{itemize}
\item Felelősség\\
Megjelenítő típusú komponenst reprezentál. Tőle származnak a megjelenítők (pl. led).
\item Ősosztályok\ Object $\rightarrow{}$ AbstractComponent $\rightarrow{}$ DisplayComponent.
\item Interfészek (nincs)
\item Attribútumok $\ $
\begin{itemize}
\item (nincs)
\end{itemize}
\item Metódusok$\ $
\begin{itemize}
	\item[] \texttt{$\#$ DisplayComponent(String name, int inputCount)}: Konstruktor. Nem lesz kimenete.
	\item[] \texttt{$+$ void addTo(Composite composite)}: Hozzáadás kompozithoz
\end{itemize}
\end{itemize}

\subsection{FlipFlop}
Absztrakt osztály.
\begin{itemize}
\item Felelősség\\
Flipflopok ősosztálya, minden flipflop 1. bemenete az órajel!
\item Ősosztályok\ Object $\rightarrow{}$ AbstractComponent $\rightarrow{}$ FlipFlop.
\item Interfészek (nincs)
\item Attribútumok $\ $
\begin{itemize}
	\item[] \texttt{$\#$ Value clk} Előző érvényes órajel, ettől és a kiértékelés pillanatában lévő órajel értékétől  függően észlelhetjük, hogy felfutó él van-e vagy sem.
	\item[] \texttt{$\#$ final int \underline{CLK}} Fixen az 1. bemenet az órajel
	\item[] \texttt{$\#$ Value q} Belső memóriája, ami a kimenetén megjelenik, órajel felfutó élénél változhat az állapota.
\end{itemize}
\item Metódusok$\ $
\begin{itemize}
	\item[] \texttt{$+$ FlipFlop(String name, int inputCount)}: 
 % TODO
	\item[] \texttt{$+$ void addTo(Composite composite)}: Hozzáadás kompozithoz
	\item[] \texttt{$+$ void commit()}: Véglegesítés
	\item[] \texttt{$+$ boolean isActive()}: Számolhat-e az FF? Ezt hívja meg az FF-ek onEvaluation() metódusa, mielőtt  bármit is csinálnának.
	\item[] \texttt{$\#$ abstract void onCommit()}: Kimenetre értékadás a logika elvégzése után.
	\item[] \texttt{$\#$ void onEvaluation()}: Nem csinálunk semmit, majd csak commit()-nál.
\end{itemize}
\end{itemize}

\subsection{SourceComponent}
Absztrakt osztály.
\begin{itemize}
\item Felelősség\\
Jelforrás típusú komponenst reprezentál. Tőle származnak a jelforrások (pl. toggle).
\item Ősosztályok\ Object $\rightarrow{}$ AbstractComponent $\rightarrow{}$ SourceComponent.
\item Interfészek (nincs)
\item Attribútumok $\ $
\begin{itemize}
\item (nincs)
\end{itemize}
\item Metódusok$\ $
\begin{itemize}
	\item[] \texttt{$\#$ SourceComponent(String name)}: Konstruktor. Nincs bemenete és egy kimenete van
	\item[] \texttt{$+$ void addTo(Composite composite)}: Hozzáadás kompozithoz.
	\item[] \texttt{$+$ abstract Value[] getValues()}: Lekérhetjük a jelforrás értékeit, hogy el tudjuk menteni.
	\item[] \texttt{$+$ abstract void reset()}: Jelforrás nullázása
	\item[] \texttt{$+$ abstract void setValues(Value[] values)}: Beállítjuk a jelforrás értékét. Kapcsoló esetén csak 1 elemű tömb  adható paraméterként!
\end{itemize}
\end{itemize}

\subsection{Wire}
\begin{itemize}
\item Felelősség\\
Vezeték osztály. Két komponens-lábat köt össze. A rajta lévő érték lekérdezhető  és beállítható.
\item Ősosztályok\ Object $\rightarrow{}$ Wire.
\item Interfészek (nincs)
\item Attribútumok $\ $
\begin{itemize}
	\item[] \texttt{$-$ Value value} Vezetéken lévő érték
\end{itemize}
\item Metódusok$\ $
\begin{itemize}
	\item[] \texttt{$+$ Wire()}: 
 % TODO
	\item[] \texttt{$+$ Value getValue()}: Vezeték értékének lekérése
	\item[] \texttt{$+$ void setValue(Value value)}: Vezeték értékének beállítása
\end{itemize}
\end{itemize}


\subsection{AndGate}
\begin{itemize}
\item Felelősség\\
ÉS kapu, az áramkör egyik alapeleme. Bemeneteire kötött komponensek  kiértékelését kezdeményezi, s a kapott értékek logikai ÉS kapcsolatát  valósítja meg, amit a kimenetén kiad.
\item Ősosztályok\ Object $\rightarrow{}$ AbstractComponent $\rightarrow{}$ AndGate.
\item Interfészek (nincs)
\item Attribútumok $\ $
\begin{itemize}
\item (nincs)
\end{itemize}
\item Metódusok$\ $
\begin{itemize}
	\item[] \texttt{$+$ AndGate(int pinsCount, String name)}: 
 % TODO
	\item[] \texttt{$+$ AndGate copy(String newName)}: 
 % TODO
	\item[] \texttt{$\#$ void onEvaluation()}: 
 % TODO
\end{itemize}
\end{itemize}

\subsection{FlipFlopD}
\begin{itemize}
\item Felelősség\\
D flipflop, mely felfutó órajelnél beírja a belső memóriába az adatbemeneten (D)  lévő értéket.
\item Ősosztályok\ Object $\rightarrow{}$ AbstractComponent $\rightarrow{}$ FlipFlop $\rightarrow{}$ FlipFlopD.
\item Interfészek (nincs)
\item Attribútumok $\ $
\begin{itemize}
	\item[] \texttt{$-$ final int \underline{D}} D bemenet lábának a száma.
\end{itemize}
\item Metódusok$\ $
\begin{itemize}
	\item[] \texttt{$+$ FlipFlopD(String name)}: 
 % TODO
	\item[] \texttt{$+$ FlipFlopD copy(String newName)}: 
 % TODO
	\item[] \texttt{$\#$ void onEvaluation()}: 
 % TODO
\end{itemize}
\end{itemize}

\subsection{FlipFlopJK}
\begin{itemize}
\item Felelősség\\
JK flipflop, mely a belső memóriáját a Követelmények résznél leírt módon  a J és K bemenetektől függően változtatja.
\item Ősosztályok\ Object $\rightarrow{}$ AbstractComponent $\rightarrow{}$ FlipFlop $\rightarrow{}$ FlipFlopJK.
\item Interfészek (nincs)
\item Attribútumok $\ $
\begin{itemize}
	\item[] \texttt{$-$ final int \underline{J}} J bemenet lábának a száma
	\item[] \texttt{$-$ final int \underline{K}} K bemenet lábának a száma
\end{itemize}
\item Metódusok$\ $
\begin{itemize}
	\item[] \texttt{$+$ FlipFlopJK(String name)}: 
 % TODO
	\item[] \texttt{$+$ FlipFlopJK copy(String newName)}: 
 % TODO
	\item[] \texttt{$\#$ void onEvaluation()}: 
 % TODO
\end{itemize}
\end{itemize}

\subsection{Gnd}
\begin{itemize}
\item Felelősség\\
A "föld" komponens, mely állandóan a hamis értéket adja ki. Nincs bemenete.
\item Ősosztályok\ Object $\rightarrow{}$ AbstractComponent $\rightarrow{}$ Gnd.
\item Interfészek (nincs)
\item Attribútumok $\ $
\begin{itemize}
\item (nincs)
\end{itemize}
\item Metódusok$\ $
\begin{itemize}
	\item[] \texttt{$+$ Gnd(String name)}: 
 % TODO
	\item[] \texttt{$+$ AbstractComponent copy(String newName)}: 
 % TODO
	\item[] \texttt{$\#$ void onEvaluation()}: 
 % TODO
\end{itemize}
\end{itemize}

\subsection{Inverter}
\begin{itemize}
\item Felelősség\\
Inverter alkatrész, mely invertálva adja ki a kimenetén a bemenetén  érkező jelet.
\item Ősosztályok\ Object $\rightarrow{}$ AbstractComponent $\rightarrow{}$ Inverter.
\item Interfészek (nincs)
\item Attribútumok $\ $
\begin{itemize}
\item (nincs)
\end{itemize}
\item Metódusok$\ $
\begin{itemize}
	\item[] \texttt{$+$ Inverter(String name)}: Konstruktor. 1 bemenet és 1 kimenet
	\item[] \texttt{$+$ AbstractComponent copy(String name)}: 
 % TODO
	\item[] \texttt{$\#$ void onEvaluation()}: 
 % TODO
\end{itemize}
\end{itemize}

\subsection{Led}
\begin{itemize}
\item Felelősség\\
Egy LED-et reprezentál, mely világít, ha bemenetén igaz érték van.
\item Ősosztályok\ Object $\rightarrow{}$ AbstractComponent $\rightarrow{}$ DisplayComponent $\rightarrow{}$ Led.
\item Interfészek (nincs)
\item Attribútumok $\ $
\begin{itemize}
\item (nincs)
\end{itemize}
\item Metódusok$\ $
\begin{itemize}
	\item[] \texttt{$+$ Led(String name)}: Konstruktor. 1 bemenetű megjelenítő
	\item[] \texttt{$+$ Led copy(String name)}: 
 % TODO
	\item[] \texttt{$+$ Value getValue()}: Visszaadja a led értékét
	\item[] \texttt{$\#$ void onEvaluation()}: 
 % TODO
	\item[] \texttt{$+$ void writeValueTo(Viewable view)}: 
 % TODO
\end{itemize}
\end{itemize}

\subsection{Mpx}
\begin{itemize}
\item Felelősség\\
4-1-es multiplexer, melynek a bemeneti lábak sorrendje a következő:  D0, D1, D2, D3, S0, S1. Ahol Dx az adatbemenetek, Sy a kiválasztóbemenetek.  Kimenetén a kiválasztóbemenetektől függően valamelyik adatbemenet kerül kiadásra.
\item Ősosztályok\ Object $\rightarrow{}$ AbstractComponent $\rightarrow{}$ Mpx.
\item Interfészek (nincs)
\item Attribútumok $\ $
\begin{itemize}
	\item[] \texttt{$-$ final int \underline{DATA0}} 
 % TODO
	\item[] \texttt{$-$ final int \underline{DATA1}} 
 % TODO
	\item[] \texttt{$-$ final int \underline{DATA2}} 
 % TODO
	\item[] \texttt{$-$ final int \underline{DATA3}} 
 % TODO
	\item[] \texttt{$-$ final int \underline{SEL0}} 
 % TODO
	\item[] \texttt{$-$ final int \underline{SEL1}} 
 % TODO
\end{itemize}
\item Metódusok$\ $
\begin{itemize}
	\item[] \texttt{$+$ Mpx(String name)}: 
 % TODO
	\item[] \texttt{$+$ Mpx copy(String newName)}: 
 % TODO
	\item[] \texttt{$\#$ void onEvaluation()}: 
 % TODO
\end{itemize}
\end{itemize}

\subsection{Node}
\begin{itemize}
\item Felelősség\\
Csomópont elem. Az egyetlen bemenetére kötött értéket kiadja az összes kimeneti lábán.
\item Ősosztályok\ Object $\rightarrow{}$ AbstractComponent $\rightarrow{}$ Node.
\item Interfészek (nincs)
\item Attribútumok $\ $
\begin{itemize}
\item (nincs)
\end{itemize}
\item Metódusok$\ $
\begin{itemize}
	\item[] \texttt{$+$ Node(int outputPinsCount, String name)}: Konstruktor. 1 bemenete van
	\item[] \texttt{$+$ AbstractComponent copy(String name)}: 
 % TODO
	\item[] \texttt{$\#$ void onEvaluation()}: 
 % TODO
\end{itemize}
\end{itemize}

\subsection{OrGate}
\begin{itemize}
\item Felelősség\\
VAGY kapu, az áramkör egyik alapeleme. Bemenetein lévő értékek logikai VAGY kapcsolatát  valósítja meg, amit a kimenetén kiad.
\item Ősosztályok\ Object $\rightarrow{}$ AbstractComponent $\rightarrow{}$ OrGate.
\item Interfészek (nincs)
\item Attribútumok $\ $
\begin{itemize}
\item (nincs)
\end{itemize}
\item Metódusok$\ $
\begin{itemize}
	\item[] \texttt{$+$ OrGate(int inputPinsCount, String name)}: Konstruktor. 1 kimenete van
	\item[] \texttt{$+$ AbstractComponent copy(String name)}: 
 % TODO
	\item[] \texttt{$\#$ void onEvaluation()}: 
 % TODO
\end{itemize}
\end{itemize}

\subsection{Scope}
\begin{itemize}
\item Felelősség\\
Egy oszcilloszkópot reprezentál. Eltárolt értékek egy sorba kerülnek bele, mely fix méretű.
\item Ősosztályok\ Object $\rightarrow{}$ AbstractComponent $\rightarrow{}$ DisplayComponent $\rightarrow{}$ Led $\rightarrow{}$ Scope.
\item Interfészek (nincs)
\item Attribútumok $\ $
\begin{itemize}
	\item[] \texttt{$-$ Queue memory} Eltárolt értékek sora.
	\item[] \texttt{$-$ int size} Eltárolható értékek száma.
\end{itemize}
\item Metódusok$\ $
\begin{itemize}
	\item[] \texttt{$+$ Scope(int size, String name)}: Konstruktor. 1 bemenetű megjelenítő
	\item[] \texttt{$+$ void addTo(Composite composite)}: Hozzáadás kompozithoz.
	\item[] \texttt{$+$ void commit()}: Eltároljuk az értéket a memóriában
	\item[] \texttt{$+$ Scope copy(String name)}: 
 % TODO
	\item[] \texttt{$+$ Value[] getValues()}: Visszaadja az eddig eltárolt értékeket
	\item[] \texttt{$\#$ void onEvaluation()}: 
 % TODO
	\item[] \texttt{$+$ void writeTo(Viewable view)}: Komponens kiírása a viewra.
	\item[] \texttt{$+$ void writeValueTo(Viewable view)}: Érték kiírása a kimenetre.
\end{itemize}
\end{itemize}

\subsection{SequenceGenerator}
\begin{itemize}
\item Felelősség\\
Jelgenerátort reprezentál, amely a beállított bitsorozatot adja ki.  Alapértelmezetten (amíg a felhasználó nem állítja be, vagy tölt be másikat) a 0,1-es  szekvenciát tárolja.
\item Ősosztályok\ Object $\rightarrow{}$ AbstractComponent $\rightarrow{}$ SourceComponent $\rightarrow{}$ SequenceGenerator.
\item Interfészek (nincs)
\item Attribútumok $\ $
\begin{itemize}
	\item[] \texttt{$-$ int index} Bitsorozat egy indexe, ez határozza meg, hogy éppen melyik értéket adja ki.
	\item[] \texttt{$-$ Value[] sequence} Tárolt bitsorozat
\end{itemize}
\item Metódusok$\ $
\begin{itemize}
	\item[] \texttt{$+$ SequenceGenerator(String name)}: Konstruktor, ami alapállapotban a 0,1-es szekvenciát állítja be.
	\item[] \texttt{$+$ void addTo(Composite composite)}: Hozzáadás kompozithoz.
	\item[] \texttt{$+$ SequenceGenerator copy(String newName)}: 
 % TODO
	\item[] \texttt{$+$ Value[] getValues()}: Jelgenerátor bitsorozatának lekérdezése
	\item[] \texttt{$\#$ void onEvaluation()}: 
 % TODO
	\item[] \texttt{$+$ void setValues(Value[] values)}: Jelgenerátor bitsorozatának beállítása
	\item[] \texttt{$+$ void step()}: A jelgenerátor lép, a bitsorozat következő elemére ugrik. A következő léptetésig  ez kerül kiadásra a kimeneteken.
	\item[] \texttt{$+$ void writeValueTo(Viewable view)}: Érték kiírása a megjelenítőre
\end{itemize}
\end{itemize}

\subsection{SevenSegmentDisplay}
\begin{itemize}
\item Felelősség\\
7-szegmenses kijelzőt reprezentál, melynek 7 bemenete vezérli a  megfelelő szegmenseket, ezek világítanak, ha az adott bemenetre logikai  igaz van kötve.
\item Ősosztályok\ Object $\rightarrow{}$ AbstractComponent $\rightarrow{}$ DisplayComponent $\rightarrow{}$ SevenSegmentDisplay.
\item Interfészek (nincs)
\item Attribútumok $\ $
\begin{itemize}
\item (nincs)
\end{itemize}
\item Metódusok$\ $
\begin{itemize}
	\item[] \texttt{$+$ SevenSegmentDisplay(String name)}: 
 % TODO
	\item[] \texttt{$+$ AbstractComponent copy(String newName)}: 
 % TODO
	\item[] \texttt{$+$ Value getSegment(int segment)}: Egy szegmens értékének lekérdezése
	\item[] \texttt{$\#$ void onEvaluation()}: 
 % TODO
	\item[] \texttt{$+$ void writeValueTo(Viewable view)}: 
 % TODO
\end{itemize}
\end{itemize}

\subsection{Toggle}
\begin{itemize}
\item Felelősség\\
Kapcsoló jelforrás, melyet a felhasználó szimuláció közben kapcsolgathat.
\item Ősosztályok\ Object $\rightarrow{}$ AbstractComponent $\rightarrow{}$ SourceComponent $\rightarrow{}$ Toggle.
\item Interfészek (nincs)
\item Attribútumok $\ $
\begin{itemize}
	\item[] \texttt{$-$ Value v} Kapcsoló állapota
\end{itemize}
\item Metódusok$\ $
\begin{itemize}
	\item[] \texttt{$+$ Toggle(String name)}: Konstruktor
	\item[] \texttt{$+$ AbstractComponent copy(String name)}: 
 % TODO
	\item[] \texttt{$+$ Value[] getValues()}: Lekérjük a kapcsoló értékét (1 elemű tömb)
	\item[] \texttt{$\#$ void onEvaluation()}: 
 % TODO
	\item[] \texttt{$+$ void setValues(Value[] newValues)}: Kapcsoló állapotának változtatása, csak 1 elemű tömböt kaphat paraméterül.
	\item[] \texttt{$+$ void writeValueTo(Viewable view)}: 
 % TODO
\end{itemize}
\end{itemize}

\subsection{Vcc}
\begin{itemize}
\item Felelősség\\
A tápfeszültés komponens, ami konstans igaz értéket ad. Nincs bemenete.
\item Ősosztályok\ Object $\rightarrow{}$ AbstractComponent $\rightarrow{}$ Vcc.
\item Interfészek (nincs)
\item Attribútumok $\ $
\begin{itemize}
\item (nincs)
\end{itemize}
\item Metódusok$\ $
\begin{itemize}
	\item[] \texttt{$+$ Vcc(String name)}: 
 % TODO
	\item[] \texttt{$+$ AbstractComponent copy(String newName)}: 
 % TODO
	\item[] \texttt{$\#$ void onEvaluation()}: 
 % TODO
\end{itemize}
\end{itemize}



\section{A tesztek részletes tervei, leírásuk a teszt nyelvén}


\subsection{Alap áramkör}
\begin{itemize}
\item Leírás\newline
Olyan áramkör, melyben 2 kapcsolóval állíthatjuk egy ÉS kapu bemeneteit, melyet egy LED jelenít meg.
\item Ellenőrzött funkcionalitás, várható hibahelyek\newline
Ellenőrizzük a kapcsoló helyes váltását, az ÉS kapu kimenetének helyes kiszámítását és a LED működését
\item Áramkör létrehozása

\begin{verbatim}
kapcs1=TOGGLE()
kapcs2=TOGGLE()
es=AND(kapcs1,kapcs2)
led=LED(es)
\end{verbatim}

\item Bemenet és kimenet\newline

\begin{tabular}{|p{7cm}|p{7cm}|} 
\hline 
\textit{Bemenet} & \textit{Kimenet} \\ \hline
\begin{verbatim}
loadCircuit test1.ovr
step
switch kapcs1
step
check -all
switch kapcs2
step
\end{verbatim}
& 
\begin{verbatim}
simulation successful
kapcs1: 0
kapcs2: 0
led: 0

kapcs1: 1

simulation successful
kapcs1: 1
kapcs2: 0
led: 0

led:
 in: 0
 out: 
kapcs1:
 in: 
 out: 1
kapcs2:
 in: 
 out: 0
es:
 in: 1, 0
 out: 0
 
kapcs2: 1

simulation successful
kapcs1: 1
kapcs2: 1
led: 1
\end{verbatim}
\\ \hline
\end{tabular}

\end{itemize}

\subsection{MPX-es áramkör}
\begin{itemize}
\item Leírás\newline
Olyan áramkört hozunk létre, melyben egy 7 szegmenses kijelzőt hajtunk meg kapcsolókkal és egy MPX-xel. A 7szegmenses kijelző [2]-[7] bemeneteire kapcsolókat kötünk, a [1] bemenetét egy MPX adja, mely 4 kapcsolóból választja ki az egyiket, tehát egy 4/1-es MPX.
\item Ellenőrzött funkcionalitás, várható hibahelyek\newline
Ellenőrizzük a MPX helyes működését, és a 7 szegmenses kijelzőt. Hiba a MPX kiválasztása során történhet, hogy rossz jelet juttat a kimenetére.

\item Áramkör létrehozása

\begin{verbatim}
inmpx1=TOGGLE()
inmpx2=TOGGLE()
inmpx3=TOGGLE()
inmpx4=TOGGLE()
selmpx1=TOGGLE()
selmpx2=TOGGLE()
mux=MPX(inmpx1,inmpx2,inmpx3,inmpx4,selmpx1,selmpx2)
seg=TOGGLE()
display=7SEG(mux,seg,0,0,0,0,0)
\end{verbatim}

\item Bemenet és kimenet\newline

\begin{tabular}{|p{7cm}|p{7cm}|} 
\hline 
\textit{Bemenet} & \textit{Kimenet} \\ \hline
\begin{verbatim}
loadCircuit test2.ovr
switch inmpx1
switch inmpx3
step
switch selmpx2
switch seg2
step
switch selmpx2
switch selmpx1
step
\end{verbatim}
& 
\begin{verbatim}
load successful

inmpx1: 1

inmpx3: 1

simulation successful
inmpx1: 1
inmpx2: 0
inmpx3: 1
inmpx4: 0
selmpx1: 0
selmpx2: 0
seg: 0
display: 1, 0, 0, 0, 0, 0, 0

selmpx2: 1

seg: 1

simulation successful
inmpx1: 1
inmpx2: 0
inmpx3: 1
inmpx4: 0
selmpx1: 0
selmpx2: 1
seg: 1
display: 1, 1, 0, 0, 0, 0, 0

selmpx2: 0

selmpx1: 1

simulation successful
inmpx1: 1
inmpx2: 0
inmpx3: 1
inmpx4: 0
selmpx1: 1
selmpx2: 0
seg: 1
display: 0, 1, 0, 0, 0, 0, 0

\end{verbatim}
\\ \hline
\end{tabular}

\end{itemize}

\subsection{Visszacsatolt stabil áramkör}
\begin{itemize}
\item Leírás\newline
Egy olyan áramkört hozunk létre, melyben egy VAGY kapu szerepel, aminek egyik bemenete egy kapcsoló, kimenetét pedig visszakötjük a második bemenetére, illetve egy csomóponton keresztül egy LED-re is eljuttatjuk.
\item Ellenőrzött funkcionalitás, várható hibahelyek\newline
Ellenőrizzük, hogy az áramkör helyesen stabilnak érzékeli-e a kapcsolást, illetve a VAGY kapu helyes működését is ellenőrizzük. Hibát a visszakötés okozhat.

\item Áramkör létrehozása

\begin{verbatim}
kapcs=TOGGLE()
vagy=OR(kapcs,node[2])
node=NODE(vagy,2)
led=LED(node[1])
\end{verbatim}

\item Bemenet és kimenet\newline

\begin{tabular}{|p{7cm}|p{7cm}|} 
\hline 
\textit{Bemenet} & \textit{Kimenet} \\ \hline
\begin{verbatim}
loadCircuit test3.ovr
step
switch kapcs
step
\end{verbatim}
& 
\begin{verbatim}
load successful

simulation successful
kapcs: 0
led: 0

kapcs: 1

simulation successful
kapcs: 1
led: 1
\end{verbatim}
\\ \hline
\end{tabular}

\end{itemize}


\subsection{Visszacsatolt nem stabil áramkör}
\begin{itemize}
\item Leírás\newline
Egy olyan áramkört hozunk létre, melyben egy ÉS kapu szerepel, aminek egyik bemenete egy kapcsoló, kimenetét pedig visszakötjük egy inverteren keresztül a második bemenetére, illetve egy csomóponton keresztül egy LED-re is eljuttatjuk.
\item Ellenőrzött funkcionalitás, várható hibahelyek\newline
Ellenőrizzük, hogy az áramkör helyesen instabilnak érzékeli-e a kapcsolást. Hibás működést ez okozhat, tehát ha az áramkör ezt rosszul állapítja meg, és nem jelzi.

\item Áramkör létrehozása

\begin{verbatim}
kapcs=TOGGLE()
inv=INV(node[2])
es=AND(kapcs,inv)
node=NODE(es,2)
led=LED(node[1])
\end{verbatim}

\item Bemenet és kimenet\newline

\begin{tabular}{|p{7cm}|p{7cm}|} 
\hline 
\textit{Bemenet} & \textit{Kimenet} \\ \hline
\begin{verbatim}
loadCircuit test4.ovr
switch kapcs
step
\end{verbatim}
& 
\begin{verbatim}
load successful

kapcs: 1

simulation failed
\end{verbatim}
\\ \hline
\end{tabular}

\end{itemize}


\subsection{Flip-flop-os áramkör}
\begin{itemize}
\item Leírás\newline
Egy olyan áramkört hozunk létre, melyben egy JK flipflop szerepel, J és K bemenetére kapcsolókat kötünk, órajelét egy jelgenerátorból kapja, és a kimenetét egy oszcilloszkóp kapja meg.
\item Ellenőrzött funkcionalitás, várható hibahelyek\newline
Ellenőrizzük a jelgenerátort, hogy megfelelő jelet adja-e ciklikusan, ellenőrizzük a JK flipflop működését, illetve, hogy megfelelelően lép-e az órajelre, továbbá ellenőrizzük, hogy az oszcilloszkóp helyesen működik-e. Hiba lehetséges a jelgenerátor működésében, a JK flipflop működésében illetve számolásában, és az oszcilloszkóp működésében.

\item Áramkör létrehozása

\begin{verbatim}
j=TOGGLE()
k=TOGGLE()
seqgen=SEQGEN()
jk=FFJK(seqgen,j,k)
scope=SCOPE(jk, 3)
\end{verbatim}

\item Bemenet és kimenet\newline

\begin{tabular}{|p{7cm}|p{7cm}|} 
\hline 
\textit{Bemenet} & \textit{Kimenet} \\ \hline
\begin{verbatim}
loadCircuit test5.ovr
switch k
step
step
switch j
step
step
switch j
switch k
step
step
\end{verbatim}
& 
\begin{verbatim}
load successful

k: 1

simulation successful
j: 0
k: 1
seqgen: 0
scope: 0

simulation successful
j: 0
k: 1
seqgen: 1
scope: 00

j: 1

simulation successful
j: 1
k: 1
seqgen: 0
scope: 000

simulation successful
j: 1
k: 1
seqgen: 1
scope: 001

j: 0

k: 0

simulation successful
j: 0
k: 0
seqgen: 0
scope: 011

simulation successful
j: 0
k: 0
seqgen: 1
scope: 111
\end{verbatim}
\\ \hline
\end{tabular}

\end{itemize}


\subsection{Kompozitos áramkör}
\begin{itemize}
\item Leírás\newline
Egy olyan áramkört valósítunk meg, melyben egy kompozit szerepel. Ez a kompozit egy 2 bites balról tölthető shiftregisztert valósít meg. A kompozitnak két bemenete van egy kapcsoló ami a balról bejövő értéket adja, és egy jelgenerátor, amely az órajelet. Belül 2 db D flipflop található összekötve. Az első flipflop kimenetét kiadja a kompozit kimenetén is, és a második flipflop bemenetére is ráadja, ezért NODE is kell. A kompozit kimenete a 2 bit és a carry.
\item Ellenőrzött funkcionalitás, várható hibahelyek\newline
Kompozit helyes működését ellenőrizzük.

\item Áramkör létrehozása

\begin{verbatim}
input=TOGGLE()
seqgen=SEQGEN()
composite SHR(clk, in){
    in2 = NODE(in, 1)
    d1 = FFD(clk, in)
    node1 = NODE(d1,2)
    d2 = FFD(clk,node1[1])
} (in2, node1[2], d2)
myshr = SHR(seqgen, input)
led1=LED(myshr[1])
led2=LED(myshr[2])
ledcarry=LED(myshr[3])
\end{verbatim}

\item Bemenet és kimenet\newline

\begin{tabular}{|p{7cm}|p{7cm}|} 
\hline 
\textit{Bemenet} & \textit{Kimenet} \\ \hline
\begin{verbatim}
loadCircuit test6.ovr
switch input
step
step
switch input
step
step
step
step
\end{verbatim}
& 
\begin{verbatim}
load successful

input: 1

simulation successful
input: 1
seqgen: 0
led1: 1
led2: 0
ledcarry: 0

simulation successful
input: 1
seqgen: 1
led1: 1
led2: 0
ledcarry: 0

input: 0

simulation successful
input: 0
seqgen: 0
led1: 0
led2: 1
ledcarry: 0

simulation successful
input: 0
seqgen: 1
led1: 0
led2: 1
ledcarry: 0

simulation successful
input: 0
seqgen: 0
led1: 0
led2: 0
ledcarry: 1

simulation successful
input: 0
seqgen: 1
led1: 0
led2: 0
ledcarry: 1
\end{verbatim}
\\ \hline
\end{tabular}

\end{itemize}


\subsection{Kompoziton belüli kompozitos áramkör}
\begin{itemize}
\item Leírás\newline
Egy olyan áramkört hozunk létre melyben egy kompozit szerepel, ami egy másik kompozitot foglal magába. A belső kompozit egyetlen invertert tartalmaz. A külső kompozit tartalmaz még egy VAGY kaput, melynek egyik bementére a belső kompozit kimenetét, másik bemenetére pedig a külső kompozit bemenetére érkező jelet kötjük. A külső kompozit bemenetére egy kapcsolót, kimenetére egy LED-et kötünk.
\item Ellenőrzött funkcionalitás, várható hibahelyek\newline
Leteszteljük, hogy működik-e a kompozit elem, ha belül bonyolultabb áramköri hálózat szerepel, jelen esetben egy kompozit, illetve egy VAGY kapu.

\item Áramkör létrehozása

\begin{verbatim}
tog = TOGGLE()
composite innerComp(in){
  inv = INV(in)
} (inv)
composite Main(in){
  inC = innerComp(in)
  or = OR(in, inC)
} (or)
m = Main(tog)
led = LED(m)
\end{verbatim}

\item Bemenet és kimenet\newline

\begin{tabular}{|p{7cm}|p{7cm}|} 
\hline 
\textit{Bemenet} & \textit{Kimenet} \\ \hline
\begin{verbatim}
loadCircuit test7.ovr
step
step
switch tog
step
step
\end{verbatim}
& 
\begin{verbatim}
load successful

simulation successful
tog: 0
led: 1

simulation successful
tog: 0
led: 1

tog: 1

simulation successful
tog: 1
led: 1

simulation successful
tog: 1
led: 1
\end{verbatim}
\\ \hline
\end{tabular}

\end{itemize}


\section{A tesztelést támogató programok tervei}

Az ellenőrizendő tesztadatokat a prototípus a kijelzőre vagy az argumentumban megadott fájlba írja a kimenetet. Ezt tudjuk összehasonlítani az előre legyártott referencia kimenettel, ami a helyes kimenetet tartalmazza. A két fájl összehasonlításához a DiffUtils cmp.exe programot használjuk.\\

Az ellenőrzés megkönnyítése érdekében a prototípus mellé szállítunk egy batch fájlt, amivel az összes teszteset lefut, és a generált kimenetet összehasonlítja az elvárt kimenetekkel (ezeket is szállítjuk a prototípus mellé).\\

A batch fájl kimenete futtatása után, minden tesztesetnél az alábbi lehet:
\begin{itemize}
\item "Teszt sikeres!" ha a generált tesztfájl megegyezik a referencia fájllal
\item Egyéb esetben pedig cmp.exe által generált hibaüzenet jelenik meg, mely megmutatja a két fájl közti eltéréseket
\end{itemize}
