\subsection{AndGate}
\begin{itemize}
\item Felelősség\\
ÉS kapu, az áramkör egyik alapeleme. Bemeneteire kötött komponensek  kiértékelését kezdeményezi, s a kapott értékek logikai ÉS kapcsolatát  valósítja meg, amit a kimenetén kiad.
\item Ősosztályok\ Object $\rightarrow{}$ AbstractComponent $\rightarrow{}$ AndGate.
\item Interfészek (nincs)
\item Attribútumok $\ $
\begin{itemize}
\item (nincs)
\end{itemize}
\item Metódusok$\ $
\begin{itemize}
	\item[] \texttt{$+$ AndGate(int pinsCount, String name)}: 
 % TODO
	\item[] \texttt{$+$ AndGate copy(String newName)}: 
 % TODO
	\item[] \texttt{$\#$ void onEvaluation()}: 
 % TODO
\end{itemize}
\end{itemize}

\subsection{FlipFlopD}
\begin{itemize}
\item Felelősség\\
D flipflop, mely felfutó órajelnél beírja a belső memóriába az adatbemeneten (D)  lévő értéket.
\item Ősosztályok\ Object $\rightarrow{}$ AbstractComponent $\rightarrow{}$ FlipFlop $\rightarrow{}$ FlipFlopD.
\item Interfészek (nincs)
\item Attribútumok $\ $
\begin{itemize}
	\item[] \texttt{$-$ final int \underline{D}} D bemenet lábának a száma.
\end{itemize}
\item Metódusok$\ $
\begin{itemize}
	\item[] \texttt{$+$ FlipFlopD(String name)}: 
 % TODO
	\item[] \texttt{$+$ FlipFlopD copy(String newName)}: 
 % TODO
	\item[] \texttt{$\#$ void onEvaluation()}: 
 % TODO
\end{itemize}
\end{itemize}

\subsection{FlipFlopJK}
\begin{itemize}
\item Felelősség\\
JK flipflop, mely a belső memóriáját a Követelmények résznél leírt módon  a J és K bemenetektől függően változtatja.
\item Ősosztályok\ Object $\rightarrow{}$ AbstractComponent $\rightarrow{}$ FlipFlop $\rightarrow{}$ FlipFlopJK.
\item Interfészek (nincs)
\item Attribútumok $\ $
\begin{itemize}
	\item[] \texttt{$-$ final int \underline{J}} J bemenet lábának a száma
	\item[] \texttt{$-$ final int \underline{K}} K bemenet lábának a száma
\end{itemize}
\item Metódusok$\ $
\begin{itemize}
	\item[] \texttt{$+$ FlipFlopJK(String name)}: 
 % TODO
	\item[] \texttt{$+$ FlipFlopJK copy(String newName)}: 
 % TODO
	\item[] \texttt{$\#$ void onEvaluation()}: 
 % TODO
\end{itemize}
\end{itemize}

\subsection{Gnd}
\begin{itemize}
\item Felelősség\\
A "föld" komponens, mely állandóan a hamis értéket adja ki. Nincs bemenete.
\item Ősosztályok\ Object $\rightarrow{}$ AbstractComponent $\rightarrow{}$ Gnd.
\item Interfészek (nincs)
\item Attribútumok $\ $
\begin{itemize}
\item (nincs)
\end{itemize}
\item Metódusok$\ $
\begin{itemize}
	\item[] \texttt{$+$ Gnd(String name)}: 
 % TODO
	\item[] \texttt{$+$ AbstractComponent copy(String newName)}: 
 % TODO
	\item[] \texttt{$\#$ void onEvaluation()}: 
 % TODO
\end{itemize}
\end{itemize}

\subsection{Inverter}
\begin{itemize}
\item Felelősség\\
Inverter alkatrész, mely invertálva adja ki a kimenetén a bemenetén  érkező jelet.
\item Ősosztályok\ Object $\rightarrow{}$ AbstractComponent $\rightarrow{}$ Inverter.
\item Interfészek (nincs)
\item Attribútumok $\ $
\begin{itemize}
\item (nincs)
\end{itemize}
\item Metódusok$\ $
\begin{itemize}
	\item[] \texttt{$+$ Inverter(String name)}: Konstruktor. 1 bemenet és 1 kimenet
	\item[] \texttt{$+$ AbstractComponent copy(String name)}: 
 % TODO
	\item[] \texttt{$\#$ void onEvaluation()}: 
 % TODO
\end{itemize}
\end{itemize}

\subsection{Led}
\begin{itemize}
\item Felelősség\\
Egy LED-et reprezentál, mely világít, ha bemenetén igaz érték van.
\item Ősosztályok\ Object $\rightarrow{}$ AbstractComponent $\rightarrow{}$ DisplayComponent $\rightarrow{}$ Led.
\item Interfészek (nincs)
\item Attribútumok $\ $
\begin{itemize}
\item (nincs)
\end{itemize}
\item Metódusok$\ $
\begin{itemize}
	\item[] \texttt{$+$ Led(String name)}: Konstruktor. 1 bemenetű megjelenítő
	\item[] \texttt{$+$ Led copy(String name)}: 
 % TODO
	\item[] \texttt{$+$ Value getValue()}: Visszaadja a led értékét
	\item[] \texttt{$\#$ void onEvaluation()}: 
 % TODO
	\item[] \texttt{$+$ void writeValueTo(Viewable view)}: 
 % TODO
\end{itemize}
\end{itemize}

\subsection{Mpx}
\begin{itemize}
\item Felelősség\\
4-1-es multiplexer, melynek a bemeneti lábak sorrendje a következő:  D0, D1, D2, D3, S0, S1. Ahol Dx az adatbemenetek, Sy a kiválasztóbemenetek.  Kimenetén a kiválasztóbemenetektől függően valamelyik adatbemenet kerül kiadásra.
\item Ősosztályok\ Object $\rightarrow{}$ AbstractComponent $\rightarrow{}$ Mpx.
\item Interfészek (nincs)
\item Attribútumok $\ $
\begin{itemize}
	\item[] \texttt{$-$ final int \underline{DATA0}} 
 % TODO
	\item[] \texttt{$-$ final int \underline{DATA1}} 
 % TODO
	\item[] \texttt{$-$ final int \underline{DATA2}} 
 % TODO
	\item[] \texttt{$-$ final int \underline{DATA3}} 
 % TODO
	\item[] \texttt{$-$ final int \underline{SEL0}} 
 % TODO
	\item[] \texttt{$-$ final int \underline{SEL1}} 
 % TODO
\end{itemize}
\item Metódusok$\ $
\begin{itemize}
	\item[] \texttt{$+$ Mpx(String name)}: 
 % TODO
	\item[] \texttt{$+$ Mpx copy(String newName)}: 
 % TODO
	\item[] \texttt{$\#$ void onEvaluation()}: 
 % TODO
\end{itemize}
\end{itemize}

\subsection{Node}
\begin{itemize}
\item Felelősség\\
Csomópont elem. Az egyetlen bemenetére kötött értéket kiadja az összes kimeneti lábán.
\item Ősosztályok\ Object $\rightarrow{}$ AbstractComponent $\rightarrow{}$ Node.
\item Interfészek (nincs)
\item Attribútumok $\ $
\begin{itemize}
\item (nincs)
\end{itemize}
\item Metódusok$\ $
\begin{itemize}
	\item[] \texttt{$+$ Node(int outputPinsCount, String name)}: Konstruktor. 1 bemenete van
	\item[] \texttt{$+$ AbstractComponent copy(String name)}: 
 % TODO
	\item[] \texttt{$\#$ void onEvaluation()}: 
 % TODO
\end{itemize}
\end{itemize}

\subsection{OrGate}
\begin{itemize}
\item Felelősség\\
VAGY kapu, az áramkör egyik alapeleme. Bemenetein lévő értékek logikai VAGY kapcsolatát  valósítja meg, amit a kimenetén kiad.
\item Ősosztályok\ Object $\rightarrow{}$ AbstractComponent $\rightarrow{}$ OrGate.
\item Interfészek (nincs)
\item Attribútumok $\ $
\begin{itemize}
\item (nincs)
\end{itemize}
\item Metódusok$\ $
\begin{itemize}
	\item[] \texttt{$+$ OrGate(int inputPinsCount, String name)}: Konstruktor. 1 kimenete van
	\item[] \texttt{$+$ AbstractComponent copy(String name)}: 
 % TODO
	\item[] \texttt{$\#$ void onEvaluation()}: 
 % TODO
\end{itemize}
\end{itemize}

\subsection{Scope}
\begin{itemize}
\item Felelősség\\
Egy oszcilloszkópot reprezentál. Eltárolt értékek egy sorba kerülnek bele, mely fix méretű.
\item Ősosztályok\ Object $\rightarrow{}$ AbstractComponent $\rightarrow{}$ DisplayComponent $\rightarrow{}$ Led $\rightarrow{}$ Scope.
\item Interfészek (nincs)
\item Attribútumok $\ $
\begin{itemize}
	\item[] \texttt{$-$ Queue memory} Eltárolt értékek sora.
	\item[] \texttt{$-$ int size} Eltárolható értékek száma.
\end{itemize}
\item Metódusok$\ $
\begin{itemize}
	\item[] \texttt{$+$ Scope(int size, String name)}: Konstruktor. 1 bemenetű megjelenítő
	\item[] \texttt{$+$ void addTo(Composite composite)}: Hozzáadás kompozithoz.
	\item[] \texttt{$+$ void commit()}: Eltároljuk az értéket a memóriában
	\item[] \texttt{$+$ Scope copy(String name)}: 
 % TODO
	\item[] \texttt{$+$ Value[] getValues()}: Visszaadja az eddig eltárolt értékeket
	\item[] \texttt{$\#$ void onEvaluation()}: 
 % TODO
	\item[] \texttt{$+$ void writeTo(Viewable view)}: Komponens kiírása a viewra.
	\item[] \texttt{$+$ void writeValueTo(Viewable view)}: Érték kiírása a kimenetre.
\end{itemize}
\end{itemize}

\subsection{SequenceGenerator}
\begin{itemize}
\item Felelősség\\
Jelgenerátort reprezentál, amely a beállított bitsorozatot adja ki.  Alapértelmezetten (amíg a felhasználó nem állítja be, vagy tölt be másikat) a 0,1-es  szekvenciát tárolja.
\item Ősosztályok\ Object $\rightarrow{}$ AbstractComponent $\rightarrow{}$ SourceComponent $\rightarrow{}$ SequenceGenerator.
\item Interfészek (nincs)
\item Attribútumok $\ $
\begin{itemize}
	\item[] \texttt{$-$ int index} Bitsorozat egy indexe, ez határozza meg, hogy éppen melyik értéket adja ki.
	\item[] \texttt{$-$ Value[] sequence} Tárolt bitsorozat
\end{itemize}
\item Metódusok$\ $
\begin{itemize}
	\item[] \texttt{$+$ SequenceGenerator(String name)}: Konstruktor, ami alapállapotban a 0,1-es szekvenciát állítja be.
	\item[] \texttt{$+$ void addTo(Composite composite)}: Hozzáadás kompozithoz.
	\item[] \texttt{$+$ SequenceGenerator copy(String newName)}: 
 % TODO
	\item[] \texttt{$+$ Value[] getValues()}: Jelgenerátor bitsorozatának lekérdezése
	\item[] \texttt{$\#$ void onEvaluation()}: 
 % TODO
	\item[] \texttt{$+$ void setValues(Value[] values)}: Jelgenerátor bitsorozatának beállítása
	\item[] \texttt{$+$ void step()}: A jelgenerátor lép, a bitsorozat következő elemére ugrik. A következő léptetésig  ez kerül kiadásra a kimeneteken.
	\item[] \texttt{$+$ void writeValueTo(Viewable view)}: Érték kiírása a megjelenítőre
\end{itemize}
\end{itemize}

\subsection{SevenSegmentDisplay}
\begin{itemize}
\item Felelősség\\
7-szegmenses kijelzőt reprezentál, melynek 7 bemenete vezérli a  megfelelő szegmenseket, ezek világítanak, ha az adott bemenetre logikai  igaz van kötve.
\item Ősosztályok\ Object $\rightarrow{}$ AbstractComponent $\rightarrow{}$ DisplayComponent $\rightarrow{}$ SevenSegmentDisplay.
\item Interfészek (nincs)
\item Attribútumok $\ $
\begin{itemize}
\item (nincs)
\end{itemize}
\item Metódusok$\ $
\begin{itemize}
	\item[] \texttt{$+$ SevenSegmentDisplay(String name)}: 
 % TODO
	\item[] \texttt{$+$ AbstractComponent copy(String newName)}: 
 % TODO
	\item[] \texttt{$+$ Value getSegment(int segment)}: Egy szegmens értékének lekérdezése
	\item[] \texttt{$\#$ void onEvaluation()}: 
 % TODO
	\item[] \texttt{$+$ void writeValueTo(Viewable view)}: 
 % TODO
\end{itemize}
\end{itemize}

\subsection{Toggle}
\begin{itemize}
\item Felelősség\\
Kapcsoló jelforrás, melyet a felhasználó szimuláció közben kapcsolgathat.
\item Ősosztályok\ Object $\rightarrow{}$ AbstractComponent $\rightarrow{}$ SourceComponent $\rightarrow{}$ Toggle.
\item Interfészek (nincs)
\item Attribútumok $\ $
\begin{itemize}
	\item[] \texttt{$-$ Value v} Kapcsoló állapota
\end{itemize}
\item Metódusok$\ $
\begin{itemize}
	\item[] \texttt{$+$ Toggle(String name)}: Konstruktor
	\item[] \texttt{$+$ AbstractComponent copy(String name)}: 
 % TODO
	\item[] \texttt{$+$ Value[] getValues()}: Lekérjük a kapcsoló értékét (1 elemű tömb)
	\item[] \texttt{$\#$ void onEvaluation()}: 
 % TODO
	\item[] \texttt{$+$ void setValues(Value[] newValues)}: Kapcsoló állapotának változtatása, csak 1 elemű tömböt kaphat paraméterül.
	\item[] \texttt{$+$ void writeValueTo(Viewable view)}: 
 % TODO
\end{itemize}
\end{itemize}

\subsection{Vcc}
\begin{itemize}
\item Felelősség\\
A tápfeszültés komponens, ami konstans igaz értéket ad. Nincs bemenete.
\item Ősosztályok\ Object $\rightarrow{}$ AbstractComponent $\rightarrow{}$ Vcc.
\item Interfészek (nincs)
\item Attribútumok $\ $
\begin{itemize}
\item (nincs)
\end{itemize}
\item Metódusok$\ $
\begin{itemize}
	\item[] \texttt{$+$ Vcc(String name)}: 
 % TODO
	\item[] \texttt{$+$ AbstractComponent copy(String newName)}: 
 % TODO
	\item[] \texttt{$\#$ void onEvaluation()}: 
 % TODO
\end{itemize}
\end{itemize}

