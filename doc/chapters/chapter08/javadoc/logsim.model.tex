\subsection{Circuit}
\begin{itemize}
\item Felelősség\\
Áramkört reprezentáló osztály, igazából egy kompozit. Felelőssége megegyzik a kompozitéval.
\item Ősosztályok\ Object $\rightarrow{}$ AbstractComponent $\rightarrow{}$ Composite $\rightarrow{}$ Circuit.
\item Interfészek (nincs)
\item Attribútumok $\ $
\begin{itemize}
\item (nincs)
\end{itemize}
\item Metódusok$\ $
\begin{itemize}
	\item[] \texttt{$+$ Circuit()}: 
 % TODO
\end{itemize}
\end{itemize}

\subsection{Simulation}
\begin{itemize}
\item Felelősség\\
Egy szimulációt reprezentáló objektum.  Utasítja az áramkört, hogy értékelje ki magát. Ha az áramkör azt jelzi magáról,  hogy nincs stacionárius állapota akkor jelezzük a felhasználónak.
\item Ősosztályok\ Object $\rightarrow{}$ Simulation.
\item Interfészek (nincs)
\item Attribútumok $\ $
\begin{itemize}
	\item[] \texttt{$\#$ Circuit circuit} Szimulált áramkör
\end{itemize}
\item Metódusok$\ $
\begin{itemize}
	\item[] \texttt{$+$ Simulation()}: 
 % TODO
	\item[] \texttt{$+$ void setCircuit(Circuit circuit)}: Szimulált áramkör beállítása
	\item[] \texttt{$+$ boolean start()}: Egy adott bemeneti kombinációkra kiértékeli a hálózatot.
\end{itemize}
\end{itemize}

\subsection{Value}
\begin{itemize}
\item Felelősség\\
Az áramkörben előfordulható értéket reprezentál.
\item Ősosztályok\ Object $\rightarrow{}$ Enum $\rightarrow{}$ Value.
\item Interfészek (nincs)
\item Attribútumok $\ $
\begin{itemize}
	\item[] \texttt{$+$ final Value \underline{FALSE}} 
 % TODO
	\item[] \texttt{$+$ final Value \underline{TRUE}} 
 % TODO
\end{itemize}
\item Metódusok$\ $
\begin{itemize}
	\item[] \texttt{$-$ Value()}: 
 % TODO
	\item[] \texttt{$+$ Value invert()}: Érték invertálása
	\item[] \texttt{$+$ static Value valueOf(String name)}: 
 % TODO
	\item[] \texttt{$+$ static Value[] values()}: 
 % TODO
\end{itemize}
\end{itemize}

