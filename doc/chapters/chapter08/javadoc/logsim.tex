\subsection{Config}
\begin{itemize}
\item Felelősség\\
Konfigurációs fájlok kezelése, azok írása az áramkör alapján, illetve azok betöltése  az áramkörbe.
\item Ősosztályok\ Object $\rightarrow{}$ Config.
\item Interfészek (nincs)
\item Attribútumok $\ $
\begin{itemize}
	\item[] \texttt{$-$ Circuit circuit} Áramkör, aminek mentjük a dolgait
	\item[] \texttt{$-$ Pattern \underline{sourceComponentPattern}} Regex kifejezés az illesztéshez (beolvasásnál)
\end{itemize}
\item Metódusok$\ $
\begin{itemize}
	\item[] \texttt{$+$ Config(Circuit circuit)}: Példány létrehozása az áramkörhöz.
	\item[] \texttt{$+$ int load(File file)}: Betölti egy fájlból a kapcsolók illetve jelgenerátorok konfigurációját
	\item[] \texttt{$+$ int save(File file)}: Elmenti a kapcsolók illetve jelgenerátorok aktuális állapotát egy fájlba
\end{itemize}
\end{itemize}

\subsection{Controller}
Interfész.
\begin{itemize}
\item Felelősség\\
Kontroller interfész.
\item Ősosztályok\ Controller.
\item Interfészek (nincs)
\item Metódusok$\ $
\begin{itemize}
	\item[] \texttt{$+$ void run(BufferedReader input)}: Vezérlés elindítása adott bemenetről.
\end{itemize}
\end{itemize}

\subsection{Parser}
\begin{itemize}
\item Felelősség\\
Áramkör értelmező objektum, feladata, hogy a paraméterként átadott, illetve  fájlban elhelyezett komponenseket értelmezze, a kapcsolatokat feltérképezze,  elvégezze az összeköttetéseket, és ezáltal felépítse az áramkört.
\item Ősosztályok\ Object $\rightarrow{}$ Parser.
\item Interfészek (nincs)
\item Attribútumok $\ $
\begin{itemize}
	\item[] \texttt{$-$ Circuit circuit} A leíróból létrehozott áramkör.
	\item[] \texttt{$-$ Pattern \underline{componentPattern}} Regex minta egy komponens-sor feldolgozásához
	\item[] \texttt{$-$ Pattern \underline{compositeEndPattern}} Regex minta egy kompozit véghez
	\item[] \texttt{$-$ Map composites} Komponensek listája név szerint.
	\item[] \texttt{$-$ Pattern \underline{compositeStartPattern}} Regex minta egy kompozit kezdethez
	\item[] \texttt{$-$ Map parameters} Kompozitokban lévő komponensek paraméter listája.
\end{itemize}
\item Metódusok$\ $
\begin{itemize}
	\item[] \texttt{$+$ Parser()}: 
 % TODO
	\item[] \texttt{$-$ void addComponentsToComposite(Composite composite, List lines, String[] inputs, String[] outputs)}: Komponens hozzáadása a kompozithoz
	\item[] \texttt{$-$ void parse(BufferedReader br)}: Bementről feldolgozás
	\item[] \texttt{$+$ Circuit parse(File file)}: Létrehoz egy áramkört a megadott fájlból
	\item[] \texttt{$-$ AbstractComponent parseComponentFromLine(Matcher matcher, Composite composite)}: Egy komponens-sor feldolgozása a fájlban
\end{itemize}
\end{itemize}

\subsection{Proto}
\begin{itemize}
\item Felelősség\\
Prototípus vezérlő osztálya.
\item Ősosztályok\ Object $\rightarrow{}$ Proto.
\item Interfészek Controller.
\item Attribútumok $\ $
\begin{itemize}
	\item[] \texttt{$-$ Circuit c} Áramkör
	\item[] \texttt{$-$ Config config} Konfiguráció menedzselése
	\item[] \texttt{$-$ Simulation s} Szimuláció
	\item[] \texttt{$-$ Viewable view} Megjelenítő
\end{itemize}
\item Metódusok$\ $
\begin{itemize}
	\item[] \texttt{$+$ Proto(String[] args)}: 
 % TODO
	\item[] \texttt{$-$ void eval(String s)}: Parancs értelmezése
	\item[] \texttt{$+$ static void main(String[] args)}: Program belépési pontja.
	\item[] \texttt{$+$ final void run(BufferedReader input)}: Felhasználó parancsait olvassa
\end{itemize}
\end{itemize}

\subsection{View}
\begin{itemize}
\item Felelősség\\
Egy konkrét kimeneti implementáció, mely OutputStreamWriter-be ír ki,  így a konzolos megjelenítés és fájlba írás megoldott.
\item Ősosztályok\ Object $\rightarrow{}$ View.
\item Interfészek Viewable.
\item Attribútumok $\ $
\begin{itemize}
	\item[] \texttt{$-$ Controller controller} Kontroller
	\item[] \texttt{$-$ PrintWriter out} Kimeneti adatfolyam, ide írunk.
\end{itemize}
\item Metódusok$\ $
\begin{itemize}
	\item[] \texttt{$+$ View(Controller c, OutputStreamWriter out)}: Létehozzuk a Viewt egy kontrollerrel és a kimenettel, ide fog menni a kimenet.
	\item[] \texttt{$+$ void newline()}: Új sor a kimeneten
	\item[] \texttt{$+$ void writeDetails(AbstractComponent ac)}: Kiírunk egy komponenst (be és kimenetek)
	\item[] \texttt{$+$ void writeLedValue(Led led)}: Kiírja a led értékét
	\item[] \texttt{$+$ void writeLoadFailed()}: Kiírjuk, hogy a betöltés sikertelen
	\item[] \texttt{$+$ void writeLoadSuccessful()}: Kiírjuk, hogy a betöltés sikeres
	\item[] \texttt{$+$ void writeSaveFailed()}: Kiírjuk, hogy a config fájl mentése sikertelen
	\item[] \texttt{$+$ void writeSaveSuccessful()}: Kiírjuk, hogy a config fájl mentés sikeres
	\item[] \texttt{$+$ void writeScopeDetails(Scope ac)}: Kiírunk egy scope-ot
	\item[] \texttt{$+$ void writeScopeValues(Scope scope)}: Kiírja a scope által tárolt értékeket
	\item[] \texttt{$+$ void writeSequenceGenerator(SequenceGenerator sg)}: Szekvenciagenerátor szekvenciájának kiírása
	\item[] \texttt{$+$ void writeSequenceGeneratorDetails(SequenceGenerator sg)}: Kiírunk egy jelgenerátort
	\item[] \texttt{$+$ void writeSequenceGeneratorSequence(SequenceGenerator sg)}: Kiírja a jelgenerátor szekvenciáját
	\item[] \texttt{$+$ void writeSequenceGeneratorValue(SequenceGenerator sg)}: Kiírja a jelgenerátor éppen kiadott értékét
	\item[] \texttt{$+$ void writeSevenSegmentDisplayValues(SevenSegmentDisplay seg)}: Kiírja a 7-szegmentes kijelző szegmenseit.
	\item[] \texttt{$+$ void writeSimulationFailed()}: Kiírjuk, hogy a szimuláció sikertelen
	\item[] \texttt{$+$ void writeSimulationSuccessful()}: Kiírjuk, hogy a szimuláció sikeres
	\item[] \texttt{$+$ void writeToggleValue(Toggle sc)}: Kiírja a kapcsoló állapotát
\end{itemize}
\end{itemize}

\subsection{Viewable}
Interfész.
\begin{itemize}
\item Felelősség\\
A kimenet interfésze.
\item Ősosztályok\ Viewable.
\item Interfészek (nincs)
\item Metódusok$\ $
\begin{itemize}
	\item[] \texttt{$+$ void newline()}: Új sor a kimeneten
	\item[] \texttt{$+$ void writeDetails(AbstractComponent ac)}: Kiírunk egy komponenst (be és kimenetek)
	\item[] \texttt{$+$ void writeLedValue(Led led)}: Kiírja a led értékét
	\item[] \texttt{$+$ void writeLoadFailed()}: Kiírjuk, hogy a betöltés sikertelen
	\item[] \texttt{$+$ void writeLoadSuccessful()}: Kiírjuk, hogy betöltés sikeres
	\item[] \texttt{$+$ void writeSaveFailed()}: Kiírjuk, hogy a config fájl sikertelen
	\item[] \texttt{$+$ void writeSaveSuccessful()}: Kiírjuk, hogy a config fájl mentés sikeres
	\item[] \texttt{$+$ void writeScopeDetails(Scope scope)}: Kiírunk egy scope-ot
	\item[] \texttt{$+$ void writeScopeValues(Scope scope)}: Kiírja a scope által tárolt értékeket
	\item[] \texttt{$+$ void writeSequenceGenerator(SequenceGenerator sg)}: Szekvenciagenerátor szekvenciájának kiírása
	\item[] \texttt{$+$ void writeSequenceGeneratorDetails(SequenceGenerator sg)}: Kiírunk egy jelgenerátort
	\item[] \texttt{$+$ void writeSequenceGeneratorSequence(SequenceGenerator sg)}: Kiírja a jelgenerátor szekvenciáját
	\item[] \texttt{$+$ void writeSequenceGeneratorValue(SequenceGenerator sg)}: Kiírja a jelgenerátor éppen kiadott értékét
	\item[] \texttt{$+$ void writeSevenSegmentDisplayValues(SevenSegmentDisplay seg)}: Kiírja a 7-szegmentes kijelző szegmenseit.
	\item[] \texttt{$+$ void writeSimulationFailed()}: Kiírjuk, hogy a szimuláció sikertelen
	\item[] \texttt{$+$ void writeSimulationSuccessful()}: Kiírjuk, hogy a szimuláció sikeres
	\item[] \texttt{$+$ void writeToggleValue(Toggle toggle)}: Kiírja a kapcsoló állapotát
\end{itemize}
\end{itemize}

