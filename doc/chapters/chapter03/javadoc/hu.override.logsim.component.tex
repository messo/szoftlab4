\subsection{AbstractComponent}
Absztrakt osztály.
\begin{itemize}
\item Felelősség\\

 % TODO
\item Ősosztályok\ Object $\rightarrow{}$ AbstractComponent.
\item Interfészek Component.
\item Attribútumok $\ $
\begin{itemize}
	\item \texttt{protected boolean alreadyEvaluated} 
 % TODO
	\item \texttt{protected Circuit circuit} 
 % TODO
	\item \texttt{protected Value[] currentValue} 
 % TODO
	\item \texttt{protected int[] indices} 
 % TODO
	\item \texttt{protected AbstractComponent[] inputs} 
 % TODO
	\item \texttt{protected Value[] lastValue} 
 % TODO
	\item \texttt{protected String name} 
 % TODO
\end{itemize}
\item Metódusok$\ $
\begin{itemize}
	\item \texttt{public void clearEvaluatedFlag()}: 
 % TODO
	\item \texttt{public Value evaluate()}: 
 % TODO
	\item \texttt{public Value evaluate(int outputPin)}: Számolás:
	\item \texttt{public String getName()}: 
 % TODO
	\item \texttt{public Value getValue()}: 
 % TODO
	\item \texttt{public Value getValue(int idx)}: 
 % TODO
	\item \texttt{public void setCircuit(Circuit parent)}: 
 % TODO
	\item \texttt{public void setInput(int inputSlot, AbstractComponent component)}: 
 % TODO
	\item \texttt{public void setInput(int inputPin, AbstractComponent component, int outputPin)}: Beállítunk egy bemenetet.
	\item \texttt{public void setInputPinsCount(int inputPinsCount)}: 
 % TODO
	\item \texttt{public void setName(String name)}: 
 % TODO
\end{itemize}
\end{itemize}

\subsection{Component}
Interfész.
\begin{itemize}
\item Felelősség\\

 % TODO
\item Ősosztályok\ Component.
\item Interfészek (nincs)
\item Metódusok$\ $
\begin{itemize}
	\item \texttt{public String getName()}: 
 % TODO
	\item \texttt{public Value getValue()}: 
 % TODO
	\item \texttt{public Value getValue(int idx)}: 
 % TODO
	\item \texttt{public void setName(String name)}: 
 % TODO
\end{itemize}
\end{itemize}

\subsection{FlipFlop}
Absztrakt osztály.
\begin{itemize}
\item Felelősség\\

 % TODO
\item Ősosztályok\ Object $\rightarrow{}$ AbstractComponent $\rightarrow{}$ FlipFlop.
\item Interfészek (nincs)
\item Attribútumok $\ $
\begin{itemize}
	\item \texttt{private boolean active} 
 % TODO
\end{itemize}
\item Metódusok$\ $
\begin{itemize}
	\item \texttt{public boolean isActive()}: 
 % TODO
	\item \texttt{public void setActive(boolean active)}: Felfutó élnél a SequenceGenerator-nak meg kell hívni ezt a hozzá kötött  FlipFlopokra, egyéb esetben törölnie az active flaget. Így tudja az FF, hogy  mikor kell ténylegesen számolnia.
\end{itemize}
\end{itemize}

\subsection{IsDisplay}
Interfész.
\begin{itemize}
\item Felelősség\\
Megjelenítő típusú komponenst reprezentál. Ezt kell implementálnia a megjelenítőknek.
\item Ősosztályok\ IsDisplay.
\item Interfészek Component.
\item Metódusok$\ $
\begin{itemize}
\item (nincs)
\end{itemize}
\end{itemize}

\subsection{IsSource}
Interfész.
\begin{itemize}
\item Felelősség\\
Jelforrás típusú komponenst reprezentál. Ezt kell implementálnia a jelforrásoknak.
\item Ősosztályok\ IsSource.
\item Interfészek Component.
\item Metódusok$\ $
\begin{itemize}
	\item \texttt{public Value[] getValues()}: Lekérjük a jelforrás értékeit, hogy el tudjuk menteni.
	\item \texttt{public void setValues(Value[] values)}: Beállítjuk a jelforrás értékét. Kapcsoló esetén csak 1 elemű tömb  adható paraméterként!
\end{itemize}
\end{itemize}

