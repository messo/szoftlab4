\subsection{Parser}
\begin{itemize}
\item Felelősség\\
Áramkör értelmező objektum, feladata, hogy a paraméterként átadott, illetve  fájlban elhelyezett komponenseket értelmezze, a kapcsolatokat feltérképezze,  elvégezze az összeköttetéseket, és ezáltal felépítse az áramkört. Fontos, hogy egy ilyen objektum csak egyszer használható, új áramkörhöz, újat kell létrehozni.
\item Ősosztályok:\ Object $\rightarrow{}$ Parser.
\item Interfészek: (nincs)
\item Attribútumok $\ $
\begin{itemize}
	\item \texttt{private static final HashMap availableComponents}: Feldolgozó által ismert komponensek listája.
	\item \texttt{private Circuit circuit}: A leíróból létrehozott áramkör.
	\item \texttt{private static Pattern componentPattern}: Regex minta egy leíró-sor feldolgozásához.
	\item \texttt{private int constComps}: Egy számláló, hogy a vcc és gnd komponenseknek eltérő változónevet tudjunk adni.
	\item \texttt{private static Pattern inputPattern}: Regex minta egy komponens bemeneteinek a feldolgozásához.
	\item \texttt{private HashMap inputs}: Minden komponens-névhez eltároljuk a bemeneteket, későbbi feldolgozás miatt.
\end{itemize}
\item Metódusok$\ $
\begin{itemize}
	\item \texttt{public Circuit parse(File file)}: Létrehoz egy áramkört a megadott fájlból
	\item \texttt{public Circuit parse(String[] content)}: Létrehoz egy áramkört az argumentumokban megadott komponensekből (olyan, mintha mindegyik paraméter egy leíró fájl egy sora lenne)
\end{itemize}
\end{itemize}

\subsection{SourceReader}
\begin{itemize}
\item Felelősség\\
A jelforrások értékeinek fájlból történő beolvasására szolgáló osztály.
\item Ősosztályok:\ Object $\rightarrow{}$ SourceReader.
\item Interfészek: (nincs)
\item Attribútumok $\ $
\begin{itemize}
\item (nincs)
\end{itemize}
\item Metódusok$\ $
\begin{itemize}
	\item \texttt{public void loadValuesToSources(List sources)}: A paraméterben kapott jelforrások értékeit megpróbálja a fájlban  található értékekkel feltölteni.
\end{itemize}
\end{itemize}

\subsection{SourceWriter}
\begin{itemize}
\item Felelősség\\
Segítségével kiírhatjuk egy fájlba a jelforrások értékeit.
\item Ősosztályok:\ Object $\rightarrow{}$ SourceWriter.
\item Interfészek: (nincs)
\item Attribútumok $\ $
\begin{itemize}
\item (nincs)
\end{itemize}
\item Metódusok$\ $
\begin{itemize}
	\item \texttt{public void add(IsSource source)}: Hozzáadjuk a fájlhoz az adott jelforrás beállítását
	\item \texttt{public void close()}: Bezárjuk a fájlt, ha végeztünk, ezt meg kell hívnunk.
\end{itemize}
\end{itemize}

