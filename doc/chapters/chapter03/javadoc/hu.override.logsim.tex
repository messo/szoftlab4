\subsection{Circuit}
\begin{itemize}
\item Felelősség\\
Áramkört reprezentál, melyhez komponeseket lehet adni, és kiértékelési ciklusokat  lehet futtatni, utóbbi a \texttt{Simulation} feladata.
\item Ősosztályok\ Object $\rightarrow{}$ Circuit.
\item Interfészek (nincs)
\item Attribútumok $\ $
\begin{itemize}
	\item \texttt{private HashMap componentMap} 
 % TODO
	\item \texttt{private List sequenceGens} 
 % TODO
	\item \texttt{private Simulation simulation} 
 % TODO
	\item \texttt{private boolean stable} 
 % TODO
\end{itemize}
\item Metódusok$\ $
\begin{itemize}
	\item \texttt{public AbstractComponent addComponent(AbstractComponent component)}: Komponens hozzáadása az áramkörhöz.
	\item \texttt{public void doEvaluationCycle()}: Egy kiértékelési ciklus lefuttatása. Az áramkörtől ezután lekérdezhető, hogy  stabil (nem változott semelyik komponens kimenete az utolsó futtatás óta)  vagy instabil állapotban van-e.
	\item \texttt{public AbstractComponent getComponentByName(String name)}: Lekérünk egy komponenst az áramkörtől a neve alapján.
	\item \texttt{public List getDisplays()}: Megjelenítő típusú komponeseket adja vissza.
	\item \texttt{public List getSequenceGenerators()}: 
 % TODO
	\item \texttt{public List getSources()}: Jelforrás típusú komponenseket adja vissza.
	\item \texttt{public boolean isStable()}: Áramkör stacionárius állapotának lekérdezése.
	\item \texttt{public void setSimulation(Simulation simulation)}: Szimuláció beállítása.
	\item \texttt{public void setStable(boolean stable)}: Áramkör stabilitásának beállítása.
	\item \texttt{public void simulationShouldBeWorking()}: Jelzi a szimuláció felé, hogy új ciklust kell indítani. Ezt egy jelforrás  beállítása után hívjuk meg.
	\item \texttt{public void stepGenerators()}: Jelgenerátorok a szimuláció szemszögéből nézve, egyszerre történő  léptetése.
\end{itemize}
\end{itemize}

\subsection{LogSim}
\begin{itemize}
\item Felelősség\\

 % TODO
\item Ősosztályok\ Object $\rightarrow{}$ LogSim.
\item Interfészek Controller.
\item Attribútumok $\ $
\begin{itemize}
	\item \texttt{ Circuit circuit} 
 % TODO
	\item \texttt{ Simulation simulation} 
 % TODO
	\item \texttt{ View view} 
 % TODO
\end{itemize}
\item Metódusok$\ $
\begin{itemize}
	\item \texttt{public Simulation getFreshSimulation()}: 
 % TODO
	\item \texttt{public static void main(String[] args)}: 
 % TODO
	\item \texttt{public void onCircuitUpdate()}: 
 % TODO
	\item \texttt{public void onExit()}: 
 % TODO
	\item \texttt{public void onStart()}: 
 % TODO
	\item \texttt{public void onStop()}: 
 % TODO
\end{itemize}
\end{itemize}

\subsection{SequenceGeneratorStepper}
\begin{itemize}
\item Felelősség\\

 % TODO
\item Ősosztályok\ Object $\rightarrow{}$ Thread $\rightarrow{}$ SequenceGeneratorStepper.
\item Interfészek (nincs)
\item Attribútumok $\ $
\begin{itemize}
	\item \texttt{private long pause} 
 % TODO
	\item \texttt{private boolean shouldRun} 
 % TODO
	\item \texttt{private Simulation simulation} 
 % TODO
\end{itemize}
\item Metódusok$\ $
\begin{itemize}
	\item \texttt{public void run()}: 
 % TODO
\end{itemize}
\end{itemize}

\subsection{Simulation}
\begin{itemize}
\item Felelősség\\

 % TODO
\item Ősosztályok\ Object $\rightarrow{}$ Thread $\rightarrow{}$ Simulation.
\item Interfészek (nincs)
\item Attribútumok $\ $
\begin{itemize}
	\item \texttt{private Circuit circuit} 
 % TODO
	\item \texttt{private final Controller controller} 
 % TODO
	\item \texttt{private AtomicInteger counter} 
 % TODO
	\item \texttt{private State currentState} 
 % TODO
	\item \texttt{private static final int cycleLimit} 
 % TODO
	\item \texttt{private final Object lock} 
 % TODO
	\item \texttt{private SequenceGeneratorStepper seqGenStepper} 
 % TODO
	\item \texttt{private boolean shouldRun} 
 % TODO
	\item \texttt{private final Object synchObj} 
 % TODO
\end{itemize}
\item Metódusok$\ $
\begin{itemize}
	\item \texttt{public Circuit getCircuit()}: 
 % TODO
	\item \texttt{public Object getLock()}: 
 % TODO
	\item \texttt{public void run()}: 
 % TODO
	\item \texttt{public void saveSources(String fileName)}: 
 % TODO
	\item \texttt{public void setCircuit(Circuit circuit)}: 
 % TODO
	\item \texttt{public void setState(State state)}: 
 % TODO
\end{itemize}
\end{itemize}

\subsection{Simulation.State}
\begin{itemize}
\item Felelősség\\
Szimuláció állapotait írja le
\item Ősosztályok\ Object $\rightarrow{}$ Enum $\rightarrow{}$ Simulation.State.
\item Interfészek (nincs)
\item Attribútumok $\ $
\begin{itemize}
	\item \texttt{public static final State PAUSED} Szimuláció szüneteltetve van, a következő jelforrás változásig.
	\item \texttt{public static final State STOPPED} Szimuláció leállt, ahhoz, hogy bármi történjen az áramkörre újra kell indítani.
	\item \texttt{public static final State WORKING} Szimuláció éppen dolgozik, egy konkrét jelforrás-kombinációt alkalmazva dolgoztatja az áramkört
\end{itemize}
\item Metódusok$\ $
\begin{itemize}
	\item \texttt{public static State valueOf(String name)}: 
 % TODO
	\item \texttt{public static State[] values()}: 
 % TODO
\end{itemize}
\end{itemize}

\subsection{Value}
\begin{itemize}
\item Felelősség\\
Az áramkörben előfordulható érték
\item Ősosztályok\ Object $\rightarrow{}$ Enum $\rightarrow{}$ Value.
\item Interfészek (nincs)
\item Attribútumok $\ $
\begin{itemize}
	\item \texttt{public static final Value FALSE} 
 % TODO
	\item \texttt{public static final Value TRUE} 
 % TODO
\end{itemize}
\item Metódusok$\ $
\begin{itemize}
	\item \texttt{public Value invert()}: 
 % TODO
	\item \texttt{public static Value valueOf(String name)}: 
 % TODO
	\item \texttt{public static Value[] values()}: 
 % TODO
\end{itemize}
\end{itemize}

