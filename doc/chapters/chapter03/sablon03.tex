% Szglab4
% ===========================================================================
%
\chapter{Analízis modell kidolgozása 1}

\thispagestyle{fancy}

\section{Objektum katalógus}

\subsection{\bf Parser}
Áramkör értelmező objektum, feladata, hogy a paraméterként átadott, illetve fájlban elhelyezett komponenseket értelmezze, a kapcsolatokat feltérképezze, elvégezze az összeköttetéseket, és ezáltal felépítse az áramkört.

\subsection{\bf ConsoleView}
Az áramkör karakteres megjelenítéséért, és a szimuláció során a változások megjelenítésének frissítéséért felelős objektum.

\subsection{\bf Simulation}
Szimuláció objektum. A szimulációért felelős. Elindítja a jelgenerátor léptetőt, s utasítja az áramkört a kiértékelésre, és figyeli ha az áramkörben változás történt. Ha változás megadott lépésen belül nem történt, tájékoztatja a felhasználót, hogy nincs stacionárius állapot. Továbbá a megadott grafikai megjelenítőt frissíti.

\subsection{\bf Circuit}
Az áramkör objektum. Ezen objektum feladata a jelgenerátor léptető kérésére a jelgenerátorok léptetése, az áramkörben található komponensek utasítása arra, hogy töröljék a "már kiértékelve" flaget, hogy ezáltal a következő kiértékelés kezdeményezésre továbbítsák azt bemeneteik számára is.
Továbbá feladata a kiértékelés elindítása az összes kijelzőre, mert a rendszer kiértékelése a kijelzők kiértékelésével kezdődik.


\subsection{\bf SequenceGeneratorStepper}
Jelgenerátor léptető objektum. Feladata, hogy a szimulációt utasítsa, hogy az áramkörben megtalálható jelgenerátorokat léptesse.

\subsection{\bf SequenceGenerator}
Jelgenerátor, az áramkört felépítő egyik alapelem, kiértékelési kezdeményezés hatására az előre betáplált jelsorozatot soron következő elemét állítja be aktuális értékként, így azon komponensek melyek bemenetére a Jelgenerátor van kötve, elérik aktuális értékét. Bemenete nem komponens jellegű így nem kezel más komponenseket.

\subsection{\bf AndGate}
ÉS kapu, az áramkör egyik alapeleme. Bemeneteire kötött komponensek kiértékelését kezdeményezi, s a kapott értékek logikai ÉS kapcsolatát valósítja meg, ezáltal a kimenetére kötött komponens eléri az aktuális értékét.Figyeli hogy ha már kiértékelődött akkor nem kezdeményezi a bemenetére kötött komponensek kiértékelését.

\subsection{\bf OrGate}
VAGY kapu, az áramkör egyik alapeleme. Bemeneteire kötött komponensek kiértékelését kezdeményezi, s a kapott értékek logikai VAGY kapcsolatát valósítja meg, ezáltal a kimenetére kötött komponens eléri az aktuális értékét.Figyeli hogy ha már kiértékelődött akkor nem kezdeményezi a bemenetére kötött komponensek kiértékelését.

\subsection{\bf Inverter}
Invertáló, az áramkör alapelemei közé tartozik. A bemenetére érkező jel logikai negáltját valósítja meg, így a kimenetén levő komponens eléri aktuális értékét.

\subsection{\bf Gnd}
Föld, az áramkört felépítő egyik elem, aktuális értéke minden kiértékelési kérésre logikai hamis. Bemenete nem létezik, így nem kezdeményez további kiértékeléseket. Állandó értéke logikai hamis.

\subsection{\bf Vcc}
Áramkör alapeleme, mely kiértékelési kezdeményezésre aktuális értékét logikai igaz ra állítja be. Állandó értéke logikai igaz.

\subsection{\bf Led}
Egy kijelző az áramkör alapeleme, bemenetére kötött komponens kiértékelését kezdeményezi, és ezáltal az aktuális értékét egy a felhasználó számára érzékelhető módon kijelzi.

\subsection{\bf Toggle}
Kapcsoló, az áramkört felépítő elem, felhasználói interakciót követően, az aktuális értékét lehet állítani. Komponens bemenete nincs, így nem kezel további komponenseket.

\section{Osztályok leírása}
\comment{Az előző alfejezetben tárgyalt objektumok felelősségének formalizálása attribútumokká, metódusokká. Csak publikus metódusok szerepelhetnek. Ebben az alfejezetben megjelennek az interfészek, az öröklés, az absztrakt osztályok. Segédosztályokra még mindig nincs szükség. Az osztályok ABC sorrendben kövessék egymást. Interfészek esetén az Interfészek, Attribútumok pontok kimaradnak.}

\subsection{Osztály1}
\begin{itemize}
\item Felelősség\\
\comment{Mi az osztály felelőssége. Kb 1 bekezdés.}
\item Ősosztályok\\
\comment{Mely osztályokból származik (öröklési hierarchia)\newline
Legősebb osztály $\rightarrow$ Ősosztály2 $\rightarrow$ Ősosztály3...}
\item Interfészek\\
\comment{Mely interfészeket valósítja meg.}
\item Attribútumok\\
\comment{Milyen attribútumai vannak}
	\begin{itemize}
		\item attribútum1: attribútum jellemzése: mire való
		\item attribútum2: attribútum jellemzése: mire való
	\end{itemize}
\item Metódusok\\
\comment{Milyen publikus metódusokkal rendelkezik. Metódusonként egy-három mondat arról, hogy a metódus mit csinál.}
	\begin{itemize}
		\item int foo(Osztály3 o1, Osztály4 o2): metódus leírása
		\item int bar(Osztály5 o1): metódus leírása
	\end{itemize}
\end{itemize}

\subsection{Osztály2}
\begin{itemize}
\item Felelősség\\
\comment{Mi az osztály felelőssége. Kb 1 bekezdés.}
\item Ősosztályok\\
\comment{Mely osztályokból származik (öröklési hierarchia)\newline
Legősebb osztály $\rightarrow$ Ősosztály2 $\rightarrow$ Ősosztály3...}
\item Interfészek\\
\comment{Mely interfészeket valósítja meg.}
\item Attribútumok\\
\comment{Milyen attribútumai vannak}
	\begin{itemize}
		\item attribútum1: attribútum jellemzése: mire való
		\item attribútum2: attribútum jellemzése: mire való
	\end{itemize}
\item Metódusok\\
\comment{Milyen publikus metódusokkal rendelkezik. Metódusonként egy-három mondat arról, hogy a metódus mit csinál.}
	\begin{itemize}
		\item int foo(Osztály3 o1, Osztály4 o2): metódus leírása
		\item int bar(Osztály5 o1): metódus leírása
	\end{itemize}
\end{itemize}

\section{Statikus struktúra diagramok}
\comment{Az előző alfejezet osztályainak kapcsolatait és publikus metódusait bemutató osztálydiagram(ok). Tipikus hibalehetőségek: csillag-topológia, szigetek.}

\begin{figure}[h]
\begin{center}
%\includegraphics[width=17cm]{chapters/chapter03/example.pdf}
\caption{x}
\label{fig:example1}
\end{center}
\end{figure}

\section{Szekvencia diagramok}
\comment{Inicializálásra, use-case-ekre, belső működésre. Konzisztens kell legyen az előző alfejezettel. Minden metódus, ami ott szerepel, fel kell tűnjön valamelyik szekvenciában. Minden metódusnak, ami szekvenciában szerepel, szereplnie kell a valamelyik osztálydiagramon.}

\begin{figure}[h]
\begin{center}
%\includegraphics[width=17cm]{chapters/chapter03/example.pdf}
\caption{x}
\label{fig:example2}
\end{center}
\end{figure}

\section{State-chartok}
\comment{Csak azokhoz az osztályokhoz, ahol van értelme. Egyetlen állapotból álló state-chartok ne szerepeljenek. A játék működését bemutató state-chart-ot készíteni tilos.}

\begin{figure}[h]
\begin{center}
%\includegraphics[width=17cm]{chapters/chapter03/example.pdf}
\caption{x}
\label{fig:example3}
\end{center}
\end{figure}

