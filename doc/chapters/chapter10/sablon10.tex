% Szglab4
% ===========================================================================
%
\chapter{Prototípus beadása}

\thispagestyle{fancy}

\section{Fordítási és futtatási útmutató}
\comment{A feltöltött program fordításával és futtatásával kapcsolatos útmutatás. Ennek tartalmaznia kell leltárszerűen az egyes fájlok pontos nevét, méretét byte-ban, keletkezési idejét, valamint azt, hogy a fájlban mi került megvalósításra.}

\subsection{Fájllista}

\begin{fajllista}

\fajl
{compile.bat} % Kezdet
{178 byte} % Idptartam
{2011.03.19.~11:11~} % Résztvevők
{Fordításra használt batch fájl} % Leírás
\fajl
{doc.bat} % Kezdet
{276 byte} % Idptartam
{2011.03.19.~11:11~} % Résztvevők
{Dokumentáció generálására készített batch fájl} % Leírás
\fajl
{run.bat} % Kezdet
{106 byte} % Idptartam
{2011.03.19.~11:11~} % Résztvevők
{Futtatáshoz használt batch fájl} % Leírás

\fajl
{src/logsim/Config.java} % Kezdet
{6246 byte} % Idptartam
{2011.03.13.~14:34~} % Résztvevők
{A kapcsolók és szekvenciagenerátorok kimentéséért és betöltéséért felelős} % Leírás

\fajl
{src/logsim/Controller.java} % Kezdet
{6246 byte} % Idptartam
{2011.03.13.~14:34~} % Résztvevők
{A vezérlés interfészét tartalmazza} % Leírás

\fajl
{src/logsim/Parser.java} % Kezdet
{362 byte} % Idptartam
{2011.03.13.~14:53~} % Résztvevők
{Az áramkörleíró fájl feldolgozását végzi} % Leírás

\fajl
{src/logsim/Proto.java} % Kezdet
{604 byte} % Idptartam
{2011.03.13.~15:49~} % Résztvevők
{A szimuláció működéséért felelős; felhasználói utasítások értelmezése} % Leírás

\fajl
{src/logsim/View.java} % Kezdet
{4405 byte} % Idptartam
{2011.03.13.~14:36~} % Résztvevők
{Konkrét kimeneti implementáció; a konzolos megjelenítésért és fájlba írásért felelős} % Leírás

\fajl
{src/logsim/Viewable.java} % Kezdet
{1684 byte} % Idptartam
{2011.03.13.~14:36~} % Résztvevők
{A kimenet interfésze} % Leírás

\fajl
{src/logsim/model/Circuit.java} % Kezdet
{721 byte} % Idptartam
{2011.03.13.~14:36~} % Résztvevők
{Áramkört reprezentáló osztály} % Leírás

\fajl
{src/logsim/model/Simulation.java} % Kezdet
{721 byte} % Idptartam
{2011.03.13.~14:36~} % Résztvevők
{Egy szimulációt reprezentáló osztály} % Leírás

\fajl
{src/logsim/model/Value.java} % Kezdet
{721 byte} % Idptartam
{2011.03.13.~14:36~} % Résztvevők
{Az áramkörben előforduló értkékeket tartalmazó osztály} % Leírás

\fajl
{src/logsim/model/component/\newline
AbstractComponent.java} % Kezdet
{3379 byte} % Idptartam
{2011.03.13.~14:46~} % Résztvevők
{Az alkatrészek absztrakt ősosztálya} % Leírás

\fajl
{src/logsim/model/component/\newline
Composite.java} % Kezdet
{767 byte} % Idptartam
{2011.03.13.~14:46~} % Résztvevők
{A kompozit elem leírása} % Leírás

\fajl
{src/logsim/model/component/\newline
DisplayComponent.java} % Kezdet
{767 byte} % Idptartam
{2011.03.13.~14:46~} % Résztvevők
{Megjelenítő típusú alkatrészek absztrakt ősosztálya} % Leírás

\fajl
{src/logsim/model/component/\newline
FlipFlop.java} % Kezdet
{767 byte} % Idptartam
{2011.03.13.~14:46~} % Résztvevők
{Flipflop típusú alkatrészek absztrakt ősosztálya} % Leírás

\fajl
{src/logsim/model/component/\newline
SourceComponent.java} % Kezdet
{1119 byte} % Idptartam
{2011.03.13.~14:46~} % Résztvevők
{Forrás típusú alkatrészek absztrakt ősosztálya} % Leírás

\fajl
{src/logsim/model/component/\newline
Wire.java} % Kezdet
{1295 byte} % Idptartam
{2011.03.13.~14:46~} % Résztvevők
{Vezetéket megvalósító osztály} % Leírás

\fajl
{src/logsim/model/component/\newline
impl/AndGate.java} % Kezdet
{946 byte} % Idptartam
{2011.03.13.~14:46~} % Résztvevők
{Az ÉS kapu alkatrészt megvalósító osztály} % Leírás

\fajl
{src/logsim/model/component/\newline
impl/FlipFlopD.java} % Kezdet
{946 byte} % Idptartam
{2011.03.13.~14:46~} % Résztvevők
{A alkatrészt megvalósító osztály} % Leírás

\fajl
{src/logsim/model/component/\newline
impl/FlipFlopJK.java} % Kezdet
{946 byte} % Idptartam
{2011.03.13.~14:46~} % Résztvevők
{Az inverter alkatrészt megvalósító osztály} % Leírás

\fajl
{src/logsim/model/component/\newline
impl/Inverter.java} % Kezdet
{946 byte} % Idptartam
{2011.03.13.~14:46~} % Résztvevők
{Az inverter alkatrészt megvalósító osztály} % Leírás

\fajl
{src/logsim/model/component/\newline
impl/Led.java} % Kezdet
{833 byte} % Idptartam
{2011.03.13.~14:46~} % Résztvevők
{A led megjelenítőt megvalósító osztály} % Leírás

\fajl
{src/logsim/model/component/\newline
impl/Node.java} % Kezdet
{1110 byte} % Idptartam
{2011.03.13.~14:46~} % Résztvevők
{Csomópont alkatrészt megvalósító osztály} % Leírás

\fajl
{src/logsim/model/component/\newline
impl/OrGate.java} % Kezdet
{1109 byte} % Idptartam
{2011.03.13.~14:46~} % Résztvevők
{VAGY kaput megvalósító osztály} % Leírás

\fajl
{src/logsim/model/component/\newline
impl/Toggle.java} % Kezdet
{1686 byte} % Idptartam
{2011.03.13.~14:46~} % Résztvevők
{A kapcsolót megvalósító osztály} % Leírás

\end{fajllista}

\subsection{Fordítás}
\comment{A fenti listában szereplő forrásfájlokból milyen műveletekkel lehet a bináris, futtatható kódot előállítani. Az előállításhoz csak a 2. Követelmények c. dokumentumban leírt környezetet szabad előírni.}

\lstset{escapeinside=`', xleftmargin=10pt, frame=single, basicstyle=\ttfamily\footnotesize, language=sh}
\begin{lstlisting}
javac -d bin *.java
\end{lstlisting}

\subsection{Futtatás}
\comment{A futtatható kód elindításával kapcsolatos teendők leírása. Az indításhoz csak a 2. Követelmények c. dokumentumban leírt környezetet szabad előírni.}

\lstset{escapeinside=`', xleftmargin=10pt, frame=single, basicstyle=\ttfamily\footnotesize, language=sh}
\begin{lstlisting}
cd bin
java Main.java
\end{lstlisting}










\section{Tesztek jegyzőkönyvei}

\subsection{1 teszteset - Kapcsoló, és kapu és LED működésének vizsgálata}
\tesztok{Apagyi Gábor}{2011.04.16}

\subsection{2 teszteset - Multiplexer és 7 szegmenses kijelző vizsgálata}
\tesztfail{Apagyi Gábor}{2011.04.16. 11 óra 14 perc}{Fordítás idejű hiba}{Mivel az eddigi tesztek sikeresen lefutottak valószínűleg a bemeneti fájlokkal lehet gond, illetve esetleg a multiplexer, vagy a 7 szegmensen kijelző implementációjával.}{Az előző részben definiált felhasználói bemenetben az egyik kapcsolóra való hivatkozáskor rossz nevet írtunk a tesztesetbe: seg2, a helyes az áramkör létrehozásakor megadott seg név. A hibát javítva a teszt sikeresen lefutott.}

\tesztfail{Apagyi Gábor}{2011.04.16. 11 óra 22 perc}{Kimenet nem megfelelő}{Mivel a 7 szegmenses kijelző kimenetén az 1-esek száma megfelelő, valószínűleg, hogy a multiplexer belső logikájával lesz a probléma, azon belül is a kiválasztó jel és kimenet hozzárendeléssel}{Valóban a multiplexer implementációja volt hibás, a belső kiválasztó logikában a sorrend megcserélődött, ezt  át kellett írni:\newline\newline

    private static final int DATA0 = 1;\newline
    private static final int DATA1 = 2;\newline
    private static final int DATA2 = 3;\newline
    private static final int DATA3 = 4;\newline
    private static final int SEL0 = 5;\newline
    private static final int SEL1 = 6;\newline\newline
    
Illetve a számolásnál a selected változó értékét 0-ról indítottuk, azonban a megfelelő tömb indexelésnél ez 1-ről indul.\newline\newline
    
    int selected = 1;\newline
        if (getInput(SEL0) == Value.TRUE) {\newline
            selected += 1;\newline
        }\newline
        if (getInput(SEL1) == Value.TRUE) {\newline
            selected += 2;\newline
        }\newline\newline

 A hibát javítva a teszt sikeresen és jó eredménnyel lefutott.}

\tesztok{Apagyi Gábor}{2011.04.16}

\subsection{3 teszteset - Visszacsatolt vagy kapu vizsgálata - STABIL}
\tesztok{Apagyi Gábor}{2011.04.16}

\subsection{4 teszteset - Visszacsatolt és kapu és inverter vizsgálata -  NEM STABIL}
\tesztok{Apagyi Gábor}{2011.04.16}

\subsection{5 teszteset - Kompozitos áramkör vizsgálata}
\tesztok{Apagyi Gábor}{2011.04.16}

\subsection{5 teszteset - Kompoziton belül kompozit áramkör vizsgálata}
\tesztok{Apagyi Gábor}{2011.04.16}












\section{Értékelés}
\comment{A projekt kezdete óta az értékelésig eltelt időben tagokra bontva, százalékban.}

\begin{ertekeles}
\tag{Horváth} % Tag neve
{23.5}        % Munka szazalekban
\tag{Német}
{24.5}
\tag{Tóth}
{25}
\tag{Oláh}
{27}
\end{ertekeles}

