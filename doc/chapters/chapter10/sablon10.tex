% Szglab4
% ===========================================================================
%
\chapter{Prototípus beadása}

\thispagestyle{fancy}

\section{Fordítási és futtatási útmutató}

\subsection{Fájllista}

\begin{fajllista}

\fajl
{compile.bat} % Kezdet
{178 byte} % Idptartam
{2011.04.18.~11:27~} % Résztvevők
{Fordításra használt batch fájl} % Leírás
\fajl
{doc.bat} % Kezdet
{276 byte} % Idptartam
{2011.04.18.~11:27~} % Résztvevők
{Dokumentáció generálására készített batch fájl} % Leírás
\fajl
{run.bat} % Kezdet
{106 byte} % Idptartam
{2011.04.18.~11:27~} % Résztvevők
{Futtatáshoz használt batch fájl} % Leírás
\fajl
{runtests.bat} % Kezdet
{106 byte} % Idptartam
{2011.04.18.~11:27~} % Résztvevők
{Az összes teszteset lefuttatásához használt batch file} % Leírás
\fajl
{verifytests.bat} % Kezdet
{106 byte} % Idptartam
{2011.04.18.~11:27~} % Résztvevők
{A kiemenetek ellenőrzésére szolgáló batch file} % Leírás

\fajl
{src/logsim/Config.java} % Kezdet
{4195 byte} % Idptartam
{2011.04.05.~11:54~} % Résztvevők
{A kapcsolók és szekvenciagenerátorok kimentéséért és betöltéséért felelős} % Leírás

\fajl
{src/logsim/Controller.java} % Kezdet
{390 byte} % Idptartam
{2011.04.05.~00:52~} % Résztvevők
{A vezérlés interfészét tartalmazza} % Leírás

\fajl
{src/logsim/Parser.java} % Kezdet
{10347 byte} % Idptartam
{2011.04.17.~21:58~} % Résztvevők
{Az áramkörleíró fájl feldolgozását végzi} % Leírás

\fajl
{src/logsim/Proto.java} % Kezdet
{5402 byte} % Idptartam
{2011.04.18.~11:27~} % Résztvevők
{A szimuláció működéséért felelős; felhasználói utasítások értelmezése} % Leírás

\fajl
{src/logsim/View.java} % Kezdet
{6712 byte} % Idptartam
{2011.04.16.~16:10~} % Résztvevők
{Konkrét kimeneti implementáció; a konzolos megjelenítésért és fájlba írásért felelős} % Leírás

\fajl
{src/logsim/Viewable.java} % Kezdet
{2475 byte} % Idptartam
{2011.04.16.~16:08~} % Résztvevők
{A kimenet interfésze} % Leírás

\fajl
{src/logsim/model/Circuit.java} % Kezdet
{297 byte} % Idptartam
{2011.04.05.~00:52~} % Résztvevők
{Áramkört reprezentáló osztály} % Leírás

\fajl
{src/logsim/model/\newline
Simulation.java} % Kezdet
{855 byte} % Idptartam
{2011.04.05.~00:52~} % Résztvevők
{Egy szimulációt reprezentáló osztály} % Leírás

\fajl
{src/logsim/model/Value.java} % Kezdet
{714 byte} % Idptartam
{2011.04.05.~00:52~} % Résztvevők
{Az áramkörben előforduló értkékeket tartalmazó osztály} % Leírás

\fajl
{src/logsim/model/component/\newline
AbstractComponent.java} % Kezdet
{4588 byte} % Idptartam
{2011.04.16.~16:19~} % Résztvevők
{Az alkatrészek absztrakt ősosztálya} % Leírás

\fajl
{src/logsim/model/component/\newline
Composite.java} % Kezdet
{14385 byte} % Idptartam
{2011.04.17.~21:58~} % Résztvevők
{A kompozit elem leírása} % Leírás

\fajl
{src/logsim/model/component/\newline
DisplayComponent.java} % Kezdet
{671 byte} % Idptartam
{2011.04.05.~00:52~} % Résztvevők
{Megjelenítő típusú alkatrészek absztrakt ősosztálya} % Leírás

\fajl
{src/logsim/model/component/\newline
FlipFlop.java} % Kezdet
{1688 byte} % Idptartam
{2011.04.16.~16:21~} % Résztvevők
{Flipflop típusú alkatrészek absztrakt ősosztálya} % Leírás

\fajl
{src/logsim/model/component/\newline
SourceComponent.java} % Kezdet
{1099 byte} % Idptartam
{2011.04.05.~11:18~} % Résztvevők
{Forrás típusú alkatrészek absztrakt ősosztálya} % Leírás

\fajl
{src/logsim/model/component/\newline
Wire.java} % Kezdet
{600 byte} % Idptartam
{2011.04.04.~12:39~} % Résztvevők
{Vezetéket megvalósító osztály} % Leírás

\fajl
{src/logsim/model/component/\newline
impl/AndGate.java} % Kezdet
{915 byte} % Idptartam
{2011.04.04.~12:39~} % Résztvevők
{Az ÉS kapu alkatrészt megvalósító osztály} % Leírás

\fajl
{src/logsim/model/component/\newline
impl/FlipFlopD.java} % Kezdet
{870 byte} % Idptartam
{2011.04.05.~00:52~} % Résztvevők
{A D flipflop alkatrészt megvalósító osztály} % Leírás

\fajl
{src/logsim/model/component/\newline
impl/FlipFlopJK.java} % Kezdet
{1453 byte} % Idptartam
{2011.04.05.~00:52~} % Résztvevők
{A JK flipflop alkatrészt megvalósító osztály} % Leírás

\fajl
{src/logsim/model/component/\newline
impl/Gnd.java} % Kezdet
{565 byte} % Idptartam
{2011.04.04.~12:39~} % Résztvevők
{A permanens logikai nullát megvalósító osztály} % Leírás

\fajl
{src/logsim/model/component/\newline
impl/Inverter.java} % Kezdet
{638 byte} % Idptartam
{2011.04.05.~00:52~} % Résztvevők
{Az inverter alkatrészt megvalósító osztály} % Leírás

\fajl
{src/logsim/model/component/\newline
impl/Led.java} % Kezdet
{846 byte} % Idptartam
{2011.04.05.~00:52~} % Résztvevők
{A led megjelenítőt megvalósító osztály} % Leírás

\fajl
{src/logsim/model/component/\newline
impl/Mpx.java} % Kezdet
{1222 byte} % Idptartam
{2011.04.04.~12:39~} % Résztvevők
{A multiplexer alkatrészt megvalósító osztály} % Leírás

\fajl
{src/logsim/model/component/\newline
impl/Node.java} % Kezdet
{1012 byte} % Idptartam
{2011.04.05.~00:52~} % Résztvevők
{Csomópont alkatrészt megvalósító osztály} % Leírás

\fajl
{src/logsim/model/component/\newline
impl/OrGate.java} % Kezdet
{983 byte} % Idptartam
{2011.04.05.~00:52~} % Résztvevők
{a VAGY kapu alkatrészt megvalósító osztály} % Leírás

\fajl
{src/logsim/model/component/\newline
impl/Scope.java} % Kezdet
{1839 byte} % Idptartam
{2011.04.05.~00:52~} % Résztvevők
{Oszcilloszkópot megvalósító osztály} % Leírás

\fajl
{src/logsim/model/component/\newline
impl/SequenceGenerator.java} % Kezdet
{2562 byte} % Idptartam
{2011.04.04.~12:39~} % Résztvevők
{A szekvenciagenerátor alkatrészt megvalósító osztály} % Leírás

\fajl
{src/logsim/model/component/\newline
impl/SevenSegmentDisplay.java} % Kezdet
{1115 byte} % Idptartam
{2011.04.04.~12:39~} % Résztvevők
{A 7 szegmenses kijelző alkatrészt megvalósító osztály} % Leírás

\fajl
{src/logsim/model/component/\newline
impl/Toggle.java} % Kezdet
{1585 byte} % Idptartam
{2011.04.05.~11:24~} % Résztvevők
{A kapcsolót megvalósító osztály} % Leírás

\fajl
{src/logsim/model/component/\newline
impl/Vcc.java} % Kezdet
{537 byte} % Idptartam
{2011.04.04.~12:39~} % Résztvevők
{A permanens logikai egyet megvalósító osztály} % Leírás

\fajl
{src/logsim/model/component/\newline
impl/Vcc.java} % Kezdet
{537 byte} % Idptartam
{2011.04.04.~12:39~} % Résztvevők
{A permanens logikai egyet megvalósító osztály} % Leírás

\fajl
{diff/cmp.exe} % Kezdet
{537 byte} % Idptartam
{2011.04.04.~12:39~} % Résztvevők
{A diffUtils összehasonlító exe-je} % Leírás

\fajl
{diff/libiconv2.dll} % Kezdet
{537 byte} % Idptartam
{2011.04.04.~12:39~} % Résztvevők
{a cmp.exe haszálja} % Leírás

\fajl
{diff/libintl3.dll} % Kezdet
{537 byte} % Idptartam
{2011.04.04.~12:39~} % Résztvevők
{a cmp.exe haszálja} % Leírás

\fajl
{tesztek/input1.txt} % Kezdet
{89 byte} % Idptartam
{2011.04.17.~21:58~} % Résztvevők
{Az 1. teszteset bemeneti parancsai} % Leírás

\fajl
{tesztek/input2.txt} % Kezdet
{131 byte} % Idptartam
{2011.04.17.~21:58~} % Résztvevők
{A 2. teszteset bemeneti parancsai} % Leírás

\fajl
{tesztek/input3.txt} % Kezdet
{49 byte} % Idptartam
{2011.04.17.~21:58~} % Résztvevők
{A 3. teszteset bemeneti parancsai} % Leírás

\fajl
{tesztek/input4.txt} % Kezdet
{43 byte} % Idptartam
{2011.04.17.~21:58~} % Résztvevők
{A 4. teszteset bemeneti parancsai} % Leírás

\fajl
{tesztek/input5.txt} % Kezdet
{99 byte} % Idptartam
{2011.04.17.~21:58~} % Résztvevők
{Az 5. teszteset bemeneti parancsai} % Leírás

\fajl
{tesztek/input6.txt} % Kezdet
{85 byte} % Idptartam
{2011.04.17.~21:58~} % Résztvevők
{A 6. teszteset bemeneti parancsai} % Leírás

\fajl
{tesztek/input7.txt} % Kezdet
{57 byte} % Idptartam
{2011.04.17.~21:58~} % Résztvevők
{A 7. teszteset bemeneti parancsai} % Leírás

\fajl
{tesztek/ref\_output1.txt} % Kezdet
{309 byte} % Idptartam
{2011.04.18.~11:27~} % Résztvevők
{Az 1. teszteset elvárt kimenete} % Leírás

\fajl
{tesztek/ref\_output2.txt} % Kezdet
{490 byte} % Idptartam
{2011.04.18.~11:27~} % Résztvevők
{A 2. teszteset elvárt kimenete} % Leírás

\fajl
{tesztek/ref\_output3.txt} % Kezdet
{117 byte} % Idptartam
{2011.04.18.~11:27~} % Résztvevők
{A 3. teszteset elvárt kimenete} % Leírás

\fajl
{tesztek/ref\_output4.txt} % Kezdet
{52 byte} % Idptartam
{2011.04.18.~11:27~} % Résztvevők
{A 4. teszteset elvárt kimenete} % Leírás

\fajl
{tesztek/ref\_output5.txt} % Kezdet
{408 byte} % Idptartam
{2011.04.18.~11:27~} % Résztvevők
{Az 5. teszteset elvárt kimenete} % Leírás

\fajl
{tesztek/ref\_output6.txt} % Kezdet
{505 byte} % Idptartam
{2011.04.18.~11:27~} % Résztvevők
{A 6. teszteset elvárt kimenete} % Leírás

\fajl
{tesztek/ref\_output7.txt} % Kezdet
{193 byte} % Idptartam
{2011.04.18.~11:27~} % Résztvevők
{A 7. teszteset elvárt kimenete} % Leírás

\fajl
{tesztek/test1.txt} % Kezdet
{78 byte} % Idptartam
{2011.04.17.~21:58~} % Résztvevők
{Az 1. teszteset áramköre} % Leírás

\fajl
{tesztek/test2.txt} % Kezdet
{221 byte} % Idptartam
{2011.04.17.~21:58~} % Résztvevők
{A 2. teszteset áramköre} % Leírás

\fajl
{tesztek/test3.txt} % Kezdet
{83 byte} % Idptartam
{2011.04.17.~21:58~} % Résztvevők
{A 3. teszteset áramköre} % Leírás

\fajl
{tesztek/test4.txt} % Kezdet
{96 byte} % Idptartam
{2011.04.17.~21:58~} % Résztvevők
{A 4. teszteset áramköre} % Leírás

\fajl
{tesztek/test5.txt} % Kezdet
{89 byte} % Idptartam
{2011.04.17.~21:58~} % Résztvevők
{Az 5. teszteset áramköre} % Leírás

\fajl
{tesztek/test6.txt} % Kezdet
{263 byte} % Idptartam
{2011.04.17.~21:58~} % Résztvevők
{A 6. teszteset áramköre} % Leírás

\fajl
{tesztek/test7.txt} % Kezdet
{161 byte} % Idptartam
{2011.04.17.~21:58~} % Résztvevők
{A 7. teszteset áramköre} % Leírás

\end{fajllista}

\subsection{Fordítás}
A hibamentes és minél inkább gördülékeny fordítás érdekében létrehoztunk egy \texttt{compile.bat} nevezetű batch fájlt, mely a projekt főkönyvtárában található. Projekt főkönytára az, amelyik a batch fájlokat és a "src" nevezetű mappát tartalmazza, melyben a program forráskódja található. Szükség estén kézzel kell módosítani a batch fájl
\begin{verbatim}
set C="C:\Program Files\Java\jdk1.6.0_24\bin\" 
\end{verbatim}
sorát, attól függően, hogy a gépen éppen melyik Java JDK verzió található és az hová van telepítve!\\

A \texttt{compile.bat} fájl az alábbi parancsokat hajtja végre:
\lstinputlisting[escapeinside=`', xleftmargin=10pt, frame=single, basicstyle=\ttfamily\footnotesize, language=sh]{../LogSimProto/compile.bat}
Ha hibamentes volt a fordítás, a "Fordítás sikeres" kimenettel értesíti a felhasználót.\\

A fordítás sikeressége után, lehetőség van a dokumentáció legenerálására is. Ehhez felhasználható a főkönyvtárban található \texttt{doc.bat} batch fájl.
Szükség estén kézzel kell módosítani a batch file \begin{verbatim}
set C="C:\Program Files\Java\jdk1.6.0_24\bin\" 
\end{verbatim}
sorát, attól függően, hogy a gépen éppen melyik Java JDK verzió található és az hová van telepítve!\\

A batch fájl az alábbi parancsokat hajtja végre:
\lstinputlisting[escapeinside=`', xleftmargin=10pt, frame=single, basicstyle=\ttfamily\footnotesize, language=sh]{../LogSimProto/doc.bat}
Ha a dokumentum generálás sikeres volt, akkor a documents nevezetű mappában megtaláhatóak a kívánt dokumentumok.



\subsection{Futtatás}
A futtatás és a beépített tesztesetek ellenőrzésének megkönnyítése érdekében elkészítettük a \texttt{run.bat}, \texttt{runtests.bat} és a \texttt{verifytests.bat} batch fájlt.
Szükség estén kézzel kell módosítani a \texttt{run.bat} batch fájlt 
\begin{verbatim}
set C="C:\Program Files\Java\jdk1.6.0_24\bin\" 
\end{verbatim} sorát, attól függően, hogy a gépen éppen melyik Java JDK verzió található és az hová van telepítve!\\

A \texttt{runtests.bat} fájl lefuttatja a 7 beépített teszt esetet.
A \texttt{runtests.bat} fájl az alábbi parancsokat hajtja végre:
\lstinputlisting[escapeinside=`', xleftmargin=10pt, frame=single, basicstyle=\ttfamily\footnotesize, language=sh]{../LogSimProto/runtests.bat}
A "build" könyvtárba outputX.txt néven lesznek a tesztek kimenetei, ahol X a teszteset számát jelüli.

A \texttt{verifytests.bat} fájl összehasonlítja a 7 beépített teszteset kimenetét a referencia kimenetekkel, majd egyenkét kiírja, hogy a tesztesetek megegyeznek-e a referenciával.
A \texttt{verifytests.bat} fájl az alábbi parancsokat hajtja végre:
\lstinputlisting[escapeinside=`', xleftmargin=10pt, frame=single, basicstyle=\ttfamily\footnotesize, language=sh]{../LogSimProto/verifytests.bat}


A \texttt{run.bat} fájl az alábbi parancsokat hajtja végre:
\lstinputlisting[escapeinside=`', xleftmargin=10pt, frame=single, basicstyle=\ttfamily\footnotesize, language=sh]{../LogSimProto/run.bat}
A "build" könyvtárból elindítja az előzőleg lefordított programot.










\section{Tesztek jegyzőkönyvei}

\subsection{1. teszteset - Kapcsoló, és kapu és LED működésének vizsgálata}
\tesztok{Apagyi Gábor}{2011.04.16}

\subsection{2. teszteset - Multiplexer és 7 szegmenses kijelző vizsgálata}
\tesztfail{Apagyi Gábor}{2011.04.16. 11 óra 14 perc}{Futás idejű hiba}{Mivel az eddigi tesztek sikeresen lefutottak valószínűleg a bemeneti fájlokkal lehet gond, illetve esetleg a multiplexer, vagy a 7 szegmensen kijelző implementációjával.}{Az előző részben definiált felhasználói bemenetben az egyik kapcsolóra való hivatkozáskor rossz nevet írtunk a tesztesetbe: seg2, a helyes az áramkör létrehozásakor megadott seg név. A hibát javítva a teszt sikeresen lefutott.}

\tesztfail{Apagyi Gábor}{2011.04.16. 11 óra 22 perc}{Kimenet nem megfelelő}{Mivel a 7 szegmenses kijelző kimenetén az 1-esek száma megfelelő, valószínű, hogy a multiplexer belső logikájával lesz a probléma, azon belül is a kiválasztó jel és kimenet hozzárendeléssel}{Valóban a multiplexer implementációja volt hibás, a belső kiválasztó logikában a sorrend megcserélődött, ezt  át kellett írni:\newline\newline

    private static final int DATA0 = 1;\newline
    private static final int DATA1 = 2;\newline
    private static final int DATA2 = 3;\newline
    private static final int DATA3 = 4;\newline
    private static final int SEL0 = 5;\newline
    private static final int SEL1 = 6;\newline\newline
    
Illetve a számolásnál a selected változó értékét 0-ról indítottuk, azonban a megfelelő tömb indexelésnél ez 1-ről indul.\newline\newline
    
    int selected = 1;\newline
        if (getInput(SEL0) == Value.TRUE) {\newline
            selected += 1;\newline
        }\newline
        if (getInput(SEL1) == Value.TRUE) {\newline
            selected += 2;\newline
        }\newline\newline

 A hibát javítva a teszt sikeresen és jó eredménnyel lefutott.}

\tesztok{Apagyi Gábor}{2011.04.16}

\subsection{3. teszteset - Visszacsatolt vagy kapu vizsgálata - STABIL}
\tesztok{Apagyi Gábor}{2011.04.16}

\subsection{4. teszteset - Visszacsatolt és kapu és inverter vizsgálata -  NEM STABIL}
\tesztok{Apagyi Gábor}{2011.04.16}

\subsection{5. teszteset - JK Flip-flop és Scope vizsgálata}
\tesztok{Jákli Gábor}{2011.04.18}

\subsection{6. teszteset - Kompozitos áramkör vizsgálata}
\tesztok{Apagyi Gábor}{2011.04.16}

\subsection{7. teszteset - Kompoziton belül kompozit áramkör vizsgálata}
\tesztok{Apagyi Gábor}{2011.04.16}












\section{Értékelés}

\begin{ertekeles}
\tag{Apagyi G.} % Tag neve
{13}        % Munka szazalekban
\tag{Dévényi A.}
{22}
\tag{Jákli G.}
{22}
\tag{Kriván B.}
{30}
\tag{Péter T.}
{13}
\end{ertekeles}

