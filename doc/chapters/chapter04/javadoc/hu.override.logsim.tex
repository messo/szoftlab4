\subsection{Circuit}
\begin{itemize}
\item Felelősség\\
Feladata a jelgenerátor léptető kérésére a jelgenerátorok léptetése, a feldolgozó  által létrehozott komponensek felvétele az áramkörbe, illetve ezek utasítása arra,  hogy töröljék a "már kiértékelve" flaget egy adott kiértékelési ciklus előtt, hogy ezáltal a  ciklusban minden kimenet értéke frissülhessen.  Továbbá feladata a kiértékelés elindítása az összes kijelzőre, mert a rendszer kiértékelése  a kijelzők kiértékelésével kezdődik.
\item Ősosztályok:\ Object $\rightarrow{}$ Circuit.
\item Interfészek: (nincs)
\item Attribútumok $\ $
\begin{itemize}
	\item \texttt{private HashMap componentMap} Komponenseket tartalmazó HashMap
	\item \texttt{private List displays} Megjelenítő típusú komponensek
	\item \texttt{private List sources} Jelforrás típusú komponensek
\end{itemize}
\item Metódusok$\ $
\begin{itemize}
	\item \texttt{AbstractComponent addComponent(AbstractComponent component)}: Komponens hozzáadása az áramkörhöz.
	\item \texttt{void doEvaluationCycle()}: Egy kiértékelési ciklus lefuttatása. Az áramkörtől ezután lekérdezhető, hogy  stabil (nem változott semelyik komponens kimenete az utolsó futtatás óta)  vagy instabil állapotban van-e.
	\item \texttt{List getDisplays()}: Megjelenítő típusú komponeseket adja vissza.
	\item \texttt{List getSources()}: Jelforrás típusú komponenseket adja vissza.
	\item \texttt{void stepGenerators()}: Jelgenerátorok a szimuláció szemszögéből nézve, egyszerre történő  léptetése.
	\item \texttt{void init()}: Előre meghatározott áramköri modell felépítése, objektumok létrehozása, kapcsolatok beállítása
\end{itemize}
\end{itemize}

\subsection{Simulation}
\begin{itemize}
\item Felelősség\\
Egy szimulációt reprezentáló objektum.  Futásakor elindítja a jelgenerátor léptetőt, s utasítja az áramkört több kiértékelési  ciklus lefuttatásához, amíg az áramkörben van változás. Ha a változás megadott lépésen belül  nem áll meg, tájékoztatja a felhasználót, hogy nincs stacionárius állapot.  Amikor leállítódik, a jelgenerátor-léptetőt is leállítja.  A szál természetéből adódóan többet nem indítható el, új szimulációhoz új példányt kell létrehozni.
\item Ősosztályok:\ Object $\rightarrow{}$ Thread $\rightarrow{}$ Simulation.
\item Interfészek: (nincs)
\item Attribútumok $\ $
\begin{itemize}
	\item \texttt{private Circuit circuit} Szimulált áramkör
	\item \texttt{private State state} Szimuláció jelenlegi állapota
\end{itemize}
\item Metódusok$\ $
\begin{itemize}
	\item \texttt{Circuit getCircuit()}: Szimulált áramkör lekérdezése
	\item \texttt{void start()}: Szimuláció elindítása
	\item \texttt{void getState(State state)}: Szimuláció állapotának lekérdezése
\end{itemize}
\end{itemize}

\subsection{Simulation.State}
\begin{itemize}
\item Felelősség\\
Szimuláció állapotait írja le
\item Ősosztályok:\ Object $\rightarrow{}$ Enum $\rightarrow{}$ Simulation.State.
\item Interfészek: (nincs)
\item Attribútumok $\ $
\begin{itemize}
	\item \texttt{public static final State PAUSED} Szimuláció szüneteltetve van, a következő jelforrás változásig.
	\item \texttt{public static final State STOPPED} Szimuláció leállt, ahhoz, hogy bármi történjen az áramkörre újra kell indítani.
	\item \texttt{public static final State WORKING} Szimuláció éppen dolgozik, egy konkrét jelforrás-kombinációt alkalmazva dolgoztatja az áramkört
\end{itemize}
\item Metódusok$\ $
\begin{itemize}
	\item \texttt{static State valueOf(String name)}: 
 % TODO
	\item \texttt{static State[] values()}: 
 % TODO
\end{itemize}
\end{itemize}

\subsection{Value}
\begin{itemize}
\item Felelősség\\
Az áramkörben előfordulható értéket reprezentál.
\item Ősosztályok:\ Object $\rightarrow{}$ Enum $\rightarrow{}$ Value.
\item Interfészek: (nincs)
\item Attribútumok $\ $
\begin{itemize}
	\item \texttt{public static final Value FALSE} 
 % TODO
	\item \texttt{public static final Value TRUE} 
 % TODO
\end{itemize}
\item Metódusok$\ $
\begin{itemize}
	\item \texttt{Value invert()}: Invertálja az adott értéket. Ennek addig van értelme, amíg 2 féle  állapot fordulhat elő a rendszerben.
\end{itemize}
\end{itemize}

