\subsection{Circuit}
\begin{itemize}
\item Felelősség\\
A komponensek létrehozása és összekötése a szimuláció elején, szimuláció által indított kiértékelési ciklus végrehajtása.
\item Ősosztályok: (nincs)
\item Interfészek: (nincs)
\item Attribútumok $\ $
\begin{itemize}
	\item \texttt{private Map componentMap}: Komponenseket tartalmazó Map, mely segítségével név szerint megtalálható egy komponens.
	\item \texttt{private List displays}: Megjelenítő típusú komponensek listája (kijelző, 7-szegmenses kijelző)
	\item \texttt{private List sources}: Jelforrás típusú komponensek listája (kapcsoló, jelgenerátor)
	\item \texttt{private List flipFlops}: Flipflopok listája (D és JK flipflopok)
	\item \texttt{private List seqGens}: Jelgenerátorok listája
\end{itemize}
\item Metódusok$\ $
\begin{itemize}
	\item \texttt{void init()}: Áramkör inicializálása; komponensek létrehozása és összekötése.
	\item \texttt{void add(AbstractComponent component)}: Komponens hozzáadása a \texttt{componentMap}-hez. Ezt minden komponensre meg kell hívni.
	\item \texttt{void add(SourceComponent sc)}: Jelforrás hozzáadása a \texttt{sources} listához.
	\item \texttt{void add(FlipFlop ff)}: FlipFlop hozzáadása a \texttt{flipFlops} listához.
	\item \texttt{void add(SequenceGenerator sg)}: Jelgenerátor hozzáadása a \texttt{seqGens} listához.
	\item \texttt{void add(DisplayComponent dc)}: Megjelenítő hozzáadása a \texttt{displays} listához.
	\item \texttt{void doEvaluationCycle()}: Egy kiértékelési ciklus lefuttatása. Az áramkörtől ezután lekérdezhető, hogy változott-e a rendszer állapota, azaz valamelyik komponens eltérő kimenetet ad-e, mint az előző ciklusban.
	\item \texttt{boolean isChanged()}: Áramkör változásának lekérdezése. Igazzal tér vissza, ha van olyan komponens, ami azt jelzi magáról, hogy változott a kimenete.
	\item \texttt{void commitFlipFlops()}: A flipflopok jelenlegi kimenetének elmentése belső állapotnak, és az órajel bemenetén lévő érték eltárolása az éldetektálás érdekében.
	\item \texttt{void stepGenerators()}: Jelgenerátorok léptetése.
\end{itemize}
\end{itemize}

\subsection{Simulation}
\begin{itemize}
\item Felelősség\\
Az áramkörön kiértékelési ciklusok futtatása az adott áramkör bemenetekre (kapcsolók állapota, jelgenerátorok jelenlegi értéke) nézve addig, amíg az áramkör nem stabilizálódik.
\item Ősosztályok: (nincs)
\item Interfészek: (nincs)
\item Attribútumok $\ $
\begin{itemize}
	\item \texttt{private Circuit circuit}: Szimulált áramkör
	\item \texttt{private State state}: Szimuláció jelenlegi állapota
\end{itemize}
\item Metódusok$\ $
\begin{itemize}
	\item \texttt{Circuit getCircuit()}: Szimulált áramkör lekérdezése
	\item \texttt{void start()}: Szimuláció elindítása a jelenlegi áramköri bemenetekre (kapcsolók állapota, jelgenerátorok jelenlegi értéke). Amennyiben stacionárius állapot jött létre, léptetjük a jelgenerátorokat és elmentjük a flipflopok állapotát. Így újbóli hívásra már a következő időpillanatban érvényes áramköri bemenetekre lehet szimulálni az áramkört.
	\item \texttt{void getState(State state)}: Szimuláció állapotának lekérdezése.
\end{itemize}
\end{itemize}

\subsection{Simulation.State}
Enumeráció.
\begin{itemize}
\item Felelősség\\
Szimuláció állapotait írja le
\item Ősosztályok: (nincs)
\item Interfészek: (nincs)
\item Attribútumok $\ $
\begin{itemize}
	\item \texttt{public static final State READY} A szimuláció kész a futásra. Ilyenkor hívható rajta a start() metódus.
	\item \texttt{public static final State WORKING} Szimuláció éppen dolgozik, egy konkrét jelforrás-kombinációt alkalmazva szimulálja az áramkört.
	\item \texttt{public static final State FAILED} A szimuláció leállt, mert az áramkörnek nincs stacionárius állapota. A start() metódus újra hívható (ha a bemenetek nem változnak, továbbra is le fog állni).
\end{itemize}
\item Metódusok$\ $
\begin{itemize}
\item (nincs)
\end{itemize}
\end{itemize}

\subsection{Value}
\begin{itemize}
\item Felelősség\\
Az áramkörben előfordulható értéket reprezentál.
\item Ősosztályok:\ Object $\rightarrow{}$ Enum $\rightarrow{}$ Value.
\item Interfészek: (nincs)
\item Attribútumok $\ $
\begin{itemize}
	\item \texttt{public static final Value FALSE} 
 % TODO
	\item \texttt{public static final Value TRUE} 
 % TODO
\end{itemize}
\item Metódusok$\ $
\begin{itemize}
	\item \texttt{Value invert()}: Invertálja az adott értéket. Ennek addig van értelme, amíg 2 féle  állapot fordulhat elő a rendszerben.
\end{itemize}
\end{itemize}

