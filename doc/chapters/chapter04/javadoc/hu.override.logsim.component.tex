\subsection{AbstractComponent}
Absztrakt osztály.
\begin{itemize}
\item Felelősség\\
Egy komponens absztrakt megvalósítása, ebből származik az összes többi  komponens. A közös logikát valósítja meg. A gyakran használt dolgokra  ad alapértelmezett implementációt (összekötés, bemenetek kiértékelése stb.)
\item Ősosztályok:\ Object $\rightarrow{}$ AbstractComponent.
\item Interfészek: Component.
\item Attribútumok $\ $
\begin{itemize}
	\item \texttt{private boolean changed} Azt mutatja meg, hogy változott-e a komponens értéke
	\item \texttt{protected boolean alreadyEvaluated} "Kiértékelt" flag, ha ez be van billenve, akkor nem számolunk újra, csak visszaadjuk az előzőleg kiszámolt értéket.
	
	\item \texttt{protected Value[] Values} Jelenlegi (számolás közben) érték, ezt csak rövid ideig tároljuk, ahhoz kell, hogy tudjuk változik-e a lastValue-hoz képest.
	
	\item \texttt{protected int[] indices} Itt tároljuk, hogy melyik bemenetre, az adott komponens melyik kimenetét kötöttük.
	
	\item \texttt{protected AbstractComponent[] inputs} Az adott bemenetekre kötött komponensek.
	\item \texttt{protected String name} Komponens neve (változó neve, ahogy a leíróban azonosítjuk)
\end{itemize}
\item Metódusok$\ $
\begin{itemize}
	\item \texttt{void clearEvaluatedFlag()}: Töröljük a komponens "kiértékelt" flagjét.
	\item \texttt{Value evaluate()}: Lekérjük a 0. kimenetén lévő értéket.
	\item \texttt{String getName()}: Név lekérdezése
	\item \texttt{void setName(String name)}: Név beállítása (változó, amivel azonosítjuk)
	\item \texttt{void setInput(int inputSlot, AbstractComponent component)}: Shortcut a másik setInput()-hoz, outputPin = 0-val.
	\item \texttt{void setInput(int inputPin, AbstractComponent component, int outputPin)}: Beállítunk egy bemenetet.
	\item \texttt{boolean isChanged()}: Volt-e változás, vagy nem?

	
\end{itemize}
\end{itemize}

\subsection{Component}
Interfész.
\begin{itemize}
\item Felelősség\\
Komponens interfész, ebből származik az IsDisplay és az IsSource interfész.  Legalapvetőbb dolgokat írja le (minden komponensnek van neve és értékei).
\item Ősosztályok: (nincs)
\item Interfészek: (nincs)
\item Metódusok$\ $
\begin{itemize}
	\item \texttt{String getName()}: Név lekérdezése.
	\item \texttt{Value getValue()}: Értéke lekérdezése a 0. kimeneten.
	\item \texttt{Value getValue(int idx)}: Érték lekérdezése az adott kimeneten.
	\item \texttt{void setName(String name)}: Név beállítása.
\end{itemize}
\end{itemize}

\subsection{FlipFlop}
Absztrakt osztály.
\begin{itemize}
\item Felelősség\\
Flipflopok ősosztálya, minden flipflop 0. bemenete az órajel!
\item Ősosztályok:\ Object $\rightarrow{}$ AbstractComponent $\rightarrow{}$ FlipFlop.
\item Interfészek: (nincs)
\item Attribútumok $\ $
\begin{itemize}
	\item \texttt{private boolean active} Ebben tároljuk, hogy a FF számolhat-e vagy sem. (felfutó él)
	\item \texttt{protected Value q} A flip-flopban tárolt érték
\end{itemize}
\item Metódusok$\ $
\begin{itemize}
	\item \texttt{boolean getActive()}: Számolhat-e az FF? Ezt hívja meg az FF-ek onEvaluation() metódusa, mielőtt  bármit is csinálnának.
	\item \texttt{void setActive(boolean active)}: Felfutó élnél a SequenceGenerator-nak meg kell hívni ezt a hozzá kötött  FlipFlopokra, egyéb esetben törölnie az active flaget. Így tudja az FF, hogy  mikor kell ténylegesen számolnia.
	\item \texttt{void setInput(int inputPin, AbstractComponent component, int outputPin)}: Ősosztály implementációjának meghívása, illetve ha egy SequenceGeneratort kötünk éppen a CLK bemenetre, akkor az addFlipFlop() metódus meghívása rajta.
\end{itemize}
\end{itemize}

\subsection{IsDisplay}
Interfész.
\begin{itemize}
\item Felelősség\\
Megjelenítő típusú komponenst reprezentál. Ezt kell implementálnia a megjelenítőknek.
\item Ősosztályok:\ (nincs).
\item Interfészek: Component.
\item Metódusok$\ $
\begin{itemize}
\item (nincs)
\end{itemize}
\end{itemize}

\subsection{IsSource}
Interfész.
\begin{itemize}
\item Felelősség\\
Jelforrás típusú komponenst reprezentál. Ezt kell implementálnia a jelforrásoknak.
\item Ősosztályok:\ (nincs).
\item Interfészek: Component.
\item Metódusok$\ $
\begin{itemize}
	\item \texttt{public Value[] getValues()}: Lekérhetjük a jelforrás értékeit, hogy el tudjuk menteni.
	\item \texttt{public void setValues(Value[] values)}: Beállítjuk a jelforrás értékét. Kapcsoló esetén csak 1 elemű tömb  adható paraméterként!
\end{itemize}
\end{itemize}

