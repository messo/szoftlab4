\subsection{AbstractComponent}
Absztrakt osztály.
\begin{itemize}
\item Felelősség\\
Egy komponens absztrakt megvalósítása, ebből származik az összes többi  komponens. A közös logikát valósítja meg. A gyakran használt dolgokra  ad alapértelmezett implementációt (összekötés, bemenetek kiértékelése stb.)
\item Ősosztályok: (nincs)
\item Interfészek: (nincs)
\item Attribútumok $\ $
\begin{itemize}
	\item \texttt{protected String name}: Komponens neve (ahogy az áramkörben azonosítjuk)
	\item \texttt{private boolean changed}: Azt mutatja meg, hogy változott-e valamelyik kimenete a komponensnek. Ez a flag az \texttt{evaluate()} meghívásánál számolódik ki, vagyis azt jelzi, hogy két kiértékelés között változott-e a kimenet.
	\item \texttt{protected boolean alreadyEvaluated}: "Kiértékelt" flag, ha ez be van billenve, akkor nem számolunk újra, csak visszaadjuk az előzőleg kiszámolt értéket.
	\item \texttt{protected Value[] values}: Kimeneteken lévő értékek. Ez frissül az \texttt{evaluate()} meghívására, ha még nem volt kiértékelve.
	\item \texttt{protected AbstractComponent[] inputs}: A bemenetekre kötött komponensek.
	\item \texttt{protected int[] indices}: Itt tároljuk, hogy melyik bemenetre, az adott komponens melyik kimenetét kötöttük.
\end{itemize}
\item Metódusok$\ $
\begin{itemize}
	\item \texttt{String getName()}: Név lekérdezése
	\item \texttt{void setName(String name)}: Név beállítása (változó, amivel azonosítjuk)
	\item \texttt{addTo(Circuit c)}: Meghívja az áramkör \texttt{add(AbstractComponent ac)} metódusát.
	\item \texttt{Value[] evaluate()}: Komponens kimenetein lévő értékek újraszámolása (ha a kiértékelt flag, nincs beállítva) a bemenetek alapján. A kimeneteken lévő értékekkel tér vissza.
	\item \texttt{void clearEvaluatedFlag()}: Töröljük a komponens "kiértékelt" flagjét.
	\item \texttt{getValue(int idx)}: Visszaadja a paraméterben megadott indexű kimenet értékét.
	\item \texttt{boolean isChanged()}: Changed flag lekérdező metódusa.
	\item \texttt{void setInput(int inputPin, AbstractComponent component, int outputPin)}: Beállítunk egy bemenetet (adott bemeneti lábra rákötjük az adott komponens adott kimeneti lábát).
\end{itemize}
\end{itemize}

\subsection{FlipFlop}
Absztrakt osztály.
\begin{itemize}
\item Felelősség\\
Flipflopok ősosztálya, itt vannak leírva a flipflopok közös logikája.
\item Ősosztályok:\ AbstractComponent
\item Interfészek: (nincs)
\item Attribútumok $\ $
\begin{itemize}
	\item \texttt{protected Value q}: A flip-flopban tárolt érték (az előző állapot)
	\item \texttt{protected Value clk}: A rendszer előző stabil állapotánál mért flip-flop órajel-bemenetére érkező érték (ennek segítségével tudunk detektálni élváltozást).
\end{itemize}
\item Metódusok$\ $
\begin{itemize}
	\item \texttt{addTo(Circuit c)}: Meghívja az áramkör \texttt{add(FlipFlop ff)} metódusát.
	\item \texttt{void commit()}: Az FF jelenlegi kimenetét és az órajel bemenetét elmentjük a \texttt{q} és \texttt{clk} attribútumba. Ezt akkor kell meghívni, amikor az áramkör az adott áramköri bemenetekre stabil állapotba ért.
	\item \texttt{boolean isActive()}: Számolhat-e az FF? Ezt kell ellenőrizniük a konkrét flipflop implementációknak, hiszen ekkor kellhet a belső állapottól eltérő állapotot kiadni.
\end{itemize}
\end{itemize}

\subsection{DisplayComponent}
Absztrakt osztály.
\begin{itemize}
\item Felelősség\\
Megjelenítő típusú komponensek ősosztálya.
\item Ősosztályok:\ AbstractComponent.
\item Interfészek: (nincs).
\item Attribútumok $\ $
\begin{itemize}
	\item (nincs)
\end{itemize}
\item Metódusok$\ $
\begin{itemize}
\item \texttt{addTo(Circuit c)}: Meghívja az áramkör \texttt{add(DisplayComponent dc)} metódusát (és az ősosztály implementációját is -- hiszen ugyanúgy kell regisztrálni a megjelenítőket is, mint a többi komponenst).
\end{itemize}
\end{itemize}

\subsection{SourceComponent}
Absztrakt osztály.
\begin{itemize}
\item Felelősség\\
Jelforrás típusú komponensek ősosztálya.
\item Ősosztályok:\ AbstractComponent.
\item Interfészek: (nincs).
\item Attribútumok $\ $
\begin{itemize}
	\item (nincs)
\end{itemize}
\item Metódusok$\ $
\begin{itemize}
\item \texttt{addTo(Circuit c)}: Meghívja az áramkör \texttt{add(SourceComponent dc)} metódusát (és az ősosztály implementációját is -- hiszen ugyanúgy kell regisztrálni a jelforrásokat is, mint a többi komponenst).
	\item \texttt{abstract Value[] getValues()}: Lekérhetjük a jelforrás értékeit. Ennek megvalósítása a konkrét implementációk feladata.
	\item \texttt{abstract setValues(Value[] values)}: Beállítjuk a jelforrás értékét. Ennek megvalósítása a konkrét implementációk feladata. (pl. kapcsoló csak 1 elemű tömböt kaphat)
\end{itemize}
\end{itemize}

