\subsection{AndGate}
\begin{itemize}
\item Felelősség\\
ÉS kapu, az áramkör egyik alapeleme. Bemeneteire kötött komponensek  kiértékelését kezdeményezi, s a kapott értékek logikai ÉS kapcsolatát  valósítja meg, amit a kimenetén kiad.
\item Ősosztályok:\ Object $\rightarrow{}$ AbstractComponent $\rightarrow{}$ AndGate.
\item Interfészek: (nincs)
\item Attribútumok $\ $
\begin{itemize}
\item (nincs)
\end{itemize}
\item Metódusok$\ $
\begin{itemize}
\item (nincs)
\end{itemize}
\end{itemize}

\subsection{FlipFlopD}
\begin{itemize}
\item Felelősség\\
D flipflop, mely felfutó órajelnél beírja a belső memóriába az adatbemeneten (D)  lévő értéket.
\item Ősosztályok:\ Object $\rightarrow{}$ AbstractComponent $\rightarrow{}$ FlipFlop $\rightarrow{}$ FlipFlopD.
\item Interfészek: (nincs)
\item Attribútumok $\ $
\begin{itemize}
\item (nincs)
\end{itemize}
\item Metódusok$\ $
\begin{itemize}
\item (nincs)
\end{itemize}
\end{itemize}

\subsection{FlipFlopJK}
\begin{itemize}
\item Felelősség\\
JK flipflop, mely a belső memóriáját a Követelmények résznél leírt módon  a J és K bemenetektől függően változtatja.
\item Ősosztályok:\ Object $\rightarrow{}$ AbstractComponent $\rightarrow{}$ FlipFlop $\rightarrow{}$ FlipFlopJK.
\item Interfészek: (nincs)
\item Attribútumok $\ $
\begin{itemize}
\item (nincs)
\end{itemize}
\item Metódusok$\ $
\begin{itemize}
\item (nincs)
\end{itemize}
\end{itemize}

\subsection{Gnd}
\begin{itemize}
\item Felelősség\\
A "föld" komponens, mely állandóan a hamis értéket adja ki. Nincs bemenete.
\item Ősosztályok:\ Object $\rightarrow{}$ AbstractComponent $\rightarrow{}$ Gnd.
\item Interfészek: (nincs)
\item Attribútumok $\ $
\begin{itemize}
\item (nincs)
\end{itemize}
\item Metódusok$\ $
\begin{itemize}
\item (nincs)
\end{itemize}
\end{itemize}

\subsection{Inverter}
\begin{itemize}
\item Felelősség\\
Inverter alkatrész, mely invertálva adja ki a kimenetén a bemenetén  érkező jelet.
\item Ősosztályok:\ Object $\rightarrow{}$ AbstractComponent $\rightarrow{}$ Inverter.
\item Interfészek: (nincs)
\item Attribútumok $\ $
\begin{itemize}
\item (nincs)
\end{itemize}
\item Metódusok$\ $
\begin{itemize}
\item (nincs)
\end{itemize}
\end{itemize}

\subsection{Led}
\begin{itemize}
\item Felelősség\\
Egy LED-et reprezentál, mely világít, ha bemenetén igaz érték van.  3 féle színe lehet, ezeket a Color enumeráció határozza meg.
\item Ősosztályok:\ Object $\rightarrow{}$ AbstractComponent $\rightarrow{}$ Led.
\item Interfészek: IsDisplay.
\item Attribútumok $\ $
\begin{itemize}
\item (nincs)
\end{itemize}
\item Metódusok$\ $
\begin{itemize}
\item (nincs)
\end{itemize}
\end{itemize}

\subsection{Mpx}
\begin{itemize}
\item Felelősség\\
4-1-es multiplexer, melynek a bemeneti lábak sorrendje a következő:  D0, D1, D2, D3, S0, S1. Ahol Dx az adatbemenetek, Sy a kiválasztóbemenetek.  Kimenetén a kiválasztóbemenetektől függően valamelyik adatbemenet kerül kiadásra.
\item Ősosztályok:\ Object $\rightarrow{}$ AbstractComponent $\rightarrow{}$ Mpx.
\item Interfészek: (nincs)
\item Attribútumok $\ $
\begin{itemize}
\item (nincs)
\end{itemize}
\item Metódusok$\ $
\begin{itemize}
\item (nincs)
\end{itemize}
\end{itemize}

\subsection{OrGate}
\begin{itemize}
\item Felelősség\\
VAGY kapu, az áramkör egyik alapeleme. Bemeneteire kötött komponensek  kiértékelését kezdeményezi, s a kapott értékek logikai VAGY kapcsolatát  valósítja meg, amit a kimenetén kiad.
\item Ősosztályok:\ Object $\rightarrow{}$ AbstractComponent $\rightarrow{}$ OrGate.
\item Interfészek: (nincs)
\item Attribútumok $\ $
\begin{itemize}
\item (nincs)
\end{itemize}
\item Metódusok$\ $
\begin{itemize}
\item (nincs)
\end{itemize}
\end{itemize}

\subsection{SequenceGenerator}
\begin{itemize}
\item Felelősség\\
Jelgenerátort reprezentál, amely a beállított bitsorozatot adja ki. A  SequenceGeneratorStepper feladata, hogy a step() metódust meghívja ezen osztály  példányain. Azokat a FF-eket vezérli, melyek CLK bemenetére ez a komponens van kötve,  vagyis ha éppen felfutó él jön, akkor ezeket engedélyezi különben nem.
\item Ősosztályok:\ Object $\rightarrow{}$ AbstractComponent $\rightarrow{}$ SequenceGenerator.
\item Interfészek: IsSource.
\item Attribútumok $\ $
\begin{itemize}
	\item \texttt{private List ffList}: Azon FF-ek listája, melyekre ez a jelgenerátor van bekötve a CLK bemenetre.
	\item \texttt{private int index}: Bitsorozat egy indexe, ez határozza meg, hogy éppen melyik értéket adja ki.
	\item \texttt{private Value[] sequence}: Tárolt bitsorozat
\end{itemize}
\item Metódusok$\ $
\begin{itemize}
	\item \texttt{void addFlipFlop(FlipFlop ff)}: A flipflop-ot feliratkoztatjuk a jelgenerátorhoz, így ha felfutó él lesz,  akkor tudunk neki jelezni.
	\item \texttt{Value[] getValues()}: Jelgenerátor bitsorozatának lekérdezése
	\item \texttt{void setValues(Value[] values)}: Jelgenerátor bitsorozatának beállítása
	\item \texttt{void step()}: A jelgenerátor lép, a bitsorozat következő elemére ugrik. A következő léptetésig  ez kerül kiadásra a kimeneteken.
\end{itemize}
\end{itemize}

\subsection{SevenSegmentDisplay}
\begin{itemize}
\item Felelősség\\
7-szegmenses kijelzőt reprezentál, melynek 7 bemenete vezérli a  megfelelő szegmenseket, ezek világítanak, ha az adott bemenetre logikai  igaz van kötve.
\item Ősosztályok:\ Object $\rightarrow{}$ AbstractComponent $\rightarrow{}$ SevenSegmentDisplay.
\item Interfészek: IsDisplay.
\item Attribútumok $\ $
\begin{itemize}
\item (nincs)
\end{itemize}
\item Metódusok$\ $
\begin{itemize}
\item (nincs)
\end{itemize}
\end{itemize}

\subsection{Toggle}
\begin{itemize}
\item Felelősség\\
Kapcsoló jelforrás, melyet a felhasználó szimuláció közben kapcsolgathat.
\item Ősosztályok:\ Object $\rightarrow{}$ AbstractComponent $\rightarrow{}$ Toggle.
\item Interfészek: IsSource.
\item Attribútumok $\ $
\begin{itemize}
\item (nincs)
\end{itemize}
\item Metódusok$\ $
\begin{itemize}
	\item \texttt{Value[] getValues()}: Lekérjük a kapcsoló értékét (1 elemű tömb)
	\item \texttt{void setValues(Value[] values)}: Kapcsoló állapotának változtatása, csak 1 elemű tömböt kaphat paraméterül.
\end{itemize}
\end{itemize}

\subsection{Vcc}
\begin{itemize}
\item Felelősség\\
A tápfeszültés komponens, ami konstans igaz értéket ad. Nincs bemenete.
\item Ősosztályok:\ Object $\rightarrow{}$ AbstractComponent $\rightarrow{}$ Vcc.
\item Interfészek: (nincs)
\item Attribútumok $\ $
\begin{itemize}
\item (nincs)
\end{itemize}
\item Metódusok$\ $
\begin{itemize}
\item (nincs)
\end{itemize}
\end{itemize}

