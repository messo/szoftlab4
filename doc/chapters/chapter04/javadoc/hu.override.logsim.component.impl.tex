\subsection{AndGate}
\begin{itemize}
\item Felelősség\\
ÉS kapu, az áramkör egyik alapeleme. Bemeneteire kötött értékeken a logikai ÉS műveletet hajtva végre, és ennek eredményét adja ki a kimenetén.
\item Ősosztályok:\ AbstractComponent
\item Interfészek: (nincs)
\item Attribútumok $\ $
\begin{itemize}
\item (nincs)
\end{itemize}
\item Metódusok$\ $
\begin{itemize}
\item (nincs)
\end{itemize}
\end{itemize}

\subsection{FlipFlopD}
\begin{itemize}
\item Felelősség\\
D flipflop, mely felfutó órajelnél beírja a belső állapotába az adatbemeneten lévő értéket. Kimenetén a belső állapota jelenik meg.
\item Ősosztályok:\ AbstractComponent $\rightarrow{}$ FlipFlop.
\item Interfészek: (nincs)
\item Attribútumok $\ $
\begin{itemize}
\item (nincs)
\end{itemize}
\item Metódusok$\ $
\begin{itemize}
\item (nincs)
\end{itemize}
\end{itemize}

\subsection{FlipFlopJK}
\begin{itemize}
\item Felelősség\\
JK flipflop, mely felfutó órajelnél a Követelmények résznél leírt módon a J és K bemenetén lévő értéktől függően változtatja a belső állapotát. Kimenetén a belső állapota jelenik meg.
\item Ősosztályok:\ AbstractComponent $\rightarrow{}$ FlipFlop.
\item Interfészek: (nincs)
\item Attribútumok $\ $
\begin{itemize}
\item (nincs)
\end{itemize}
\item Metódusok$\ $
\begin{itemize}
\item (nincs)
\end{itemize}
\end{itemize}

\subsection{Gnd}
\begin{itemize}
\item Felelősség\\
A "föld" komponens, mely állandóan a hamis értéket adja ki. Nincs bemenete.
\item Ősosztályok:\ AbstractComponent.
\item Interfészek: (nincs)
\item Attribútumok $\ $
\begin{itemize}
\item (nincs)
\end{itemize}
\item Metódusok$\ $
\begin{itemize}
\item (nincs)
\end{itemize}
\end{itemize}

\subsection{Inverter}
\begin{itemize}
\item Felelősség\\
Inverter alkatrész, mely invertálva adja ki a kimenetén a bemenetén érkező jelet.
\item Ősosztályok:\ AbstractComponent.
\item Interfészek: (nincs)
\item Attribútumok $\ $
\begin{itemize}
\item (nincs)
\end{itemize}
\item Metódusok$\ $
\begin{itemize}
\item (nincs)
\end{itemize}
\end{itemize}

\subsection{Led}
\begin{itemize}
\item Felelősség\\
Egy LED-et reprezentál, mely világít, ha bemenetén igaz érték van.
\item Ősosztályok:\ AbstractComponent $\rightarrow{}$ DisplayComponent.
\item Interfészek: (nincs)
\item Attribútumok $\ $
\begin{itemize}
\item (nincs)
\end{itemize}
\item Metódusok$\ $
\begin{itemize}
\item (nincs)
\end{itemize}
\end{itemize}

\subsection{Mpx}
\begin{itemize}
\item Felelősség\\
4-1-es multiplexer, amely 4 adatbemenettel, 2 kiválasztó-bemenettel és 1 kimenettel rendelkezik. A kiválasztó-bemenetekre adott értéktől függ, hogy melyik adatbemenet értéke jelenik meg az adatkimeneten.
\item Ősosztályok:\ AbstractComponent.
\item Interfészek: (nincs)
\item Attribútumok $\ $
\begin{itemize}
\item (nincs)
\end{itemize}
\item Metódusok$\ $
\begin{itemize}
\item (nincs)
\end{itemize}
\end{itemize}

\subsection{OrGate}
\begin{itemize}
\item Felelősség\\
VAGY kapu, az áramkör egyik alapeleme. Bemeneteire kötött értékeken a logikai VAGY műveletet hajtva végre, és ennek eredményét adja ki a kimenetén.
\item Ősosztályok:\ AbstractComponent.
\item Interfészek: (nincs)
\item Attribútumok $\ $
\begin{itemize}
\item (nincs)
\end{itemize}
\item Metódusok$\ $
\begin{itemize}
\item (nincs)
\end{itemize}
\end{itemize}

\subsection{SequenceGenerator}
\begin{itemize}
\item Felelősség\\
Jelgenerátort reprezentál, amely a beállított bitsorozatot adja ki.
\item Ősosztályok:\ AbstractComponent $\rightarrow{}$ SourceComponent.
\item Interfészek: (nincs)
\item Attribútumok $\ $
\begin{itemize}
	\item \texttt{private int index}: A bitsorozat egy indexe, ez határozza meg, hogy éppen melyik értéket adja ki a kimenetén.
	\item \texttt{private Value[] sequence}: Tárolt bitsorozat
\end{itemize}
\item Metódusok$\ $
\begin{itemize}
\item \texttt{addTo(Circuit c)}: Meghívja az áramkör \texttt{add(SequenceGenerator sg)} metódusát (és az ősosztály implementációját is -- hiszen ugyanúgy kell regisztrálni a jelforrásokat is, mint a többi komponenst).
	\item \texttt{Value[] getValues()}: Jelgenerátor bitsorozatának lekérdezése
	\item \texttt{void setValues(Value[] values)}: Jelgenerátor bitsorozatának beállítása
	\item \texttt{void step()}: A jelgenerátor lép, a bitsorozat következő elemére ugrik. A következő léptetésig ez kerül kiadásra a kimeneteken.
\end{itemize}
\end{itemize}

\subsection{SevenSegmentDisplay}
\begin{itemize}
\item Felelősség\\
7-szegmenses kijelzőt reprezentál, melynek 7 bemenete vezérli a  megfelelő szegmenseket, ezek világítanak, ha az adott bemenetre logikai igaz van kötve.
\item Ősosztályok:\ AbstractComponent $\rightarrow{}$ DisplayComponent.
\item Interfészek: (nincs)
\item Attribútumok $\ $
\begin{itemize}
\item (nincs)
\end{itemize}
\item Metódusok$\ $
\begin{itemize}
\item (nincs)
\end{itemize}
\end{itemize}

\subsection{Toggle}
\begin{itemize}
\item Felelősség\\
Kapcsoló jelforrás, melynek két állapota lehet; egyikben logikai igazat, másikban logikai hamist ad ki.
\item Ősosztályok:\ AbstractComponent $\rightarrow{}$ SourceComponent.
\item Interfészek: (nincs)
\item Attribútumok $\ $
\begin{itemize}
\item (nincs)
\end{itemize}
\item Metódusok$\ $
\begin{itemize}
	\item \texttt{Value[] getValues()}: Lekérjük a kapcsoló értékét (1 elemű tömb)
	\item \texttt{void setValues(Value[] values)}: Kapcsoló állapotának változtatása, csak 1 elemű tömböt kaphat paraméterül.
\end{itemize}
\end{itemize}

\subsection{Vcc}
\begin{itemize}
\item Felelősség\\
A tápfeszültés komponens, ami konstans igaz értéket ad. Nincs bemenete.
\item Ősosztályok:\ AbstractComponent.
\item Interfészek: (nincs)
\item Attribútumok $\ $
\begin{itemize}
\item (nincs)
\end{itemize}
\item Metódusok$\ $
\begin{itemize}
\item (nincs)
\end{itemize}
\end{itemize}

