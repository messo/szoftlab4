% Szglab4
% ===========================================================================
%
\section{Napló}

\begin{naplo}

\bejegyzes
{2011.02.15.~21:30~}
{2 óra}
{Kriván B.\newline
Dévényi A.\newline
Jákli G.}
{Értekezlet.\newline
Döntések: Megegyeztünk a feladat értelme\-zését illetően.
Milyen kérdéseket teszünk fel a konzulensnek az első konzultáción?\newline
Ezeket a Apagyi G. és Péter T. számára továbbítottuk.}

\bejegyzes
{2011.02.16.~09:00~}
{2 óra}
{Kriván B.\newline
Dévényi A.\newline
Apagyi G.\newline
Péter T.}
{Értekezlet.\newline
Döntések:
\begin{itemize}
\setlength{\itemsep}{0cm}%
\setlength{\parskip}{0cm}%
\item Fejlesztői környezetben megállapodtunk (\ref{sec:devenvironment})
\item Projekt szervezési struktúráját rögzítettük (\ref{sec:orgstructure})
\item A felmerülő algoritmusokról is konzultáltunk.
\end{itemize}
}

\bejegyzes
{2011.02.16.~15:00~}
{1 óra}
{Jákli G.}
{Tevékenység: Projekt terv bővítése, formázá\-sa, javítása (\ref{sec:projectplan})}

\bejegyzes
{2011.02.17.~16:00~}
{1 óra}
{Jákli G.\newline
Kriván B.\newline
Dévényi A.}
{Tevékenység: \aref{sec:reqdef} és \aref{sec:projectplan} alfejezet közös átdolgozása}

\bejegyzes
{2011.02.17.~19:15~}
{45 perc}
{Jákli G.\newline
Kriván B.\newline
Dévényi A.}
{Értekezlet.\newline
Döntések: \Aref{sec:taskdesc} alfejezet főbb gondolatait megfogalmaztuk, és meghatároztuk, hogy mik legyenek a mindenképpen lejegyezendő dolgok.}

\bejegyzes
{2011.02.17.~20:00~}
{50 perc}
{Jákli G.}
{Tevékenység: A megbeszéltek alapján el\-kezdte \aref{sec:taskdesc} alfejezet megírását.}

\bejegyzes
{2011.02.17.~23:00~}
{30 perc}
{Péter T.}
{Tevékenység: Szervezési struktúra kiegészí\-tése képernyőképekkel}

\bejegyzes
{2011.02.18.~00:00~}
{30 perc}
{Kriván B.\newline
Dévényi A.\newline
Péter T}
{Értekezlet (Msn megbeszélés).\newline
Döntések: \Aref{sec:taskdesc} alfejezet módosításának elhatározása és a szótárba (\ref{sec:dictionary}) bekerülő szavak rögzítése}

\bejegyzes
{2011.02.18.~00:30~}
{30 perc}
{Péter T.}
{Az előző értekezleten meghatározott szavak felvétele a szótárba, még csak felsorolás szint\-jén}

\bejegyzes
{2011.02.18.~14:00~}
{1 óra}
{Péter T.}
{Szótárban lévő szavak magyarázatainak kitöl\-tése}

\bejegyzes
{2011.02.18.~15:30~}
{30 perc}
{Apagyi G.}
{Szótárban lévő szavak megmagyarázásának folytatása}

\bejegyzes
{2011.02.19.~12:00~}
{2 óra}
{Kriván B.}
{Tevékenység: helyesírási hibák javítása, szó\-tár (\ref{sec:dictionary}) szerkesztése (sorrendek változtatása, további szavak bevezetése, meglévők magya\-rázatainak szerkesztése)}

\bejegyzes
{2011.02.19.~19:30~}
{30 perc}
{Kriván B.\newline
Jákli G.}
{Értekezlet (Msn megbeszélés).\newline
Döntések: Új essential use-caset kell rajzolni. Szükség van Gnd és Vcc komponensre.}

\bejegyzes
{2011.02.19.~20:00~}
{30 perc}
{Jákli G.}
{Új essential use-case megrajzolása (\ref{sec:usecasediagram}), use-case leírások megírása (\ref{sec:usecasedesc})}

\bejegyzes
{2011.02.19.~20:00~}
{30 perc}
{Kriván B.}
{Gnd és Vcc komponens felvétele (\ref{sec:taskdesc}), megfelelő részek szerkesztése, use-case konvertálása \LaTeX{}-kompatibilis formátumba és felvétele a dokumentációba}

\bejegyzes
{2011.02.19.~22:00~}
{30 perc}
{Dévényi A.}
{A dokumentáció figyelmes átolvasása, az összes formai és nyelvtani hiba kijavítása}

\bejegyzes
{2011.02.20.~18:30~}
{20 perc}
{Kriván B.\newline
Dévényi A.\newline
Jákli G.}
{Értekezlet (Msn megbeszélés)\newline
Döntések: Szükség van új komponensekre a jobb használhatóság érdekében: 7-szegmenses kijelző, 4-1-es multiplexer, D és JK flip-flop}

\bejegyzes
{2011.02.20.~18:50~}
{30 perc}
{Jákli G.}
{Tevékenység: komponensek felvétele a feladat leírásba (\ref{sec:taskdesc})}

\end{naplo}

