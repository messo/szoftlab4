% Szglab4
% ===========================================================================
%
\chapter{Szkeleton beadás}

\thispagestyle{fancy}

\section{Fordítási és futtatási útmutató}

\subsection{Fájllista}

\begin{fajllista}

\fajl
{compile.bat} % Kezdet
{178 byte} % Idptartam
{2011.03.19.~11:11~} % Résztvevők
{Fordításra használt batch fájl} % Leírás
\fajl
{doc.bat} % Kezdet
{276 byte} % Idptartam
{2011.03.19.~11:11~} % Résztvevők
{Dokumentáció generálására készített batch fájl} % Leírás
\fajl
{run.bat} % Kezdet
{106 byte} % Idptartam
{2011.03.19.~11:11~} % Résztvevők
{Futtatáshoz használt batch fájl} % Leírás

\fajl
{src/logsim/Skeleton.java} % Kezdet
{6246 byte} % Idptartam
{2011.03.13.~14:34~} % Résztvevők
{Menüt tartalmazza. A felhasználó választása alapján a megfelelő tesztesetet elindítja} % Leírás

\fajl
{src/logsim/log/Loggable.java} % Kezdet
{362 byte} % Idptartam
{2011.03.13.~14:53~} % Résztvevők
{Loggolást segítő interfész} % Leírás

\fajl
{src/logsim/log/\newline
LoggableInt.java} % Kezdet
{604 byte} % Idptartam
{2011.03.13.~15:49~} % Résztvevők
{Wrapper osztály az int típus köré, mely lehetővé teszi az intek loggolását} % Leírás

\fajl
{src/logsim/log/Logger.java} % Kezdet
{5062 byte} % Idptartam
{2011.03.13.~14:38~} % Résztvevők
{Loggolást megvalósító osztály} % Leírás

\fajl
{src/logsim/model/Circuit.java} % Kezdet
{4405 byte} % Idptartam
{2011.03.13.~14:36~} % Résztvevők
{Az áramkört megvalósító osztály} % Leírás

\fajl
{src/logsim/model/\newline
Simulation.java} % Kezdet
{1684 byte} % Idptartam
{2011.03.13.~14:36~} % Résztvevők
{A szimulációt megvalósító osztály} % Leírás

\fajl
{src/logsim/model/Value.java} % Kezdet
{721 byte} % Idptartam
{2011.03.13.~14:36~} % Résztvevők
{Az áramkörben előforduló értkékeket tartalmazó osztály} % Leírás

\fajl
{src/logsim/model/component/\newline
AbstractComponent.java} % Kezdet
{3379 byte} % Idptartam
{2011.03.13.~14:46~} % Résztvevők
{Az alkatrészek absztrakt ősosztálya} % Leírás

\fajl
{src/logsim/model/component/\newline
DisplayComponent.java} % Kezdet
{767 byte} % Idptartam
{2011.03.13.~14:46~} % Résztvevők
{Megjelenítő típusú alkatrészek absztrakt ősosztálya} % Leírás

\fajl
{src/logsim/model/component/\newline
SourceComponent.java} % Kezdet
{1119 byte} % Idptartam
{2011.03.13.~14:46~} % Résztvevők
{Forrás típusú alkatrészek absztrakt ősosztálya} % Leírás

\fajl
{src/logsim/model/component/\newline
Wire.java} % Kezdet
{1295 byte} % Idptartam
{2011.03.13.~14:46~} % Résztvevők
{Vezetéket megvalósító osztály} % Leírás

\fajl
{src/logsim/model/component/\newline
impl/Inverter.java} % Kezdet
{946 byte} % Idptartam
{2011.03.13.~14:46~} % Résztvevők
{Az inverter alkatrészt megvalósító osztály} % Leírás

\fajl
{src/logsim/model/component/\newline
impl/Led.java} % Kezdet
{833 byte} % Idptartam
{2011.03.13.~14:46~} % Résztvevők
{A led megjelenítőt megvalósító osztály} % Leírás

\fajl
{src/logsim/model/component/\newline
impl/Node.java} % Kezdet
{1110 byte} % Idptartam
{2011.03.13.~14:46~} % Résztvevők
{Csomópont alkatrészt megvalósító osztály} % Leírás

\fajl
{src/logsim/model/component/\newline
impl/OrGate.java} % Kezdet
{1109 byte} % Idptartam
{2011.03.13.~14:46~} % Résztvevők
{VAGY kaput megvalósító osztály} % Leírás

\fajl
{src/logsim/model/component/\newline
impl/Toggle.java} % Kezdet
{1686 byte} % Idptartam
{2011.03.13.~14:46~} % Résztvevők
{A kapcsolót megvalósító osztály} % Leírás

\fajl
{src/logsim/model/skeleton/\newline
Simulation1.java} % Kezdet
{1712 byte} % Idptartam
{2011.03.19.~13:56~} % Résztvevők
{Az első tesztesetet tartalmazó osztály} % Leírás

\fajl
{src/logsim/model/skeleton/\newline
Simulation2.java} % Kezdet
{2147 byte} % Idptartam
{2011.03.19.~14:16~} % Résztvevők
{A második tesztesetet tartalmazó osztály} % Leírás

\fajl
{src/logsim/model/skeleton/\newline
Simulation3.java} % Kezdet
{2622 byte} % Idptartam
{2011.03.19.~14:17~} % Résztvevők
{A harmadik tesztesetet tartalmazó osztály} % Leírás

\fajl
{src/logsim/model/skeleton/\newline
Simulation4.java} % Kezdet
{2326 byte} % Idptartam
{2011.03.19.~14:17~} % Résztvevők
{A negyedik tesztesetet tartalmazó osztály} % Leírás

\fajl
{src/logsim/model/skeleton/\newline
Simulation5.java} % Kezdet
{2924 byte} % Idptartam
{2011.03.19.~14:18~} % Résztvevők
{Az ötödik tesztesetet tartalmazó osztály} % Leírás

\end{fajllista}

\subsection{Fordítás}
A hibamentes és minél inkább gördülékenyebb fordítás érdekében létrehoztunk egy \texttt{compile.bat} nevezetű batch fájlt, mely a projekt főkönyvtárában található. Projekt főkönytára az, amelyik a batch fájlokat és a "src" nevezetű mappát tartalmazza, melyben a program forráskódja található. Szükség estén kézzel kell módosítani a batch fájl
\begin{verbatim}
set C="C:\Program Files\Java\jdk1.6.0_24\bin\" 
\end{verbatim}
sorát, attól függően, hogy a gépen éppen melyik Java JDK verzió található és az hová van telepítve!\\

A \texttt{compile.bat} fájl az alábbi parancsokat hajtja végre:
\lstinputlisting[escapeinside=`', xleftmargin=10pt, frame=single, basicstyle=\ttfamily\footnotesize, language=sh]{../LogSimSkeleton/compile.bat}
Ha hibamentes volt a fordítás, azt a "Fordítás sikeres" kimenettel értesíti a felhasználót.\\

A fordítás sikeressége után, lehetőség van a dokumentáció legenerálására is. Ehhez felhasználható a főkönyvtárban található \texttt{doc.bat} batch fájl.
Szükség estén kézzel kell módosítani a batch file \begin{verbatim}
set C="C:\Program Files\Java\jdk1.6.0_24\bin\" 
\end{verbatim}
sorát, attól függően, hogy a gépen éppen melyik Java JDK verzió található és az hová van telepítve!\\

A batch fájl az alábbi parancsokat hajtja végre:
\lstinputlisting[escapeinside=`', xleftmargin=10pt, frame=single, basicstyle=\ttfamily\footnotesize, language=sh]{../LogSimSkeleton/doc.bat}
Ha a dokumentum generálás sikeres volt, akkor a documents nevezetű mappában megtaláhatóak a kívánt dokumentumok.

\subsection{Futtatás}
A futtatás megkönnyításe érdekében elkészítettük a \texttt{run.bat} batch fájlt.
Szükség estén kézzel kell módosítani a batch file 
\begin{verbatim}
set C="C:\Program Files\Java\jdk1.6.0_24\bin\" 
\end{verbatim} sorát, attól függően, hogy a gépen éppen melyik Java JDK verzió található és az hová van telepítve!\\

A \texttt{run.bat} fájl az alábbi parancsokat hajtja végre:
\lstinputlisting[escapeinside=`', xleftmargin=10pt, frame=single, basicstyle=\ttfamily\footnotesize, language=sh]{../LogSimSkeleton/run.bat}
A "build" könyvtárból elindítja az előzőleg lefordított programot.

\section{Értékelés}

Csapatunk már a múlt félévben kialakult, így nem voltunk ismeretlenek egymás számára, ismertük egymás képességeit. Az elején hamar egyeszségre jutottunk, hogy miként fog történni a kommunikáció a csapaton belül és milyen módon tároljuk az elkészült anyagokat, úgy, hogy mindig mindenki a legfrissebb fájlokhoz férhessen hozzá. 

A közös munka elején mégis gondot okozott számunkra a csapatban dolgozás. Ezt azonban hamar felismertük és közös megbeszéléseket tartva pontosan definiáltuk mindenkinek az aktuális munkáját és a továbbiakban figyelemmel kisértük a másik munkájának a haladását is. Előfordult, hogy a munka haladtával merültek fel kérdések, mely más csapattagok átlal készített megoldásokat is érintette. Ilyenkor gyors megbeszélés és döntés után a csapat probléma nélkül tudta folytatni a közös munkát. Minden döntésünkről és annak okairól leírás készül, melyet elolvasva azok a csapattagok is követni tudták a fejleményeket, akik esetleg nem tudtak résztvenni egy közös gyűlésen.

A munkaórák eloszlását közös megegyezés alapján korrigáltuk és ebből számítottuk a végső százalékokat, melyek az alábbiak szerint alakultak:

\begin{ertekeles}
\tag{Apagyi G.} % Tag neve
{10}        % Munka szazalekban
\tag{Dévényi A.}
{25}
\tag{Jákli G.}
{25}
\tag{Kriván B.}
{30}
\tag{Péter T.}
{10}
\end{ertekeles}
