% Szglab4
% ===========================================================================
%
\chapter{Szkeleton beadás}

\thispagestyle{fancy}

\section{Fordítási és futtatási útmutató}
\comment{A feltöltött program fordításával és futtatásával kapcsolatos útmutatás. Ennek tartalmaznia kell leltárszerűen az egyes fájlok pontos nevét, méretét byte-ban, keletkezési idejét, valamint azt, hogy a fájlban mi került megvalósításra.}

\subsection{Fájllista}

\begin{fajllista}

\fajl
{src/logsim/Skeleton.java} % Kezdet
{250 byte} % Idptartam
{2009.10.10~18:05~} % Résztvevők
{Menüt tartalmazza. A felhasználó választása alapján a megfelelő tesztesetet elindítja} % Leírás

\fajl
{src/logsim/log/Loggable.java} % Kezdet
{250 byte} % Idptartam
{2009.10.10~18:05~} % Résztvevők
{Loggolást segítő interfész} % Leírás

\fajl
{src/logsim/log/\newline
LoggableInt.java} % Kezdet
{250 byte} % Idptartam
{2009.10.10~18:05~} % Résztvevők
{Wrapper osztály az int típus köré, mely lehetővé teszi az intek loggolását} % Leírás

\fajl
{src/logsim/log/Logger.java} % Kezdet
{250 byte} % Idptartam
{2009.10.10~18:05~} % Résztvevők
{Loggolást megvalósító osztály} % Leírás

\fajl
{src/logsim/model/Circuit.java} % Kezdet
{250 byte} % Idptartam
{2009.10.10~18:05~} % Résztvevők
{Az áramkört megvalósító osztály} % Leírás

\fajl
{src/logsim/model/\newline
Simulation.java} % Kezdet
{250 byte} % Idptartam
{2009.10.10~18:05~} % Résztvevők
{A szimulációt megvalósító osztály} % Leírás

\fajl
{src/logsim/model/Value.java} % Kezdet
{250 byte} % Idptartam
{2009.10.10~18:05~} % Résztvevők
{Az áramkörben előforduló értkékeket tartalmazó osztály} % Leírás

\fajl
{src/logsim/model/component/\newline
AbstractComponent.java} % Kezdet
{250 byte} % Idptartam
{2009.10.10~18:05~} % Résztvevők
{Az alkatrészek absztrakt ősosztálya} % Leírás

\fajl
{src/logsim/model/component/\newline
DisplayComponent.java} % Kezdet
{250 byte} % Idptartam
{2009.10.10~18:05~} % Résztvevők
{Megjelenítő típusú alkatrészek absztrakt ősosztálya} % Leírás

\fajl
{src/logsim/model/component/\newline
SourceComponent.java} % Kezdet
{250 byte} % Idptartam
{2009.10.10~18:05~} % Résztvevők
{Forrás típusú alkatrészek absztrakt ősosztálya} % Leírás

\fajl
{src/logsim/model/component/\newline
Wire.java} % Kezdet
{250 byte} % Idptartam
{2009.10.10~18:05~} % Résztvevők
{Vezetéket megvalósító osztály} % Leírás

\fajl
{src/logsim/model/component/\newline
impl/Inverter.java} % Kezdet
{250 byte} % Idptartam
{2009.10.10~18:05~} % Résztvevők
{Az inverter alkatrészt megvalósító osztály} % Leírás

\fajl
{src/logsim/model/component/\newline
impl/Led.java} % Kezdet
{250 byte} % Idptartam
{2009.10.10~18:05~} % Résztvevők
{A led megjelenítőt megvalósító osztály} % Leírás

\fajl
{src/logsim/model/component/\newline
impl/Node.java} % Kezdet
{250 byte} % Idptartam
{2009.10.10~18:05~} % Résztvevők
{Csomópont alkatrészt megvalósító osztály} % Leírás

\fajl
{src/logsim/model/component/\newline
impl/OrGate.java} % Kezdet
{250 byte} % Idptartam
{2009.10.10~18:05~} % Résztvevők
{VAGY kaput megvalósító osztály} % Leírás

\fajl
{src/logsim/model/component/\newline
impl/Toggle.java} % Kezdet
{250 byte} % Idptartam
{2009.10.10~18:05~} % Résztvevők
{A kapcsoló forrást megvalósító osztály} % Leírás

\fajl
{src/logsim/model/skeleton/\newline
Circuit1.java} % Kezdet
{250 byte} % Idptartam
{2009.10.10~18:05~} % Résztvevők
{Az első tesztesetet tartalmazó osztály} % Leírás

\fajl
{src/logsim/model/skeleton/\newline
Circuit2.java} % Kezdet
{250 byte} % Idptartam
{2009.10.10~18:05~} % Résztvevők
{A második tesztesetet tartalmazó osztály} % Leírás

\fajl
{src/logsim/model/skeleton/\newline
Circuit3.java} % Kezdet
{250 byte} % Idptartam
{2009.10.10~18:05~} % Résztvevők
{A harmadik tesztesetet tartalmazó osztály} % Leírás

\fajl
{src/logsim/model/skeleton/\newline
Circuit4.java} % Kezdet
{250 byte} % Idptartam
{2009.10.10~18:05~} % Résztvevők
{A negyedik tesztesetet tartalmazó osztály} % Leírás

\fajl
{src/logsim/model/skeleton/\newline
Circuit5.java} % Kezdet
{250 byte} % Idptartam
{2009.10.10~18:05~} % Résztvevők
{Az ötödik tesztesetet tartalmazó osztály} % Leírás

\end{fajllista}

\subsection{Fordítás}
A hibamentes és minél inkább gördülékenyebb fordítás érdekében létrehoztunk egy compile.bat nevezetű batch fájlt, mely a projekt főkönyvtárában található. Projekt főkönytára az, amelyik batch fájlokat tartalmaz és a "src" nevezetű mappát, melyben a program forráskódja található.

A compile.bat fájl az alábbi parancsokat hajtja végre:
\lstset{escapeinside=`', xleftmargin=10pt, frame=single, basicstyle=\ttfamily\footnotesize, language=sh}
\begin{lstlisting}
@echo off
set PATH=%PATH%;C:\Program Files\Java\jdk1.6.0_21\bin\
cd src
javac logsim\Skeleton.java
cd..
if not errorlevel 1 echo Forditas sikeres
pause
\end{lstlisting}
Vagyis, beállítja a PATH környezeti változót, majd lefordítja a programot. Ha hibamentes volt a fordítás, azt a "Fordítás sikeres" kimenettel értesíti a felhasználót.
Fontos, hogy a PATH változóban az aktuális JDK bináris könyvtárának az elérési útvonala legyen. Szükség esetén módosítani kell a compile.bat batch fájlt!


A fordítás sikeressége után, lehetőség van a dokumentáció legenerálására is. Ehhez felhasználható a főkönyvtárban található doc.bat batch fájl, mely az alábbi parancsokat hajtja végre:
\lstset{escapeinside=`', xleftmargin=10pt, frame=single, basicstyle=\ttfamily\footnotesize, language=sh}
\begin{lstlisting}
@echo off
cd src
javadoc logsim logsim.log logsim.model logsim.model.component
 logsim.model.component.impl logsim.model.skeleton -d ..\documents
cd..
if not errorlevel 1 echo Dokumentum generalas sikeres volt. 
pause
\end{lstlisting}
Ha a dokumentum generálás sikeres volt, akkor a documents nevezetű mappában megtaláhatóak a kívánt dokumentumok.



\subsection{Futtatás}
A futtatás megkönnyításe érdekében elkészítettük a run.bat batch fájlt.
Ez az alábbi parancsokat hajtja végre:
\lstset{escapeinside=`', xleftmargin=10pt, frame=single, basicstyle=\ttfamily\footnotesize, language=sh}
\begin{lstlisting}
@echo off
cd src
java logsim.Skeleton
\end{lstlisting}
Az "src" könyvtárból elindítja az előzőleg lefordított programot.


\section{Értékelés}
\comment{A projekt kezdete óta az értékelésig eltelt időben tagokra bontva, százalékban.}

\begin{ertekeles}
\tag{Horváth} % Tag neve
{23.5}        % Munka szazalekban
\tag{Német}
{24.5}
\tag{Tóth}
{25}
\tag{Oláh}
{27}
\end{ertekeles}

