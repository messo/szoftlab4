% Szglab4
% ===========================================================================
%
\section{Napló}

\begin{naplo}

\bejegyzes
{2011.03.22.~12:00~} % Kezdet
{2,5 óra} % Időtartam
{Apagyi G.} % Résztvevők
{Prototípus áramkör leíró nyelvének definiálása} % Leírás

\bejegyzes
{2011.03.22.~14:00~}
{1,5 óra}
{Kriván B.\newline
Jákli G.\newline
Dévényi A.}
{Értekezlet: Specifikáció módosítása miatt szükségszerű változtatások megbeszélése}

\bejegyzes
{2011.03.22.~20:00~}
{1 óra}
{Jákli Gábor}
{Összes részletes use-case}

\bejegyzes
{2011.03.22.~22:00~}
{1 óra}
{Dévényi A.}
{Felhasználói parancsok}

\bejegyzes
{2011.03.23.~14:00~}
{1 óra}
{Dévényi A.}
{Konfigurációs fájl nyelvtana, kimeneti nyelv}

\bejegyzes
{2011.03.23.~15:00~}
{45 perc}
{Jákli G.}
{Új use case, 7.1.1 és 7.4}

\bejegyzes
{2011.03.26.~16:00~}
{1,5 óra}
{Péter T.}
{Tesztelési terv és tesztesetek}

\bejegyzes
{2011.03.28.~11:30~}
{2 óra}
{Kriván B.}
{Oszcilloszkóp és kompozit elem felvétele a megfelelő fejezetekbe, illetve az osztálydiagram javítása.}

\bejegyzes
{2011.03.28.~12:00~}
{30 perc}
{Dévényi A.}
{7.0-ás fejezet megírása.}

\bejegyzes
{2011.03.28.~12:30~}
{45 perc}
{Jákli G.}
{A dokumentum átnézése, formázás javítása és helyesírás ellenőrzés.}

\end{naplo}

