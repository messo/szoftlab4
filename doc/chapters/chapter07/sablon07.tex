% Szglab4
% ===========================================================================
%
\chapter{Prototípus koncepciója}

\parindent 0pt
\setcounter{secnumdepth}{3}
\setcounter{tocdepth}{3}
\thispagestyle{fancy}

\setcounter{section}{-1}
\section{Változtatások}

A specifikáció módosulása miatt a következő változtatásokat kellett tenni a modellben.
Bevezettük a Composite osztályt, mely egy kompozit áramköri elemet ír le. Mivel az áramkör is lényegében egy kompozit elem, ezért a Circuit osztály a Composite-ból öröklődik. Ha egy kompozit elemnek nincsen stacionárius állapota, akkor azt jelzi kifele (az őt magába foglaló kompozitnak - amennyiben az áramkörről van szó, akkor a szimulációt értesíti) és ezáltal az egész áramkörnek nem lesz stacionárius állapota. Az áramkör annyiból speciális kompozit, hogy nincsen be- és kimenete.

A másik új elem, mely bevezetésre került az oszcillátor. Ez egy olyan speciális LED, mely kijelzi az aktuális értékét, de lekérhető tőle az addig eltárolt értékek is. A memóriája véges, ha megtelik, az új érték a legrégebbi érték helyére íródik be.

\section{Prototípus interface-definíciója}

\subsection{Az interfész általános leírása}
A prototípus szabványos ki- és bemeneten kommunikál a felhasználóval. Az elkészített prototípus program egy
saját parancsrendszert használ. A parancs kiadása után a program végrehajtja azt és kiírja az eredményt a kimenetre. Az automatikus tesztelés elősegítése érdekében lehetőség van arra, hogy 
a parancsokat egy előre elkészített fájlból olvassa és a kimenetet fájlba mentse. A program az áramkört
fájlból olvassa. A tesztelés elősegítése érdekében elkészítünk néhány áramkört, azonban a felhasználó a megadott áramkört leíró fájl specifikációja alapján saját áramkört is készíthet, majd tesztelhet.

\subsection{Bemeneti nyelv}

\subsubsection{Felhasználói parancsok}

A parancsokat a standard bemenetről, illetve fájlból olvassa be a program. Minden parancsot egy sorvége karakter zár le.\newline
Megjegyzés: minden parancs ad visszajelzést a felhasználónak a végrehajtott eseményről, ennek formátuma \aref{sec:output} alfejezetben olvasható.\newline

\textit{loadCircuit <file>}
\begin{itemize}
	\item Leírás: A megadott áramkört betölti a szimulációs program. A szintaktikát lásd \ref{sec:circuit_def} fejezetet.
\end{itemize}

\textit{loadSettings <file>}
\begin{itemize}
	\item Leírás: A jelenlegi áramkörhöz a megadott konfigurációs fájl betöltése. A szintaktikát lásd \ref{sec:settings_def} fejezetet.
\end{itemize}

\textit{saveSettings <file>}
\begin{itemize}
	\item Leírás: A pillanatnyilag használt konfiguráció fájlba mentése.
\end{itemize}

\textit{switch <név>}
\begin{itemize}
	\item Leírás: A megnevezett kapcsoló átállítása.
\end{itemize}

\textit{setSeqGen <név> <bit1>[<bit2>...]}
\begin{itemize}
	\item Leírás: A megnevezett szekvenciagenerátor az értékparaméterek szerint beállítódik.
	\item Megjegyzés: A szekvencia legalább 1 bitből áll.
\end{itemize}

\textit{check <név> | -all}
\begin{itemize}
	\item Leírás: A megadott áramköri elem bemeneteinek és kimeneteinek kilistázása.
	\item Opció: a \texttt{check -all} parancs kilistázza az összes áramköri elem bemenetét és kimenetét.
\end{itemize}

\textit{step}
\begin{itemize}
	\item Leírás: A parancs hatására lefut egy szimulációs ciklus, melynek két eredménye lehet:
	\begin{itemize}
		\item véges lépésen belül stabilizálódik a rendszer, ekkor a kapcsoló(k), szekvenciagenerátor(ok) és kijelző(k) értéke(i) kiíródnak.
		\item nem stabilizálódik az áramkör; hibaüzenet
	\end{itemize}
\end{itemize}

\subsubsection{Az áramkör leíró fájl nyelvtana}
\label{sec:circuit_def}
Az áramkör leíró fájlban adjuk meg a szimulálandó hálózatot. Egyszerű szövegfájl, melyben az értelmezendő parancsok soronként tagolódnak. A program feltételezi, hogy ez egy, a Követelmények c. fejezetben megfogalmazott hálózatot ír le, azaz pl. nincsen lebegő ki- és/vagy bemenet.\\
A fájl egy sora az alábbi formátumú:
\begin{verbatim}
	<név> = <komponens> ( <paraméter1> [, <paraméter2>, ...] )
\end{verbatim}
Megjegyzés: a szóközök nem bírnak jelentéssel, ezek csak az olvashatóságot növelik.\\

Minden sor egy komponenst ír le, az egyenlőség baloldalán található a komponens neve, mellyel a továbbiak során az adott komponensre, illetve annak valamelyik kimenetére hivatkozhatunk. A jobb oldalán pedig az adott komponens típusát határozzuk meg, illetve a paramétereket, melyek főképpen bemenetek:
\begin{itemize}
	\item \texttt{0}: konstans 0.
	\item \texttt{1}: konstans 1.
	\item \texttt{<név>[i]}: egy adott komponens i. kimenete. Ha nem írjuk oda, hogy \texttt{[i]}, akkor az alapértelmezett az, hogy a komponens 1. kimenetét kötjük oda. A név csak az angol ábc betűit és a számokat tartalmazhatja, illetve csak betűvel kezdődhet.
	\item Egyéb paraméter típus, ami a komponens létrehozásához kell (ilyen pl. a csomópontnak a kimeneti száma, illetve az oszcilloszkóp által eltárolható értékek száma), mely általában egy egész szám.
\end{itemize}
Ettől jelentősen eltér a kompozit definiálása (ennek felhasználása az áramkörben viszont ilyen alakú), ezt lásd később.

A lehetséges komponens típusok (és azok bemeneteinek) az alábbiak:
\begin{itemize}
%			\item OR(...)			%+
%			\item AND(...)			%+			
%			\item INV(...)		%+			
%			\item VCC(...)			%+
%			\item GND(...)			%+
%			\item MPX(...)			%+
%			\item FFJK(...)			%
%			\item FFD(...)			%
%			\item LED(...)			%+
%			\item 7SEG(...)			%
%			\item TOGGLE(...)		%+
%			\item SEQGEN(...)			%+
%			\item NODE(...)			%+
			
\item \texttt{OR(in1, in2 [, in3, ...])}
	\begin{itemize}
	\item Leírás: VAGY kapu létrehozása.
	\item Paraméterek: 
		\begin{itemize}
			\item N bemenetű kapu esetén ide N db bemenet kerül. A bemenetek számát nem kell megadni, azt a feldolgozó automatikusan észleli a megkapott paraméterek számából. Minimum 2 bemenetet meg kell adni.
		\end{itemize}
	\item Példa: \texttt{or = OR(kapcs1, kapcs2, kapcs3)} - három komponens rákapcsolása a kapura, mely így egy három bemenetes vagy kapu lesz.
	\end{itemize}

\item \texttt{AND(in1, in2 [, in3, ...])}
	\begin{itemize}
	\item Leírás: ÉS kapu létrehozása.
	\item Paraméterek: 
		\begin{itemize}
			\item N bemenetű kapu esetén ide N db bemenet kerül. A bemenetek számát nem kell megadni, azt a feldolgozó automatikusan észleli a megkapott paraméterek számából. Minimum 2 bemenetet meg kell adni.
		\end{itemize}
	\item Példa: \texttt{and = AND(kapcs1, kapcs2)} - két komponens rákapcsolása a kapura, mely így egy két bemenetes vagy kapu lesz.
	\end{itemize}
	
\item \texttt{INV(in)}
	\begin{itemize}
	\item Leírás: inverter létrehozása.
	\item Paraméterek: 
		\begin{itemize}
			\item in: az inverter bemenete.
		\end{itemize}
	\item Példa: \texttt{inv = INV(kapcs1)} - egy kapcsoló rákapcsolása az inverterre.
	\end{itemize}
	
%\item \texttt{VCC()}
%	\begin{itemize}
%	\item Leírás: konstans igaz jel létrehozása.
%	\item Példa: \texttt{vcc = VCC()}
%	\end{itemize}
%	
%\item \texttt{GND()}
%	\begin{itemize}
%	\item Leírás: konstans hamis jel létrehozása.
%	\item Példa: \texttt{GND()}
%	\end{itemize}	

\item \texttt{MPX(in1, in2, in3, in4, s1, s2)}
	\begin{itemize}
	\item Leírás: 4:1 multiplexer létrehozása.
	\item Paraméterek: 
		\begin{itemize}
			\item in1..in4: a 4 adatbemenet.
			\item s1, s2: a 2 kiválasztó bemenet
		\end{itemize}
	\item Példa: \texttt{mpx = MPX(kapcs1, kapcs2, kapcs3, kapcs4, or1, and1)}
	\end{itemize}	
	
\item \texttt{FFJK(clk,J,K)}
	\begin{itemize}
	\item Leírás: J-K flip-flop létrehozása
	\item Paraméterek: 
		\begin{itemize}
			\item clk: órajel bemenet
			\item J: a J bemenet
			\item K: a K bemenet
		\end{itemize}
	\item Példa: \texttt{jk = FFJK(seqgen1, and1, and2)}
	\end{itemize}

\item \texttt{FFD(clk, D)}
	\begin{itemize}
	\item Leírás: D flip-flop létrehozása.
	\item Paraméterek: 
		\begin{itemize}
			\item clk: órajel bemenet
			\item D: a D bemenet
		\end{itemize}
	\item Példa: \texttt{ffd = FFD(seqgen1, or1)}
	\end{itemize}
	
\item \texttt{LED(in)}
	\begin{itemize}
	\item Leírás: LED létrehozása.
	\item Paraméterek: 
		\begin{itemize}
			\item in: led bemenete
		\end{itemize}
	\item Példa: \texttt{led = LED(kapcs1)}
	\end{itemize}

\item \texttt{7SEG(seg1, seg2, seg3, seg4, seg5, seg6, seg7)}
	\begin{itemize}
	\item Leírás: 7 szegmenses kijelző létrehozása.
	\item Paraméterek: 
		\begin{itemize}
			\item seg1..seg7: szegmensek bemenetei
		\end{itemize}
	\item Példa: \texttt{display = 7SEG(seg1, seg2, seg3, seg4, seg5, seg6, seg7)}
	\end{itemize}
	
\item \texttt{TOGGLE()}
	\begin{itemize}
	\item Leírás: kapcsoló létrehozása.
	\item Példa: \texttt{t = TOGGLE()}
	\end{itemize}		

\item \texttt{SEQGEN(seq)}
	\begin{itemize}
	\item Leírás: szekvencia generátor létrehozása.
	\item Paraméterek: 
		\begin{itemize}
			\item seq: meg kell adni egy sorozatot, melyet a generátor egymás után kiad.
		\end{itemize}
	\item Példa: \texttt{seqgen = SEQGEN(011000110)}
	\end{itemize}

\item \texttt{NODE(in, n)}
	\begin{itemize}
	\item Leírás: csomópont létrehozása.
	\item Paraméterek: 
		\begin{itemize}
			\item in: a csomópont bemenete
			\item n: csomópont kimeneteinek a száma
		\end{itemize}
	\item Példa: \texttt{node = NODE(kapcs1, 3)} - három kimenetű csomópontot hoz létre, melynek bemenete a kapcs1.
	\end{itemize}
	
\item \texttt{SCOPE(in, size)}
\begin{itemize}
	\item Leírás: oszcilloszkóp létrehozása.
	\item Paraméterek: 
		\begin{itemize}
			\item in: az oszcilloszkóp bemenete
			\item size: az oszcilloszkóp által tárolható értékek száma
		\end{itemize}
	\item Példa: \texttt{scope = SCOPE(kapcs1, 3)} - 3 értéket eltároló oszcilloszkóp, melynek bemenetére a kapcs1 nevű komponens van kötve.
\end{itemize}

\end{itemize}

Kompozit definiálása:\\
A kompozitok definiálása lényegesen eltér az előzőektől, hiszen ott le kell írni a kompozit belsejét, majd erre a hálózatban hivatkozni lehet, ahogy a többi komponensre. Példa a kompozitra:
\begin{verbatim}
composite MY_COMPOSITE( x, y, z ) {

	and = AND(x, y, z)
	or = OR(and, y)

} (or)
\end{verbatim}

A fenti példa egy olyan kompozitot ír le, melynek 3 bemenete van: x, y és z. Ezeket a bemeneteket egy 3-bemenetű ÉS és egy 2-bemenetű VAGY kapuban használjuk fel. A kompozitnak egy kimenete van, ezen pedig a VAGY kapu kimenete fog látszódni. Példa egy ilyen kompozit használatára:
\begin{verbatim}
	toggle = TOGGLE()
	comp = MY_COMPOSITE(toggle, 1, 1)
	led = LED(comp)
\end{verbatim}
	
Példa egy áramkör leíró fájlra:\\

Egy olyan minta hálózatot hozunk létre melyben található két kapcsoló egy és kapura kötve és az és kapu kimenete egy inverteren keresztül egy ledre kapcsolódik.
\begin{verbatim}
  composite MY_COMPOSITE( x, y, z ) {

    and = AND(x, y, z)
    or = OR(and, y)

  } (or)

  t1 = TOGGLE()
  t2 = TOGGLE()
  comp = MY_COMPOSITE(t1, t2, 1)
  led = LED(comp)
\end{verbatim}	

\subsubsection{A konfigurációs fájl nyelvtana}
\label{sec:settings_def}

A konfigurációs fájl minden sorában egy jelforrás beállítása szerepel, mely a következő egységekből áll:
\begin{itemize}
	\item az elem neve
	\item egyenlőségjel
	\item az elem értéke (szekvencia generátor esetében a bitszekvenciát egybe kell megadni, lásd példa)
\end{itemize}

példa:
\begin{verbatim}
toggle1 = 0
seqGen1 = 01101
\end{verbatim}

\subsection{Kimeneti nyelv}
\label{sec:output}

A program történései, visszajelzése a standard kimeneten jelennek meg, illetve ezek fájlba is kiíródhatnak. A program minden parancs után visszajelzést ad a felhasználónak a végrehajtott eseményről. A fentebb definiált parancsokra a következő jelzéseket kapja a felhasználó:\newline

\textit{loadCircuit <file>}\newline
Lehetséges kimenetek
\begin{itemize}
	\item
	\begin{verbatim}
	load successful
	\end{verbatim}
	\begin{itemize}
		\item Leírás: a betöltés sikeres, amennyiben az áramkört tartalmazó fájl szintaktikája megfelel \aref{sec:circuit_def} alfejezetben leírtaknak.
	\end{itemize}
	\item 
	\begin{verbatim}
	load failed
	\end{verbatim}
	\begin{itemize}
		\item Leírás: a betöltés sikertelen, amennyiben az áramkört tartalmazó fájl szintaktikája nem felel meg \aref{sec:circuit_def} alfejezetben leírtaknak.
	\end{itemize}
\end{itemize}

\textit{loadSettings <file>}\newline
Lehetséges kimenetek
\begin{itemize}
	\item
	\begin{verbatim}
	load successful
	\end{verbatim}
	\begin{itemize}
		\item Leírás: az értékek betöltése sikeres, amennyiben a konfigurációs fájlban szereplő áramköri elemek megfeleltethetők az aktuális áramkörben szereplő elemekkel, illetve a megadott értékek helyesek.
		\item Megjegyzés: azon elemek, melyek beállítására nem volt információ a konfigurációs fájlban automatikusan nullázódnak.
	\end{itemize}
	\item
	\begin{verbatim}
	load failed
	\end{verbatim}
	\begin{itemize}
		\item Leírás: az értékek betöltése sikeres, amennyiben a konfigurációs fájlban szereplő áramköri elem nem feleltethető meg az aktuális áramkörben szereplő elemek egyikével sem, illetve ha valamelyik érték helytelen.
	\end{itemize}
\end{itemize}

\textit{saveSettings <file>}\newline
Lehetséges kimenetek
\begin{itemize}
	\item
	\begin{verbatim}
	save successful
	\end{verbatim}
	\begin{itemize}
		\item Leírás: a konfigurációs értékek sikeresen fájlba mentődtek.
	\end{itemize}
\end{itemize}

\textit{switch <név>}\newline
Lehetséges kimenetek
\begin{itemize}
	\item
	\begin{verbatim}
	<név>: <érték>
	\end{verbatim}
	\begin{itemize}
		\item Leírás: az [elem] megadja a módosított kapcsoló nevét, míg az [érték] megmutatja, hogy milyen értékre változott az aktuális kapcsoló kimenete. 
	\end{itemize}
\end{itemize}

\textit{setSeqGen <név> <bit1>[<bit2><bit3>...]}\newline
Lehetséges kimenetek
\begin{itemize}
	\item 
	\begin{verbatim}
	<név>: <bit1>[<bit2><bit3>...]
	\end{verbatim}
	\begin{itemize}
		\item Leírás: az [elem] megadja a módosított generátor nevét, míg az [érték1, érték2, ...] megmutatja, hogy milyen értékekre változott az aktuális generátor kimenete. 
	\end{itemize}
\end{itemize}

\textit{check <név> | -all}\newline
Lehetséges kimenetek
\begin{itemize}
	\item
	\begin{verbatim}
<név>:
  in: <in1> [, <in2>, ...]
  out: <out1> [, <out2>, ...]
	\end{verbatim}
	\begin{itemize}
		\item Leírás: kiírja a megadott elem ki és bemeneteit a fenti formátumban.
		\item Megjegyzés: a \texttt{check -all} parancsra az összes elemet kilistázza a megadott formában új sor karakterrel elválasztva
	\end{itemize}
\end{itemize}

\textit{step}\newline
Lehetséges kimenetek
\begin{itemize}
	\item
	\begin{verbatim}
	simulation successful
	<elem1>: <érték>
	<elem2>: <érték>
	...
	\end{verbatim}
	\begin{itemize}
		\item Leírás: a szimuláció sikeres, amennyiben véges lépésen belül stabilizálódni tud az áramkör. Ekkor a kapcsoló(k), szekvenciagenerátor(ok) és a megjelenítő(k) értéke(i) kiíródnak.
	\end{itemize}
	\item
	\begin{verbatim}
	simulation failed
	\end{verbatim}
	\begin{itemize}
		\item Leírás: a szimuláció sikertelen, amennyiben véges lépésen belül nem tud stabilizálódni az áramkör.
	\end{itemize}
\end{itemize}

\newpage

\section{Összes részletes use-case}

%\begin{figure}[h]
%\begin{center}
%\includegraphics[width=17cm]{chapters/chapter07/example.pdf}
%\caption{x}
%\label{fig:ProtoUseCase}
%\end{center}
%\end{figure}

\usecase{Áramkör betöltése}{Az áramkört leíró fájl betöltése}{Felhasználó}{A loadCircuit parancsot használva betöltheti az
áramkört leíró fájlt, amely a program követelményeinek megfelel}

\usecase{Konfiguráció betöltése}{Egy áramkör konfigurációjának betöltése}{Felhasználó}{A loadSettings paranccsal betölt egy egyedi a konfigurációt 
az áramkörhöz, amely például tartalmazhatja a szekvencia generátorok által kiadott bitsorozatokat vagy a kapcsolók állását.}

\usecase{Konfiguráció mentése}{Áramkör konfigurációjának mentése}{Felhasználó}{A saveSettings parancs kiadásával
 menti az aktuális áramkör konfigurációját.}

\usecase{Kapcsoló kapcsolása}{Kapcsoló állásnak módosítás}{Felhasználó}{Az adott áramkörben a neve alapján azonosított kapcsoló állásának
módosítása a switch parancs használatával.}

\usecase{Szekvenciagenerátor módosítás}{Szekvenciagenerátor bitsorozatának megadása}{Felhasználó}{Az adott áramkörben a neve alapján 
azonosított szekvenciagenerátor által kiadott bitsorozat megadása a setSeqGen paranccsal.}

\usecase{Elem vizsgálata}{Egy, az áramkörben lévő alkatrész vizsgálata}{Felhasználó}{Az adott áramkörben a check parancs használatával a megadott nevű alkatrész be- és kimeneteinek lekérdezése.}

\usecase{Áramkör szimulálása}{A betöltött áramkör szimulálása}{Felhasználó}{A step parancs kiadásával szimulálja 
a betöltött áramkört.}

\usecase{Teszt eredményének ellenőrzése}{A program által generált kimenetet összehasonlítja a referencia kimenettel}{Felhasználó}{A teszt lefutását követően egy script összehasonlítja a kapott eredményeket a várt eredményekkel.}

\section{Tesztelési terv}
A tesztelés lehetőséget nyújt a program funkcióinak és menetének széleskörű vizsgálatára. Tesztelés során lehetőség nyílik a különböző tesztesetek kipróbálására. A teszt bemenetét bemeneti állományokból kapja, és a teszt eredményét kimeneti fájlban, és konzolon jeleníti meg. Ezáltal lehet összevetni a kiválasztott teszt várt, és tényleges eredményét.

\teszteset{Alap áramkör}{Alap áramkör, mely kapcsolókból, és egyszerű nem visszacsatolt ÉS kapukból áll, kimeneti értékeket ledek jelzik}{Ez a teszteset leteszteli a kapcsoló, az ÉS kapu, és a led működését.}

\teszteset{MPX-es áramkör}{Olyan áramkör, mely kapcsolókból és MPX-ből áll, a kimeneti értékeket 7 szegmenses kijelző jelzi.}{Ez a teszteset leteszteli a MPX és a 7szegmenses kijelző működését.}

\teszteset{Visszacsatolt stabil áramkör}{Olyan áramkör, melyben egy visszacsatolás történik, de az áramkör stabil marad, tehát egy VAGY kapu visszakötve, kimenet értékét egy led jelzi.}{Ez a teszteset első sorban a visszacsatolás helyes működését teszteli le, de mivel visszacsatolás is történik és egyúttal a kimeneti értéket is megjelenítjük leden, ezért csomópont is kell; így másodsorban a csomópont elem helyes működését is teszteli.}

\teszteset{Visszacsatolt nem stabil áramkör}{Olyan áramkör, melyben egy visszacsatolás történik, de az áramkör nem lesz stabil, mert egy visszacsatolt invertert tartalmaz.}{Ez a teszteset azt teszteli, hogy a szimuláció ilyen esetben leáll és ezt jelzi a felhasználónak.}

\teszteset{Flip-flop-os áramkör}{Olyan áramkör melyben szerepel egy flipflop, egy jelgenerátor, és egy oszcilloszkóp, melyre a flipflop kimenetét kötjük}{E teszteset során letesztelhetjük a jelgenerátor helyes működését és megvizsgálhatjuk, hogy a flip-flop helyesen lép-e felfutó élre, illetve helyesen működik-e, valamint az oszcilloszkóp helyesen tárolja-e az értékeket.}

\teszteset{Kompozitos áramkör}{Egy olyan áramkör, melyben szerepel egy kompozit elem, mely egy egyszerű áramköri hálózat tartalmaz.}{Ez a teszteset a kompozit elem működését teszteli le, annak helyességét ellenőrizhetjük.}

\teszteset{Kompoziton belüli kompozitos áramkör}{Előző áramkörhöz hasonló áramkör, a különbség az, hogy a kompoziton belüli áramköri hálózat tartalmaz egy újabb kompozitot a többi elemen kívül.}{Ez az áramkör leteszteli, hogyan működik a program olyan esetben, mikor a kompozit további kompozitot tartalmaz, illetve a kompoziton belüli kompozit jól működik-e más elemekkel összekötve.}

\section{Tesztelést támogató segéd- és fordítóprogramok specifikálása}
A program által generált kimeneti fájl és az elvárt eredményeket tartalmazó fájlok összehasonlítására 
a DiffUtils-ban (http://www.gnu.org/software/diffutils/) található cmp.exe-t fogjuk használni.
