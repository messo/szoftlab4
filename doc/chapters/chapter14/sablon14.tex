% Szglab4
% ===========================================================================
%
\chapter{Összefoglalás}

\thispagestyle{fancy}

\section{Projekt összegzés}
\comment{A projekt tapasztalatait összegző részben a csapatoknak a projektről kialakult véleményét várjuk. A megválaszolandók köre az alábbi. }

\begin{munka}
\munkaido{Apagyi Gábor}{23}
\munkaido{Dévényi Attila}{35}
\munkaido{Jákli Gábor}{37}
\munkaido{Kriván Bálint}{61}
\munkaido{Péter Tamás Pál}{22.5}
\osszesmunkaido{178.5}
\end{munka}

\begin{forrassor}
\munkaido{Szkeleton}{500}
\munkaido{Protó}{600}
\munkaido{Grafikus}{700}
\end{forrassor}

\begin{itemize}
\item Mit tanultak a projektből konkrétan és általában? \newline

Végre gyakorlatban is láthattuk azt, amit az előző félévben a Szoftvertechnológia tárgyból tanultunk. Kipróbálhattuk magunkat egy viszonylag hosszabb projektmunkában, megismerhettük egymás erősségeit és gyengeségeit. 

\item Mi volt a legnehezebb és a legkönnyebb? \newline

Legnehezebb a közös időpontok megszervezése, valamint a közös tervezések közben felmerült problémák megoldásainak közös elfogadása. A legkönnyebb az implementálás volt az elkészült tervek alapján, mely szinte csak gépelésről szólt.

\item Összhangban állt-e az idő és a pontszám az elvégzendő feladatokkal? \newline

Az első részben kicsit tévútra indultunk és ott fölösleges órák mentek el a rossz megoldásra, de ezt a negyedik beadásra sikerült korrigálni. Összességben korrektnek érezzük a kapott pontszámokat.

\item Ha nem, akkor hol okozott ez nehézséget? \newline

Ahogy már említettük az első részben az analízis modell rosszra sikeredett, ezt sok idő volt korrigálni, és maga a rossz megoldás is sok időt elvett.

\item Milyen változtatási javaslatuk van? \newline

Több évfolyamtárstól hallottuk, hogy labvezérek által számonkért beadandó dokumentumok minőségének a szórása igen nagy. Mi úgy érezzük, hogy beletettünk elég sok munkát és ennek megfelelően jogosnak érezzük az elért eredményeket, pontszámokat, azonban néhány másik csoport ennek töredékéért megkapja ugyanazt, vagy akár többet, annak ellenére, hogy nem biztos, hogy jobb minőségű munkát adtak ki a kezükből. Ha ezen egy kicsit lehetne javítani, homogenizálni a labvezérek egyéni elvárásait, akkor talán kicsit jobb lenne a több munkát belefektető csapatok hangulata. Továbbá ha megnézzük az eltöltött órák számát, akkor kicsit kevésnek érezzük a tárgyért járó 2 kreditet, 3-nak esetleg 4-nek jobban örülnénk.

\item Milyen feladatot ajánlanának a projektre? \newline

Mindenképpen valami ehhez hasonlót, tehát a projekt végére egy ,,használható'' termék-szerűség készüljön el, amit akár valódi célokra is fel lehet használni (gondolok itt arra, hogy például a most elkészült programot a Digitális technika I-II. című tárgy keretében akár még használni is lehetne).

\end{itemize}
