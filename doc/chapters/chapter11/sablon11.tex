% Szglab4
% ===========================================================================
%
\chapter{Grafikus felület specifikációja}

\thispagestyle{fancy}

\section{A grafikus interfész}

\Aref{fig:main}. ábra mutatja a főablakot, a benne lévő áramkör csak illusztráció. A két menü almenüi \aref{fig:menus}. ábrán látszódnak. A Fájl menü almenüi beszédesek, a felső három menüpontra megnyílik egy fájlválasztó ablak, ahol megadható egy fájl, majd az adott akció lefut. A Kilépés menüpont segítségével kiléphetünk az alkalmazásból. Az Egyéb menü Néjegy menüpontjára kapcsolva pedig megnyílik \aref{fig:about}. ábrán látható ablak.

\begin{figure}[h]
\begin{center}
\includegraphics[width=4.25in]{chapters/chapter11/screenshots/felulet.png}
\caption{Főablak}
\label{fig:main}
\end{center}
\end{figure}

\begin{figure}[h]
\begin{center}
\includegraphics[width=1.98in]{chapters/chapter11/screenshots/menus1.png}
\includegraphics[width=0.83in]{chapters/chapter11/screenshots/menus2.png}
\caption{Fájl és az Egyéb menü almenüi}
\label{fig:menus}
\end{center}
\end{figure}

\begin{figure}[h]
\begin{center}
\includegraphics[width=2.8125in]{chapters/chapter11/screenshots/about.png}
\caption{Névjegy}
\label{fig:about}
\end{center}
\end{figure}

\section{A grafikus rendszer architektúrája}
\comment{A felület működésének elve, a grafikus rendszer architektúrája (struktúra diagramok). A struktúra diagramokon a prototípus azon és csak azon osztályainak is szerepelnie kell, amelyekhez a grafikus felületet létrehozó osztályok kapcsolódnak.}

\subsection{A felület működési elve}
\comment{Le kell írni, hogy a grafikai megjelenésért felelős osztályok, objektumok hogyan kapcsolódnak a meglevő rendszerhez, a megjelenítés során mi volt az alapelv. Törekedni kell az MVC megvalósításra. Alapelvek lehetnek: \textbf{push} alapú: a modell értesíti a felületet, hogy változott; \textbf{pull} alapú: a felület kérdezi le a modellt, hogy változott-e; \textbf{kevert}: a kettő kombinációja.}

Az általunk elkészített grafikus felület "pull" típusú, vagyis a grafikus rendszer kérdezi le a modell objektumoktól az aktuális állapotukat.
Azokhoz a modellobjektumokhoz, melyeket megjelenítünk, elkészítettünk egy-egy wrapper osztályt, mely a megjelenítésért és a megjelenítéshez szükséges információk tárolásáért felel. 
Az áramkört egy JPanel-ra rajzoljuk, mely biztosítja számunkra, hogy az elhelyezhető legyen bármilyen ablakon. Áramkör újrarajzoláskor, az eltárolt objektumok egyenkét rajzolják ki magukat az előzöleg megadott koordináták alapján.
Bármilyen felhasználói interakciónál, melynél változhat az áramkör állapota, az egész áramkört újrarajzoljuk, biztosítva ezzel, hogy a kirajzolt áramkör mindig az aktuális állapotban legyen megjelenítve.



\subsection{A felület osztály-struktúrája}
\comment{Osztálydiagram. Minden új osztály, és azon régiek, akik az újakhoz közvetlenül kapcsolódnak.}

\section{A grafikus objektumok felsorolása}
\comment{Az új osztályok felsorolása. Az régi osztályok közül azoknak a felsorolása, ahol változás volt. Ezek esetén csak a változásokat kell leírni.}

\subsection{Osztály1}
\begin{itemize}
\item Felelősség\newline
\comment{Mi az osztály felelőssége. Kb 1 bekezdés. Ha szükséges, akkor state-chart is.}
\item Ősosztályok\newline
\comment{Mely osztályokból származik (öröklési hierarchia)\newline
Legősebb osztály $\rightarrow$ Ősosztály2 $\rightarrow$ Ősosztály3...}
\item Interfészek\newline
\comment{Mely interfészeket valósítja meg.}
\item Attribútumok\newline
\comment{Milyen attribútumai vannak}
	\begin{itemize}
		\item attribútum1: attribútum jellemzése: mire való, láthatósága (UML jelöléssel), típusa
		\item attribútum2: attribútum jellemzése: mire való, láthatósága (UML jelöléssel), típusa
	\end{itemize}
\item Metódusok\newline
\comment{Milyen publikus, protected és privát  metódusokkal rendelkezik. Metódusonként precíz leírás, ha szükséges, activity diagram is  a metódusban megvalósítandó algoritmusról.}
	\begin{itemize}
		\item int foo(Osztály3 o1, Osztály4 o2): metódus leírása, láthatósága (UML jelöléssel)
		\item int bar(Osztály5 o1): metódus leírása, láthatósága (UML jelöléssel)
	\end{itemize}
\end{itemize}

\subsection{Osztály2}
\begin{itemize}
\item Felelősség\newline
\comment{Mi az osztály felelőssége. Kb 1 bekezdés. Ha szükséges, akkor state-chart is.}
\item Ősosztályok\newline
\comment{Mely osztályokból származik (öröklési hierarchia)\newline
Legősebb osztály $\rightarrow$ Ősosztály2 $\rightarrow$ Ősosztály3...}
\item Interfészek\newline
\comment{Mely interfészeket valósítja meg.}
\item Attribútumok\newline
\comment{Milyen attribútumai vannak}
	\begin{itemize}
		\item attribútum1: attribútum jellemzése: mire való, láthatósága (UML jelöléssel), típusa
		\item attribútum2: attribútum jellemzése: mire való, láthatósága (UML jelöléssel), típusa
	\end{itemize}
\item Metódusok\newline
\comment{Milyen publikus, protected és privát  metódusokkal rendelkezik. Metódusonként precíz leírás, ha szükséges, activity diagram is  a metódusban megvalósítandó algoritmusról.}
	\begin{itemize}
		\item int foo(Osztály3 o1, Osztály4 o2): metódus leírása, láthatósága (UML jelöléssel)
		\item int bar(Osztály5 o1): metódus leírása, láthatósága (UML jelöléssel)
	\end{itemize}
\end{itemize}

\section{Kapcsolat az alkalmazói rendszerrel}
\comment{Szekvencia-diagramokon ábrázolni kell a grafikus rendszer működését. Konzisztens kell legyen az előző alfejezetekkel. Minden metódus, ami ott szerepel, fel kell tűnjön valamelyik szekvenciában. Minden metódusnak, ami szekvenciában szerepel, szereplnie kell a valamelyik osztálydiagramon.}

\begin{figure}[h]
\begin{center}
\includegraphics[width=17cm]{chapters/chapter11/pdfs/1_program_start.pdf}
\caption{Program indítása}
\label{fig:program_start}
\end{center}
\end{figure}

\begin{figure}[h]
\begin{center}
\includegraphics[width=17cm]{chapters/chapter11/pdfs/2_loadcircuit.pdf}
\caption{Áramkör betöltése}
\label{fig:loadcircuit}
\end{center}
\end{figure}