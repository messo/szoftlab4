\subsection{AbstractComponent (vált.)}
Absztrakt osztály.
\begin{itemize}
\item Felelősség\\
Egy komponens absztrakt megvalósítása, ebből származik az összes többi  komponens. A közös logikát valósítja meg. A gyakran használt dolgokra  ad alapértelmezett implementációt (kimenetekre és bemenetekre kötés, kiértékelés stb.)
\item Ősosztályok\ Object $\rightarrow{}$ AbstractComponent.
\item Interfészek (nincs)
\item Attribútumok $\ $
\item Metódusok$\ $
\begin{itemize}
	\item[] \texttt{$+$ ComponentView createView(ComponentViewCreator cvc)}: Lekérjük a komponenst ábrázoló viewt, de a tényleges rajzolást nem mi végezzük, hanem  a ComponentViewCreator, kihasználva a Visitor tervezési mintát.
\end{itemize}
\end{itemize}

\subsection{Composite (vált.)}
\begin{itemize}
\item Felelősség\\
Kompozit elem leírása, kiértékelésnél a tartalmazott komponenseket kiértékeli, lépteti  a jelgenerátorokat stb. Ha nem áll be stacionárius állapotba a kiértékelésnél, akkor ezt jelzi kifelé.
\item Ősosztályok\ Object $\rightarrow{}$ AbstractComponent $\rightarrow{}$ Composite.
\item Interfészek (nincs)
\item Attribútumok $\ $
\item Metódusok$\ $
\begin{itemize}
	\item[] \texttt{$+$ ComponentView createView(ComponentViewCreator cvc)}: 
\end{itemize}
\end{itemize}

\subsection{Wire (vált.)}
\begin{itemize}
\item Felelősség\\
Vezeték osztály. Két komponens-lábat köt össze. A rajta lévő érték lekérdezhető  és beállítható.
\item Ősosztályok\ Object $\rightarrow{}$ Wire.
\item Interfészek (nincs)
\item Attribútumok $\ $
\item Metódusok$\ $
\begin{itemize}
	\item[] \texttt{$+$ WireView createView(ComponentViewCreator cvc, Point start, Point end)}: Lekérjük a vezetéket ábrázoló viewt, de a tényleges rajzolást nem mi végezzük, hanem  a ComponentViewCreator, kihasználva a Visitor tervezési mintát.
\end{itemize}
\end{itemize}

