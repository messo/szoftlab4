\subsection{CircuitView}
\begin{itemize}
\item Felelősség\\
Áramkört kirajzoló panel.
\item Ősosztályok\ JPanel $\rightarrow{}$ CircuitView.
\item Interfészek MouseListener.
\item Attribútumok $\ $
\begin{itemize}
	\item[] \texttt{$-$ List drawables}: kirajzolandók listája
	\item[] \texttt{$-$ FrameView parent}: főablak
	\item[] \texttt{$-$ Map positions}: kirajzolandók pozíciója
\end{itemize}
\item Metódusok$\ $
\begin{itemize}
	\item[] \texttt{$+$ CircuitView()}: Áramkört kirajzoló panel
	\item[] \texttt{$+$ void mouseClicked(MouseEvent me)}: Egérkattintás kezelő
	\item[] \texttt{$+$ void paint(Graphics g)}: Áramkör kirajzolása
	\item[] \texttt{$+$ void refresh()}: Áramkör újrarajzolása.
	\item[] \texttt{$+$ void setParent(FrameView parent)}: Szülő beállítása
	\item[] \texttt{$+$ void updateDrawables(List drawables, Map coords)}: Kirajzolandó objektumok és koordinátáik beállítása
\end{itemize}
\end{itemize}

\subsection{Drawable}
Interfész.
\begin{itemize}
\item Felelősség\\
Áramköri panelre rajzolható objektum.
\item Ősosztályok\ Drawable.
\item Interfészek (nincs)
\item Metódusok$\ $
\begin{itemize}
	\item[] \texttt{$+$ void draw(Graphics g)}: Kirajzolási logika
	\item[] \texttt{$+$ Dimension getDimension()}: Lekérhetjük az objektumtól a méretét, ha beszélhetünk ilyenről.
	\item[] \texttt{$+$ void onClick(Controller controller)}: Komponensre kapcsolás logikája (visszahívhat a vezérlőre)
\end{itemize}
\end{itemize}

\subsection{Frame}
\begin{itemize}
\item Felelősség\\
Alkalmazás főablaka. Ő tartalmazza a CircuitView-t és a menüsort valamint a gombokat.
\item Ősosztályok\ JFrame $\rightarrow{}$ Frame.
\item Interfészek ActionListener, FrameView.
\item Attribútumok $\ $
\begin{itemize}
	\item[] \texttt{$-$ CircuitView circuitView} 
	\item[] \texttt{$-$ Controller controller}: vezérlő
	\item[] \texttt{$-$ Timer t}: időzítő
\end{itemize}
\item Metódusok$\ $
\begin{itemize}
	\item[] \texttt{$+$ Frame(Controller controller)}: Kontruktor
	\item[] \texttt{$-$ void aboutCloseBtnActionPerformed(ActionEvent evt)}: Névjegy ablak bezárása
	\item[] \texttt{$-$ void aboutMIActionPerformed(ActionEvent evt)}: Névjegy menüpont eseményvezérlője
	\item[] \texttt{$+$ void actionPerformed(ActionEvent e)}: Timer tick eventje
	\item[] \texttt{$-$ void closeDetailedBTNActionPerformed(ActionEvent evt)}: Komponens részletei ablak bezárása
	\item[] \texttt{$+$ void drawCircuit()}: Áramkör kirajzolása
	\item[] \texttt{$-$ void exitMIActionPerformed(ActionEvent evt)}: Kilépés menüpont eseményvezérlője
	\item[] \texttt{$+$ Controller getController()}: Lekérdezhető a vezérlő
	\item[] \texttt{$-$ void loadCircuitMIActionPerformed(ActionEvent evt)}: Áramkör betöltése menüpont eseményvezérlője
	\item[] \texttt{$-$ void loadConfigMIActionPerformed(ActionEvent evt)}: Konfig fájl betöltése menüpont eseményvezérlője
	\item[] \texttt{$+$ void makeItVisible()}: Megjelenítés
	\item[] \texttt{$+$ void onFailedSimulation()}: Áramkör szimulációja nem sikerült
	\item[] \texttt{$+$ void onSuccessfulSimulation()}: Áramkör szimulációja sikeres
	\item[] \texttt{$-$ void saveConfigMIActionPerformed(ActionEvent evt)}: Konfig fájl mentése menüpont eseményvezérlője
	\item[] \texttt{$-$ void saveSeqBTNActionPerformed(ActionEvent evt)}: Új szekvencia elmentése
	\item[] \texttt{$+$ void setDrawables(List drawables, Map positions)}: Megjelenítendő objektumok és koordinátáik átadása a megjelenítőnek
	\item[] \texttt{$+$ void setPeriod(int pt)}: Szimuláció sebességének beállítása
	\item[] \texttt{$+$ void showDetails(AbstractComponent ac)}: Általános komponens részleteinek megjelenítése
	\item[] \texttt{$+$ void showDetails(Scope s)}: Scope részleteinek megjelenítése
	\item[] \texttt{$+$ void showDetails(SequenceGenerator sg)}: Szekvenciagenerátor részleteinek megjelenítése
	\item[] \texttt{$-$ void simSpeedSaveBtnActionPerformed(ActionEvent evt)}: Új sebesség mentése
	\item[] \texttt{$-$ void simulationDelayActionPerformed(ActionEvent evt)}: Szimuláció sebességének beállítására szolgáló ablak megjelenítése
	\item[] \texttt{$-$ void StartStopActionPerformed(ActionEvent evt)}: Szimuláció start/stop
	\item[] \texttt{$-$ void stepBtnActionPerformed(ActionEvent evt)}: Léptetés gomb eseményvezérlője
\end{itemize}
\end{itemize}

\subsection{FrameView}
Interfész.
\begin{itemize}
\item Felelősség\\
Főablak interfésze
\item Ősosztályok\ FrameView.
\item Interfészek (nincs)
\item Metódusok$\ $
\begin{itemize}
	\item[] \texttt{$+$ void drawCircuit()}: Kirajzoljuk az áramkört.
	\item[] \texttt{$+$ Controller getController()}: Lekérdezzük a vezérlőt
	\item[] \texttt{$+$ void makeItVisible()}: Itt kell megadni, hogy a főablak, hogy tehető láthatóvá.
	\item[] \texttt{$+$ void onFailedSimulation()}: Itt adható meg, hogy mi történjen, ha nem stabil az áramkör
	\item[] \texttt{$+$ void onSuccessfulSimulation()}: Itt adható meg, hogy mi történjen, ha sikeres egy szimulációs lépés
	\item[] \texttt{$+$ void setDrawables(List drawables, Map positions)}: Beállítjuk a kirajzolandó objektumokat és azok pozícióját.
	\item[] \texttt{$+$ void setPeriod(int pt)}: Szimuláció sebességének beállítása
	\item[] \texttt{$+$ void showDetails(AbstractComponent ac)}: 
Általános komponens részleteinek megjelenítése
	\item[] \texttt{$+$ void showDetails(Scope s)}: 
Scope részleteinek megjelenítése
	\item[] \texttt{$+$ void showDetails(SequenceGenerator sg)}: 
Szekvenciagenerátor részleteinek megjelenítése
\end{itemize}
\end{itemize}

