\subsection{AndGate (vált.)}
\begin{itemize}
\item Felelősség\\
ÉS kapu, az áramkör egyik alapeleme. Bemeneteire kötött komponensek  kiértékelését kezdeményezi, s a kapott értékek logikai ÉS kapcsolatát  valósítja meg, amit a kimenetén kiad.
\item Ősosztályok\ Object $\rightarrow{}$ AbstractComponent $\rightarrow{}$ AndGate.
\item Interfészek (nincs)
\item Attribútumok $\ $
\begin{itemize}
\item (nincs)
\end{itemize}
\item Metódusok$\ $
\begin{itemize}
	\item[] \texttt{$+$ ComponentView createView(ComponentViewCreator cvc)}: 
\end{itemize}
\end{itemize}

\subsection{FlipFlopD (vált.)}
\begin{itemize}
\item Felelősség\\
D flipflop, mely felfutó órajelnél beírja a belső memóriába az adatbemeneten (D)  lévő értéket.
\item Ősosztályok\ Object $\rightarrow{}$ AbstractComponent $\rightarrow{}$ FlipFlop $\rightarrow{}$ FlipFlopD.
\item Interfészek (nincs)
\item Attribútumok $\ $
\item Metódusok$\ $
\begin{itemize}
	\item[] \texttt{$+$ ComponentView createView(ComponentViewCreator cvc)}: 
\end{itemize}
\end{itemize}

\subsection{FlipFlopJK (vált.)}
\begin{itemize}
\item Felelősség\\
JK flipflop, mely a belső memóriáját a Követelmények résznél leírt módon  a J és K bemenetektől függően változtatja.
\item Ősosztályok\ Object $\rightarrow{}$ AbstractComponent $\rightarrow{}$ FlipFlop $\rightarrow{}$ FlipFlopJK.
\item Interfészek (nincs)
\item Attribútumok $\ $
\item Metódusok$\ $
\begin{itemize}
	\item[] \texttt{$+$ ComponentView createView(ComponentViewCreator cvc)}: 
\end{itemize}
\end{itemize}

\subsection{Gnd (vált.)}
\begin{itemize}
\item Felelősség\\
A "föld" komponens, mely állandóan a hamis értéket adja ki. Nincs bemenete.
\item Ősosztályok\ Object $\rightarrow{}$ AbstractComponent $\rightarrow{}$ Gnd.
\item Interfészek (nincs)
\item Attribútumok $\ $
\begin{itemize}
\item (nincs)
\end{itemize}
\item Metódusok$\ $
\begin{itemize}
	\item[] \texttt{$+$ ComponentView createView(ComponentViewCreator cvc)}: 
\end{itemize}
\end{itemize}

\subsection{Inverter (vált.)}
\begin{itemize}
\item Felelősség\\
Inverter alkatrész, mely invertálva adja ki a kimenetén a bemenetén  érkező jelet.
\item Ősosztályok\ Object $\rightarrow{}$ AbstractComponent $\rightarrow{}$ Inverter.
\item Interfészek (nincs)
\item Attribútumok $\ $
\begin{itemize}
\item (nincs)
\end{itemize}
\item Metódusok$\ $
\begin{itemize}
	\item[] \texttt{$+$ ComponentView createView(ComponentViewCreator cvc)}: 
\end{itemize}
\end{itemize}

\subsection{Led (vált.)}
\begin{itemize}
\item Felelősség\\
Egy LED-et reprezentál, mely világít, ha bemenetén igaz érték van.
\item Ősosztályok\ Object $\rightarrow{}$ AbstractComponent $\rightarrow{}$ DisplayComponent $\rightarrow{}$ Led.
\item Interfészek (nincs)
\item Attribútumok $\ $
\begin{itemize}
\item (nincs)
\end{itemize}
\item Metódusok$\ $
\begin{itemize}
	\item[] \texttt{$+$ ComponentView createView(ComponentViewCreator cvc)}: 
\end{itemize}
\end{itemize}

\subsection{Mpx (vált.)}
\begin{itemize}
\item Felelősség\\
4-1-es multiplexer, melynek a bemeneti lábak sorrendje a következő:  D0, D1, D2, D3, S0, S1. Ahol Dx az adatbemenetek, Sy a kiválasztóbemenetek.  Kimenetén a kiválasztóbemenetektől függően valamelyik adatbemenet kerül kiadásra.
\item Ősosztályok\ Object $\rightarrow{}$ AbstractComponent $\rightarrow{}$ Mpx.
\item Interfészek (nincs)
\item Attribútumok $\ $
\item Metódusok$\ $
\begin{itemize}
	\item[] \texttt{$+$ ComponentView createView(ComponentViewCreator cvc)}: 
\end{itemize}
\end{itemize}

\subsection{Node (vált.)}
\begin{itemize}
\item Felelősség\\
Csomópont elem. Az egyetlen bemenetére kötött értéket kiadja az összes kimeneti lábán.
\item Ősosztályok\ Object $\rightarrow{}$ AbstractComponent $\rightarrow{}$ Node.
\item Interfészek (nincs)
\item Attribútumok $\ $
\begin{itemize}
\item (nincs)
\end{itemize}
\item Metódusok$\ $
\begin{itemize}
	\item[] \texttt{$+$ ComponentView createView(ComponentViewCreator cvc)}: 
\end{itemize}
\end{itemize}

\subsection{OrGate (vált.)}
\begin{itemize}
\item Felelősség\\
VAGY kapu, az áramkör egyik alapeleme. Bemenetein lévő értékek logikai VAGY kapcsolatát  valósítja meg, amit a kimenetén kiad.
\item Ősosztályok\ Object $\rightarrow{}$ AbstractComponent $\rightarrow{}$ OrGate.
\item Interfészek (nincs)
\item Attribútumok $\ $
\begin{itemize}
\item (nincs)
\end{itemize}
\item Metódusok$\ $
\begin{itemize}
	\item[] \texttt{$+$ ComponentView createView(ComponentViewCreator cvc)}: 
\end{itemize}
\end{itemize}

\subsection{Scope (vált.)}
\begin{itemize}
\item Felelősség\\
Egy oszcilloszkópot reprezentál. Eltárolt értékek egy sorba kerülnek bele, mely fix méretű.
\item Ősosztályok\ Object $\rightarrow{}$ AbstractComponent $\rightarrow{}$ DisplayComponent $\rightarrow{}$ Led $\rightarrow{}$ Scope.
\item Interfészek (nincs)
\item Attribútumok $\ $
\item Metódusok$\ $
\begin{itemize}
	\item[] \texttt{$+$ ComponentView createView(ComponentViewCreator cvc)}: 
\end{itemize}
\end{itemize}

\subsection{SequenceGenerator (vált.)}
\begin{itemize}
\item Felelősség\\
Jelgenerátort reprezentál, amely a beállított bitsorozatot adja ki.  Alapértelmezetten (amíg a felhasználó nem állítja be, vagy tölt be másikat) a 0,1-es  szekvenciát tárolja.
\item Ősosztályok\ Object $\rightarrow{}$ AbstractComponent $\rightarrow{}$ SourceComponent $\rightarrow{}$ SequenceGenerator.
\item Interfészek (nincs)
\item Attribútumok $\ $
\item Metódusok$\ $
\begin{itemize}
	\item[] \texttt{$+$ ComponentView createView(ComponentViewCreator cvc)}: 
\end{itemize}
\end{itemize}

\subsection{SevenSegmentDisplay (vált.)}
\begin{itemize}
\item Felelősség\\
7-szegmenses kijelzőt reprezentál, melynek 7 bemenete vezérli a  megfelelő szegmenseket, ezek világítanak, ha az adott bemenetre logikai  igaz van kötve.
\item Ősosztályok\ Object $\rightarrow{}$ AbstractComponent $\rightarrow{}$ DisplayComponent $\rightarrow{}$ SevenSegmentDisplay.
\item Interfészek (nincs)
\item Attribútumok $\ $
\begin{itemize}
\item (nincs)
\end{itemize}
\item Metódusok$\ $
\begin{itemize}
	\item[] \texttt{$+$ ComponentView createView(ComponentViewCreator cvc)}: 
\end{itemize}
\end{itemize}

\subsection{Toggle (vált.)}
\begin{itemize}
\item Felelősség\\
Kapcsoló jelforrás, melyet a felhasználó szimuláció közben kapcsolgathat.
\item Ősosztályok\ Object $\rightarrow{}$ AbstractComponent $\rightarrow{}$ SourceComponent $\rightarrow{}$ Toggle.
\item Interfészek (nincs)
\item Attribútumok $\ $
\item Metódusok$\ $
\begin{itemize}
	\item[] \texttt{$+$ ComponentView createView(ComponentViewCreator cvc)}: 
\end{itemize}
\end{itemize}

\subsection{Vcc (vált.)}
\begin{itemize}
\item Felelősség\\
A tápfeszültés komponens, ami konstans igaz értéket ad. Nincs bemenete.
\item Ősosztályok\ Object $\rightarrow{}$ AbstractComponent $\rightarrow{}$ Vcc.
\item Interfészek (nincs)
\item Attribútumok $\ $
\begin{itemize}
\item (nincs)
\end{itemize}
\item Metódusok$\ $
\begin{itemize}
	\item[] \texttt{$+$ ComponentView createView(ComponentViewCreator cvc)}: 
\end{itemize}
\end{itemize}

