% Szglab4
% ===========================================================================
%
\section{Napló}

\begin{naplo}

\bejegyzes
{2010.03.12.~14:00~}
{1,5 óra}
{Kriván B.}
{Javasolt módosítások elvégzése az előző fejezetben, rövid errate készítése jelen fejezet elé.}

\bejegyzes
{2010.03.13.~00:00~}
{2 óra}
{Péter T.}
{Use-casek leírása szöveges formátumban}

\bejegyzes
{2010.03.13.~09:30~}
{30 perc}
{Kriván B.}
{Use-case diagram megrajzolása}

\bejegyzes
{2010.03.13.~10:00~}
{2 óra}
{Kriván B.}
{Use-casek leírásának \LaTeX{} formátumra való alakítása, apróbb finomítások}

\bejegyzes
{2010.03.14.~16:00~}
{2 óra}
{Apagyi G.}
{Első 3 (\ref{fig:init}, \ref{fig:test1}, \ref{fig:test2}) szekvenciadiagram megrajzolása}

\bejegyzes
{2010.03.14.~18:00~}
{2 óra}
{Kriván B.}
{\ref{fig:init}, \ref{fig:test1}, és \ref{fig:test2} diagramok és az utolsó (\ref{fig:test5_1} és \ref{fig:test5_2}) szekvenciadiagram formázása, tömörítése, leírások frissítése}

\bejegyzes
{2010.03.14.~18:00~}
{3 óra}
{Dévényi A.}
{\ref{fig:test3_1} és \ref{fig:test3_2} diagram formázása, tömörítése és \aref{fig:test4_1} és \ref{fig:test4_2}) szekvenciadiagram megrajzolása leírások frissítése}

\bejegyzes
{2010.03.14.~18:00~}
{1 óra}
{Jákli G.}
{5.4-es fejezet megírása}

\end{naplo}

