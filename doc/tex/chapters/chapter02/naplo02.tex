% Szglab4
% ===========================================================================
%
\section{Napló}

\begin{naplo}

\bejegyzes
{2011.02.15.~21:30~}
{2 óra}
{Kriván B.\newline
Dévényi A.\newline
Jákli G.}
{Értekezlet.\newline
Döntések: Megegyeztünk a feladat értelme\-zését illetően.
Milyen kérdéseket teszünk fel a konzulensnek az első konzultáción?}

\bejegyzes
{2011.02.16.~09:00~}
{2 óra}
{Kriván B.\newline
Dévényi A.\newline
Apagyi G.\newline
Péter T.}
{Értekezlet.\newline
Döntések:
\begin{itemize}
\setlength{\itemsep}{0cm}%
\setlength{\parskip}{0cm}%
\item Fejlesztői környezetben megállapodtunk (\ref{sec:devenvironment})
\item Projekt szervezési struktúráját rögzítettük (\ref{sec:orgstructure})
\item A felmerülő algoritmusokról is konzultáltunk.
\end{itemize}
}

\bejegyzes
{2011.02.16.~15:00~}
{1 óra}
{Jákli G.}
{Tevékenység: Projekt terv bővítése, formázá\-sa, javítása (\ref{sec:projectplan})}

\bejegyzes
{2011.02.17.~16:00~}
{1 óra}
{Jákli G.\newline
Kriván B.\newline
Dévényi A.}
{Tevékenység: \aref{sec:reqdef} és \aref{sec:projectplan} alfejezet átdolgozása}

\bejegyzes
{2011.02.17.~19:15~}
{45 perc}
{Jákli G.\newline
Kriván B.\newline
Dévényi A.}
{Értekezlet.\newline
Döntések: \Aref{sec:taskdesc} alfejezet főbb gondolatait megfogalmaztuk, és meghatároztuk, hogy mik legyenek a mindenképpen lejegyezendő dolgok.}

\bejegyzes
{2011.02.17.~20:00~}
{50 perc}
{Jákli G.}
{Tevékenység: A megbeszéltek alapján el\-kezdte \aref{sec:taskdesc} alfejezet megírását.}


\end{naplo}

